{applications,approximation,discrete,namd,performance,programming}.tex\Level 0 {An introduction to binary files}

\begin{purpose}
  In this section you will be introduced to the different types of
  binary files that you encounter while programming.
\end{purpose}

One of the first things you become aware of when you start programming
is the distinction between the readable source code, and the
unreadable, but executable, program code. In this tutorial you will
learn about a couple more file types:
\begin{itemize}
\item A source file can be compiled to an \emph{object file}, which is
  a bit like a piece of an executable: by itself it does nothing, but
  it can be combined with other object files to form an executable.
\item A \emph{library} is a bundle of object files that can be used to
  form an executable. Often, libraries are written by an expert and
  contain code for specialized purposes such as linear algebra
  manipulations. Libraries are important enough that they can be
  commercial, to be bought if you need expert code for a certain purpose.
\end{itemize}
You will now learn how these types of files are created and used.

\Level 0 {Simple compilation}

\begin{purpose}
  In this section you will learn about executables and object files.
\end{purpose}

Let's start with a simple program that has the whole source in one
file.
\listing{One file: \n{hello.c}}{tutorials-public/compile/c/hello.c}
Compile this program with your favourite compiler; we will use \n{gcc}
in this tutorial, but substitute your own as desired. As a result of
the compilation, a file \n{a.out} is created, which is the executable.
\begin{verbatim}
%% gcc hello.c
%% ./a.out
hello world
\end{verbatim}
You can get a more sensible program name with the \n{-o} option:
\begin{verbatim}
%% gcc -o helloprog hello.c
%% ./helloprog
hello world
\end{verbatim}

Now we move on to a program that is in more than one source file.
\listing{Main program: \n{fooprog.c}}{tutorials-public/compile/c/fooprog.c}
\listing{Subprogram: \n{fooprog.c}}{tutorials-public/compile/c/foosub.c}
As before, you can make the program with one command.
\begin{verbatim}
%% gcc -o foo fooprog.c foosub.c 
%% ./foo
hello world
\end{verbatim}
However, you can also do it in steps, compiling each file separately
and then linking them together.
\begin{verbatim}
%% gcc -c fooprog.c
%% gcc -c foosub.c
%% gcc -o foo fooprog.o foosub.o
%% ./foo
hello world
\end{verbatim}
The \n{-c} option tells the compiler to compile the source file,
giving an \indexterm{object file}. The third command than acts as the
\indexterm{linker}, tieing together the object files into an
executable. (With programs that are spread over several files there is
always the danger of editing a subroutine definition and then
forgetting to update all the places it is used. See the `make'
tutorial, section~\ref{tut:gnumake}, for a way of dealing with this.) 

\Level 0 {Libraries}
\index{libraries!creating and using|(}

\begin{purpose}
  In this section you will learn about libraries.
\end{purpose}

If you have written some subprograms, and you want to share them with
other people (perhaps by selling them), then handing over individual
object files is inconvenient. Instead, the solution is to combine them
into a library. First we look at \indexterm{static libraries}, for
which the \indexterm{archive utility} \n{ar} is used. A~static library
is linked into your executable, becoming part of it. This may lead to
large executables; you will learn about shared libraries
next, which do not suffer from this problem.

Create a directory to contain your library (depending on what your
library is for this can be a
system directory such as \n{/usr/bin}), and create the library file
there. 
\begin{verbatim}
%% mkdir ../lib
%% ar cr ../lib/libfoo.a foosub.o
\end{verbatim}
The \n{nm} command tells you what's in the library:
\begin{verbatim}
%% nm ../lib/libfoo.a 

../lib/libfoo.a(foosub.o):
00000000 T _bar
         U _printf
\end{verbatim}
Line with \n{T} indicate functions defined in the library file;
a~\n{U} indicates a function that is used.

The library can be linked into your executable by explicitly giving
its name, or by specifying a library path:
\begin{verbatim}
%% gcc -o foo fooprog.o ../lib/libfoo.a
# or
%% gcc -o foo fooprog.o -L../lib -lfoo
%% ./foo
hello world
\end{verbatim}
A third possibility is to use the \n{LD_LIBRARY_PATH} shell
variable. Read the man page of your compiler for its use, and give the
commandlines that create the \n{foo} executable, linking the library
through this path.
 
Although they are somewhat more complicated to use, 
\indexterm{shared libraries}
have several advantages. For
instance, since they are not linked into the executable but only
loaded at runtime, they lead to (much) smaller executables. They are
not created with \n{ar}, but through the compiler. For instance:
\begin{verbatim}
%% gcc -dynamiclib -o ../lib/libfoo.so foosub.o
%% nm ../lib/libfoo.so 

../lib/libfoo.so(single module):
00000fc4 t __dyld_func_lookup
00000000 t __mh_dylib_header
00000fd2 T _bar
         U _printf
00001000 d dyld__mach_header
00000fb0 t dyld_stub_binding_helper
\end{verbatim}
Shared libraries are not actually linked into the executable;
instead, the executable will contain the information where the library
is to be found at execution time:
\begin{verbatim}
%% gcc -o foo fooprog.o -L../lib -Wl,-rpath,`pwd`/../lib -lfoo
%% ./foo
hello world
\end{verbatim}
The link line now contains the library path twice:
\begin{enumerate}
\item once with the \n{-L} directive so that the linker can resolve
  all references, and
\item once with the linker directive \verb+-Wl,-rpath,`pwd`/../lib+ which
  stores the path into the executable so that it can be found at runtime.
\end{enumerate}
Build the executable again, but without the \n{-Wl} directive. Where
do things go wrong and why? You can also fix this problem by using
\n{LD_LIBRARY_PATH}. Explore this.

Use the command \n{ldd} to get information about what shared libraries
your executable uses. (On Mac OS~X, use \n{otool -L} instead.)

\index{libraries!creating and using|)}

