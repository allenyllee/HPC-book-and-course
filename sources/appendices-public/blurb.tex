This course requires no great mathematical sophistication. Mostly it
assumes that you know the basics of linear algebra: what are matrices
and vectors, and the most common operations on them.

In the following appendices we will cover some less common bits of
theory that have been moved out of the main storyline of the
preceeding chapters.

\begin{itemize}
\item In \ref{app:norms} we cover some linear algebra material that
  you may not, or probably will not, have seen before.
\item Appendix \ref{app:complexity} gives the basic definitions of
  complexity theory, which you need for analyzing algorithms.
\item Much of scientific computing is about the treatment of Ordinary
  and Partial Differential Equations. In this book, they are discussed
  in chapters \ref{ch:odepde}
  and~\ref{ch:parallellinear}. Appendix~\ref{app:pde} gives some
  background knowledge about \acp{PDE}, and appendix~\ref{app:taylor}
  discusses Taylor expansion, which is a basic mechanism for getting a
  computational form of
  \acp{ODE} and \acp{PDE}.
\item In the discussion of computational methods and of parallel
  architectures, graph theory often comes up. The basic concepts are
  discussed in appendix~\ref{app:graph}.
\begin{notready}
\item The \ac{FFT} algorithm is used in chapter~\ref{ch:md} on
  Molecular Dynamics; we derive and discuss it in appendix~\ref{app:fft}.
\end{notready}
\item Finally, \ref{app:fsa} gives the definition and a short
  discussion of \acfp{FSA}, which comes up in the definition of CPUs.
\end{itemize}
