%%%%%%%%%%%%%%%%%%%%%%%%%%%%%%%%%%%%%%%%%%%%%%%%%%%%%%%%%%%%%%%%
%%%%%%%%%%%%%%%%%%%%%%%%%%%%%%%%%%%%%%%%%%%%%%%%%%%%%%%%%%%%%%%%
%%%%
%%%% This text file is part of the source of 
%%%% `Introduction to High-Performance Scientific Computing'
%%%% by Victor Eijkhout, copyright 2012
%%%%
%%%% This book is distributed under a Creative Commons Attribution 3.0
%%%% Unported (CC BY 3.0) license and made possible by funding from
%%%% The Saylor Foundation \url{http://www.saylor.org}.
%%%%
%%%%
%%%%%%%%%%%%%%%%%%%%%%%%%%%%%%%%%%%%%%%%%%%%%%%%%%%%%%%%%%%%%%%%
%%%%%%%%%%%%%%%%%%%%%%%%%%%%%%%%%%%%%%%%%%%%%%%%%%%%%%%%%%%%%%%%

Traditional scientific computing applications concern continuous data,
often stemming from \acp{PDE}. In this chapter we will concern two
applications areas, sorting and graph computations, that are not
traditionally considered as scientific computing, but that are
interesting from a point of high performance and large scale computing.

\Level 0 {Sorting}
\index{sorting|(}

Sorting is not a common operation in scientific computing: one expects
it to be more important in databases, whether these be financial or
biological (for instance in sequence alignment). However, it sometimes
comes up, for instance in \indexac{AMR} and other applications where
significant manipulations of data structures occurs.

In this section we will briefly look at the QuickSort
algorithm and how it can be done in parallel. For more details,
see~\cite{Kumar:parcomp-book} and the references therein.

\Level 1 {Brief introduction to sorting}

There are many sorting algorithms. One way to distinguish them is by
their computational complexity, that is, given an array of $n$
elements, how many operations does it take to sort them, as a function
of~$n$. For some sorting algorithms, the answer to this question is
not simple to give. While some algorithms work largely independent of
the state of the input data, for others, the operation count does
depend on it. One could imagine that a sorting algorithm would make a
pass over the data to check if it was already sorted. In that case,
the complexity is~$n$ operations for a sorted list, and something
higher for an unsorted list.

The so-called \indexterm{bubble sort} algorithm has a complexity
independent of the data to be sorted. This algorithm is given by:

\begin{displayalgorithm}
  \For{$t$ from $n-1$ down to $1$}{
    \For{$e$ from 1 to $t$}{
      \If{elements $e$ and $e+1$ are ordered the wrong way}{exchange
      them}
    }
  }
  \caption{The bubble sort algorithm}
\end{displayalgorithm}

It is easy to see that this algorithm has a complexity of~$O(n^2)$:
the inner loop does $t$ comparisons and up to $t$ exchanges. Summing
this from $1$ to $n-1$ gives approximately $n^2/2$ comparisons and a
at most the same number of exchanges.

Theoretically one can show that a sorting algorithm has to have at
least complexity~$O(n\log n)$\footnote{One can consider a sorting
  algorithm as a decision tree: a first comparison is made, depending
  on it two other comparisons are made, et cetera. Thus, an actual
  sorting becomes a path through this decision tree. If every path has
  running time~$h$, the tree has $2^h$ nodes. Since a sequence of $n$
  elements can be ordered in $n!$ ways, the tree needs to have enough
  paths to accomodate all of these; in other words, $2^h\geq
  n!$. Using Stirling's formula, this means that $n\geq O(n\log
  n)$}. There are indeed several algorithms that are guaranteed to
attain this complexity, but a very popular algorithm, called
\indexterm{quicksort} has only an `expected' complexity of~$O(n\log
n)$, and a worst case complexity of~$O(n^2)$.

\Level 1 {Quicksort} 
\index{quicksort|(}

Quicksort is a recursive algorithm, that, unlike bubble sort, is not
deterministic. It is a two step procedure, based on a reordering of
the sequence\footnote{The name is explained by its origin with the
  Dutch computer scientist Edsger Dijkstra; see
  \url{http://en.wikipedia.org/wiki/Dutch_national_flag_problem}.}:

\begin{displayalgorithm}
  \TitleOfAlgo{Dutch National Flag ordering of an array}
  \Input{An array of elements, and a `pivot' value}
  \Output{The input array with elements ordered as red-white-blue,
    where red elements are larger than the pivot, white elements are
    equal to the pivot, and blue elements are less than the pivot}
\end{displayalgorithm}

We state without proof that this can be done in $O(n)$ operations.
With this, quicksort becomes:

\begin{displayalgorithm}
  \TitleOfAlgo{Quicksort}
  \Input{An array of elements}
  \Output{The input array, sorted}
  \While{The array is longer than one element}{
    pick an arbitrary value as pivot \;
    apply the Dutch National Flag reordering to this array \;
    Quicksort( the blue elements ) \; Quicksort( the red elements ) \;
  }
\end{displayalgorithm}

The indeterminacy of this algorithm, and the variance in its
complexity, stems from the pivot choice. In the worst case, the pivot
is always the (unique) smallest element of the array. There will then
be no blue elements, the only white element is the pivot, and the
recursive call will be on the array of $n-1$ red elements. It is easy
to see that the running time will then be~$O(n^2)$. On the other hand,
if the pivot is always (close to) the median, that is, the element
that is intermediate in size, then the recursive calls will have an
about equal running time, and we get a recursive formula for the
running time:
\[ T_n = 2T_{n/2} + O(n) \]
which  is (again without proof) $O(n\log n)$.

We will now consider parallel implementations of quicksort.

\Level 1 {Quicksort in shared memory}

A simple parallelization of the quicksort algorithm can be achieved by
executing the two recursive calls in parallel. This is easiest
realized with a shared memory model, and threads
(section~\ref{sec:threads}) for the recursive calls. However, this
implementation is not efficient. 

On an array of length~$n$, and with perfect pivot choice, there will
be $n$~threads active in the final stage of the algorithm. Optimally,
we would want a parallel algorithm to run in $O(\log n)$ time, but
here the time is dominated by the initial reordering of the array by
the first thread.

\begin{exercise}
  Make this argument precise. What is the total running time, the
  speedup, and the efficiency of parallelizing the quicksort algorithm
  this way?
\end{exercise}

Since shared memory is not the most interesting case, we will forego
trying to make the thread implementation more efficient, and we will
move on straight away to distributed memory parallelization.

\Level 1 {Quicksort on a hypercube}

As was apparent from the previous section, for an efficient
parallelization of the quicksort algorithm, we need to make the Dutch
National Flag reordering parallel too. Let us then assume that the
array has been partitioned over the $p$ processors of a hypercube of
dimension~$d$ (meaning that $p=2^d$).

In the first step of the parallel algorithm, we choose a pivot, and
broadcast it to all processors. All processors will then apply the
reordering independently on their local data. 

In order to bring together the red and blue elements in this first
level, every processor is now paired up with one that has a binary
address that is the same in every bit but the most significant one. In
each pair, the blue elements are sent to the processor that has a
1~value in that bit; the red elements go to the processor that has a
0~value in that bit.

After this exchange (which is local, and therefore fully parallel),
the processors with an address $1xxxxx$ have all the red elements, and
the processors with an address $0xxxxx$ have all the blue
elements. The previous steps can now be repeated on the subcubes.

This algorithm keeps all processors working in every step; however, it
is susceptible to load imbalance if the chosen pivots are far from the
median. Moreover, this load imbalance is not lessened during the sort
process.

\Level 1 {Quicksort on a general parallel processor}

Quicksort can also be done on any parallel machine that has a linear
ordering of the processors. We assume at first that every processor
holds exactly one array element, and, because of the flag reordering,
sorting will always involve a consecutive set of processors.

Parallel quicksort of an array (or subarray in a recursive call)
starts by constructing a binary tree on the processors storing the
array. A~pivot value is chosen and broadcast through the tree. The
tree structure is then used to count on each processor how many
elements in the left and right subtree are less than, equal to, or
more than the pivot value. 

With this information, the root processor can compute where the
red/white/blue regions are going to be stored. This information is
sent down the tree, and every subtree computes the target locations
for the elements in its subtree.

If we ignore network contention, the reordering can now be done in
unit time, since each processor sends at most one element. This means
that each stage only takes time in summing the number of blue and red
elements in the subtrees, which is $O(\log n)$ on the top level,
$O(\log n/2)$~on the next, et cetera. This makes for almost perfect
speedup.

\index{quicksort|)}
\index{sorting|)}

\Level 0 {Graph problems}

Various problems in scientific computing can be formulated as graph
problems (for
an introduction to graph theory see Appendix~\ref{app:graph}); for
instance, you have 
encountered the problem of load
balancing (section~\ref{sec:graph-loadbalancing}) and finding
independent sets (section~\ref{sec:redblackgreen}).
%
Many traditional graph algorithms are
not immediately, or at least not efficiently, applicable, since the
graphs are often distributed, and traditional graph theory assume global
knowledge of the whole graph. 

Recently, new types of graph computations in have arisen in scientific
computing. Here the graph are no longer tools, but objects of study
themselves. Examples are the \indexterm{World Wide Web} or the
\indextermsub{social}{graph} of \indexterm{Facebook}, or the graph
of all possible \indexterm{protein interactions} in a living organism.

For this reason, combinatorial computational
science is becoming a discipline in its own right.  We will look at
some specific graph problems in this chapter.

\Level 1 {`Real world' graphs}

In discussions such as in section~\ref{sec:2dbvp} you have seen how
the discretization of \acp{PDE} leads to computational problems that
has a graph aspect to them. Such graphs have properties that make them
amenable to certain kinds of problems.
%
For instance, using \acp{FDM} or \acp{FEM} to model two or
three-dimensional objects leads graphs where each node is connected to
just a few neighbours. This makes it easy to find
\indextermp{separator}, which in turn allows such solution methods as
\indexterm{nested dissection}; see section~\ref{sec:dissection}.

There are however applications with computationally intensive graph
problems that do not look like \ac{FEM} graphs. We will briefly look
at the example of the world-wide web, and algorithms such
\indexterm{Google}'s \indexterm{PageRank} which try to find
authoratative nodes.

For now, we will call such graphs \indextermsubp{random}{graph},
although this term has a technical meaning
too~\cite{Erdos:randomgraph}.

\Level 2 {Properties of random graphs}

The graphs we have seen in most of this course have properties that
stem from the fact that they model objects in our three-dimensional
world. Thus, the typical distance between two nodes is
typically~$O(N^{1/3})$ where $N$ is the number of nodes. Random graphs
do not behave like this: they often have a \indexterm{small world}
property where the typical distance is~$O(\log N)$. A~famous example
is the graph of film actors and their connection by having appeared in
the same movie: according to `Six degrees of separation', no two actors
have a distance more than six in this graph. In graph terms this means
that the diameter of the graph is six.

Small-world graphs have other properties, such as the existence of
cliques (although these feature too in higher order \ac{FEM} problems)
and hubs: nodes of a high degree. This leads to implications such as
the following: deleting a random node in such a graph does not have a
large effect on shortest paths.

\begin{exercise}
  Considering the graph of airports and the routes that exist between
  them. If there are only hubs and non-hubs, argue that deleting a
  non-hub has no effect on shortest paths between other airports. On
  the other hand, consider the nodes ordered in a two-dimensional
  grid, and delete an arbitrary node. How many shortest paths are affected?
\end{exercise}

\begin{notready}
\Level 2 {Concepts}

Two measures of centrality: eigenvector centrality and
betweenness. Low EC and high betweenness indicates a `gatekeeper'
function.

\begin{description}
  \item[Degree, diameter,distance] are defined as usual.
  \item[Betweenness]%\footnote{\url{http://en.wikipedia.org/wiki/Betweenness#Betweenness_centrality}}
    is defined as 
    \[ C(v)=\frac
       {\sum_{s\not=t\not=v} \#\hbox{shortest paths between $s,t$}}
       {\sum_{s\not=t\not=v} \#\hbox{shortest paths between $s,t$ that go through $v$}}
    \]
  \item[Eigenvector centrality] is the magnitude of a node's component
    in the Perron vector; see section~\ref{sec:pagerank}.
  \item[Closeness] measures the average distance of a node to all
    others:
    \[ c(v) = \frac{1}{n-1}\sum_{t\not=v}d(t,v). \]
  \item[Clustering coefficient] measures how much of a graph is
    organized in cliques. A~global coefficient is
    \[ \frac{\#\hbox{closed triplets}}{\#\hbox{open triplets}} \]
    where an open triplet is a triplet $i,j,k$ so that
    \[ (i,j)\in E,\quad (i,k)\in E, \quad (j,k)\not\in E \]
    while for a closed triplet the last relation is an
    inclusion. A~local definition is based on the fact that in a
    \indexterm{clique} of $k$ nodes there are $k(k-1)/2$ edges (in an
    undirected graph). Defining the \indexterm{neighbourhood} of a
    node~$i$ as 
    \[ N_i=\{j\not=i\colon (i,j)\in E\}, \]
    the local clustering coefficient can be defined as 
    \[ \frac{|\{e_{jk}\}_{j,k\in N_i}|}{k(k-1)/2}. \]
\end{description}
\end{notready}

\Level 1 {Hypertext algorithms}
\label{sec:pagerank}

There are several algorithms based on linear algebra
for measuring the importance of web
sites~\cite{Langville2005eigenvector}. We will briefly define a few
and discuss computational implications.

\Level 2 {HITS}

In the HITS (Hypertext-Induced
Text Search) algorithm, sites have a \emph{hub} score that measures how many
other sites it points to, and an \emph{authority} score that measures
how many sites point to it. To calculate such scores we define an
\indexterm{incidence matrix}~$L$, where
\[ L_{ij}=
\begin{cases}
  1&\mbox{document $i$ points to document $j$}\\
  0&\mbox{otherwise}
\end{cases}
\]
The authority scores $x_i$ are defined as the sum of the hub scores
$y_j$ of everything that points to~$i$, and the other way around. Thus
\[
\begin{array}{l}
  x=L^ty\\ y=Lx
\end{array}
\]
or $x=LL^tx$ and $y=L^tLy$, showing that this is an eigenvalue
problem. The eigenvector we need has only nonnegative entries; this is
known as the \indexterm{Perron vector} for a
\indextermsub{nonnegative}{matrix}, see appendix~\ref{app:perron}. The
Perron vector is computed by a \indexterm{power method}; see next section.

A practical search strategy is:
\begin{itemize}
\item Find all documents that contain the search terms;
\item Build the subgraph of these documents, and possible one or two
  levels of documents related to them;
\item Compute authority and hub scores on these documents, and present
  them to the user as an ordered list.
\end{itemize}

\Level 2 {Pagerank}
\index{PageRank|(}

The PageRank~\cite{PageBrin:PageRank} basic idea is similar to
HITS. Again we define a connectivity matrix
\[ M_{ij}=
\begin{cases}
  1&\mbox{if page $j$ links to $i$}\\
  0&\mbox{otherwise}
\end{cases}
\]
With $e=(1,\ldots,1)$, the vector $d^t=e^tM$ counts how many links
there are on a page: $d_i$~is the number of links on page~$i$. We
construct a diagonal matrix $D=\diag(d_1,\ldots)$ we normalize $M$ to
$T=MD\inv$. 

Now the columns sums (that is, the sum of the elements in any column)
of $T$ are all~$1$, which we can express as $e^tTe$. Such a matrix is
called \indextermsub{stochastic}{matrix}. It has the following
interpretation:
\begin{quote}
  If $p$ is a vector of probabilities, that is, $p_i$ is the
  probability that the user is looking at page~$i$, then $Tp$ is the
  vector of probabilities after the user has clicked on a random link.
\end{quote}
The PageRank algorithm introduces another element: sometimes the user
will get bored from clicking, and will go to an arbitrary page (there
are also provisions for pages with no outgoing links). If we
call $s$ the chance that the user will click on a link, then the
chance of going to an arbitrary page is $1-s$. Together, we now have
the process
\[ p'\leftarrow sTp+(1-s)e, \]
that is, if $p$~is a vector of probabilities then $p'$ is a vector of
probabilities that describes where the user is after making one page
transition, either by clicking on a link or by `teleporting'.

The PageRank vector is the stationary point of this process; you can
think of it as the probability distribution after the user has made
infinitely many transitions. The PageRank vector satisfies
\[ p=sTp+(1-s)e \Leftrightarrow (I-sT)p=(1-s)e. \]
Thus, we now have to wonder whether $I-sT$ has an inverse.
If the inverse exists it satisfies
\[ (I-sT)\inv = I+sT+s^2T^2+\cdots \]
It is not hard to see that the inverse exists: with the Gershgorin theorem
(appendix~\ref{app:gershgorin}) you can see that the eigenvalues of
$T$ satisfy $|\lambda|\leq 1$. Now use that $s<1$, so the series of
partial sums converges.

The above formula for the inverse also indicates a way to compute the
PageRank vector~$p$ by using a series of matrix-vector
multiplications.

\begin{exercise}
  Write pseudo-code for computing the PageRank vector, given the
  matrix~$T$. Show that you never need to compute the powers of~$T$
  explicitly. (This is an instance of \indexterm{Horner's rule}.
\end{exercise}

In the case that $s=1$, meaning that we rule out teleportation, the
PageRank vector satisfies $p=Tp$, which is again the problem of
finding the \indexterm{Perron vector}; see appendix~\ref{app:perron}.

We find the Perron vector by a power iteration (section~\ref{app:power-method})
\[ p^{(i+1)} = T p^{(i)}. \]
This is a sparse matrix vector product, but unlike in the \ac{BVP}
case the sparsity is unlikely to have a structure such as
bandedness. Computationally, one probably has to use the same
parallelism arguments as for a dense matrix: the matrix has to be
distributed two-dimensionally~\cite{OgAi:sparsestorage}.

\index{PageRank|)}

\endinput

Triadic closure: if $(i,j)\in E$ and $(i,k)\in E$ then $(j,k)\in E$;
example: friend or trust relationship.

A \indextermsub{scale-free}{network}, or one that observes a
\indexterm{power law}, is one where the fraction of nodes with
degree~$k$ is proportional to $k^{-\gamma}$ where $\gamma$ is
positive, typically $2<\gamma<3$. There are claims that graphs such as
the \ac{WWW} obey power laws; this is a claim about the logical
structure, not the physical one, which is very hard to
infer~\cite{Willinger:internet}.

A small-world network can be created by extending an existing network
by \emph{preferential attachment}:
the chance of attaching a new node is proportional to the degree
of the old node.

Power law networks usually have a very small diameter.

