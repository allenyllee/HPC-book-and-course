%%%%%%%%%%%%%%%%%%%%%%%%%%%%%%%%%%%%%%%%%%%%%%%%%%%%%%%%%%%%%%%%
%%%%%%%%%%%%%%%%%%%%%%%%%%%%%%%%%%%%%%%%%%%%%%%%%%%%%%%%%%%%%%%%
%%%%
%%%% This text file is part of the source of 
%%%% `Introduction to High-Performance Scientific Computing'
%%%% by Victor Eijkhout, copyright 2012
%%%%
%%%% This book is distributed under a Creative Commons Attribution 3.0
%%%% Unported (CC BY 3.0) license and made possible by funding from
%%%% The Saylor Foundation \url{http://www.saylor.org}.
%%%%
%%%%
%%%%%%%%%%%%%%%%%%%%%%%%%%%%%%%%%%%%%%%%%%%%%%%%%%%%%%%%%%%%%%%%
%%%%%%%%%%%%%%%%%%%%%%%%%%%%%%%%%%%%%%%%%%%%%%%%%%%%%%%%%%%%%%%%

MapReduce~\cite{Google:mapreduce} is a programming model for certain
parallel operations. One of its distinguishing characteristics is that
it is implemented using \indexterm{functional programming}.
The MapReduce model handles computations of the following form:
\begin{itemize}
\item For all available data, select items that satisfy a certain
  criterium;
\item and emit a key-value pair for them. This is the mapping stage.
\item Optionally there can be a combine/sort stage where all pairs with the
  same key value are grouped together.
\item Then do a global reduction on the keys, yielding one or more of
  the corresponding values. This is the reduction stage.
\end{itemize}

We will now give a few examples of using MapReduce, and present the
functional programming model that underlies the MapReduce abstraction.

\paragraph*{{\bf Expressive power of the MapReduce model}}
 
The reduce part of the MapReduce model makes it a prime candidate for
computing global statistics on a dataset.
One example would be to count how many times each of a set of words
appears in some set of documents. The function being mapped knows the
set of words, and outputs for each document a pair of document name
and a list with the occurrence counts of the words. The reduction then
does a componentwise sum of the occurrence counts.

The combine stage of MapReduce makes it possible to transform data.
An example is a `Reverse Web-Link Graph': the map function
outputs target-source pairs for each link to a target URL found in
a page named "source". The reduce function concatenates the list of
all source URLs associated with a given target URL and emits the pair
target-list(source).

A less obvious example is computing PageRank
(section~\ref{sec:pagerank}) with MapReduce. Here we use the fact that
the PageRank computation relies on a distributed sparse matrix-vector
product. Each web page corresponds to a column of the web matrix~$W$;
given a probability $p_j$ of being on page~$j$, that page can then
compute tuples $\langle i,w_{ij}p_j\rangle$. The combine stage of MapReduce
then sums together $(Wp)_i=\sum_j w_{ij}p_j$.

Database operations can be implemented with MapReduce but since it has
a relatively large latency, it is unlikely to be competitive with
standalone databases, which are optimized for fast processing of a
single query, rather than bulk statistics. For other applications see
\url{http://horicky.blogspot.com/2010/08/designing-algorithmis-for-map-reduce.html}.

\paragraph*{{\bf Functional programming}}
\index{functional programming|(}
%
The mapping and reduction operations are easily implemented on any
type of parallel architecture, using a combination of threading and
message passing. However, at Google where this model was developed
traditional parallelism was not attractive for two reasons. First of
all, processors could fail during the computation, so a traditional
model of parallelism would have to be enhanced with \indexterm{fault
  tolerance} mechanisms. Secondly, the computing hardware could
already have a load, so parts of the computation may need to be
migrated, and in general any type of synchronization between tasks
would be very hard.

MapReduce is one way to abstract from such details of parallel
computing, namely through adopting a functional programming model. In
such a model the only operation is the evaluation of a function,
applied to some arguments, where the arguments are themselves the
result of a function application, and the result of the computation is
again used as argument for another function application. In
particular, in a strict functional model there are no variables, so
there is no static data.

A function application, written in \indexterm{Lisp} style as \n{(f a
  b)} (meaning that the function \n{f} is applied to arguments \n{a}
and \n{b}) would then be executed by collecting the inputs from
whereven they are to the processor that evaluates the
function~\n{f}. The mapping stage of a MapReduce process is denoted
\begin{verbatim}
  (map f (some list of arguments))
\end{verbatim}
and the result is a list of the function results of applying \n{f} to
the input list. All details of parallelism and of guaranteeing that
the computation successfully finishes are handled by the \n{map}
function.

Now we are only missing the reduction stage, which is just as simple:
\begin{verbatim}
(reduce g (map f (the list of inputs)))
\end{verbatim}
The \n{reduce} function takes a list of inputs and performs a
reduction on it.

The attractiveness of this functional model lies in the fact that
functions can not have \indexterm{side effects}: because they can only
yield a single output result, they can not change
their environment, and hence there is no coordination problem of
multiple tasks accessing the same data. 

Thus, MapReduce is a useful
abstraction for programmers dealing with large amounts of data.
Of course, on an implementation level the MapReduce software uses
familiar concepts such as decomposing the data space, keeping a work
list, assigning tasks to processors, retrying failed operations, et
cetera.

\index{functional programming|)}

\endinput

\paragraph{{\bf Performance characteristics}}

MapReduce is sometimes mentioned as an alternative to databases: the
map and reduce functions can be used to form certain queries in the
data. However, MapReduce has a much higher latency than database
queries, which are typically optimized for minimum response time.
