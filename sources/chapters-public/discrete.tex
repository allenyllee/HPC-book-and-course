Interaction between discrete elements: 
\begin{itemize}
\item external
\item nearby 
\item far
\end{itemize}

External forces are simple and conveniently parallel.

Nearby forces are easy to handle if spatial domain decomposition is
used: at best ghost regions needed.

Load imbalance because of particle migration.

Far-field forces are difficult because every particle interacts with
every other: naive algorithms are $O(n^2)$.

Particle-mesh methods: move particles to nearby mesh points, use the
fact that the far-field equation satisfies a PDE that is easy to
solve, use FFT or multigrid (complexity $O(n\log n)$, calculate forces
by interpolation.

Approximation by letting faraway particles act as group:
\begin{enumerate}
\item Barnes-Hut
\item \acfp{FMM}
\end{enumerate}
Also $n\log n$ complexity.
