\label{ex:recursivedoubling}
  The operation
\begin{verbatim}
for (i) {
  x[i+1] = a[i]*x[i] + b[i];
}
\end{verbatim}
  can not be handled by a pipeline because there is
  a \indexterm{dependency} between input of one iteration of the operation
  and the output of the previous.
  However, you can transform the loop into one that is mathematically
  equivalent, and potentially more efficient to compute. Derive an
  expression that computes \texttt{x[i+2]} from \texttt{x[i]} without
  involving \texttt{x[i+1]}. This is known as \indexterm{recursive
    doubling}. Assume you have plenty of temporary storage. You can now
  perform the calculation by
  \begin{itemize}
  \item Doing some preliminary calculations;
  \item computing \texttt{x[i],x[i+2],x[i+4],...}, and from these,
  \item compute the missing terms \texttt{x[i+1],x[i+3],...}.
  \end{itemize}
  Analyze the efficiency of this scheme by giving formulas for
  $T_0(n)$ and~$T_s(n)$. Can you think of an argument
  why the preliminary calculations may be of lesser importance in some
  circumstances?
