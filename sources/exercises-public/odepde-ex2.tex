  Consider the initial value problem $y'(t)=y(t)(1-y(t))$.
Observe that $y\equiv 0$ and $y\equiv 1$ are solutions. These are called `equilibrium solutions'.
\begin{enumerate}
\item A solution is stable, if perturbations `converge back to the
  solution', meaning that for $\epsilon$ small enough,
  \[ \hbox{if  $y(t)=\epsilon$ for some~$t$, then
    $\lim_{t\rightarrow\infty}y(t)=0$} \]
  and
  \[ \hbox{if  $y(t)=1+\epsilon$ for some~$t$, then
    $\lim_{t\rightarrow\infty}y(t)=1$} \]
  This requires for instance that \[ y(t)=\epsilon\Rightarrow y'(t)<0. \]
  Investigate this behaviour. Is zero a stable solution? Is one?
\item Consider the explicit method
  \[ y_{k+1}=y_k+\Delta t y_k(1-y_k) \]
  for computing a numerical solution
  to the differential equation. Show that
  \[ y_k\in(0,1)\Rightarrow y_{k+1}>y_k,\qquad
     y_k>1\Rightarrow y_{k+1}<y_k \]
\item Write a small program to investigate the behaviour of the
  numerical solution under various choices for $\Delta t$. Include
  program listing and a couple of runs in your homework submission.
\item You see from running your program that the numerical solution
  can oscillate. Derive a condition on $\Delta t$ that makes the
  numerical solution monotone. It is enough to show that
  $y_k<1\Rightarrow y_{k+1}<1$, and $y_k>1\Rightarrow y_{k+1}>1$.
\item Now consider the implicit method
  \[ y_{k+1}-\Delta t y_{k+1}(1-y_{k+1})=y_k \]
  and show that $y_{k+1}$ is
  easily computed from~$y_k$. Write a program, and investigate the
  behaviour of the numerical solution under various choices
  for~$\Delta t$.
\item Show that the numerical solution of the implicit scheme is
  monotone for all choices of~$\Delta t$.
\end{enumerate}
