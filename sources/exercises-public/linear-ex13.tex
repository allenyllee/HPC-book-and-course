  Earlier, you saw that 2D \acs{BVP} (section~\ref{sec:2dbvp}) give
  rise to a certain kind of matrix. We stated, without proof, that for
  these matrices pivoting is not needed. We can now formally prove
  this, focusing on the crucial property of \indexterm{diagonal
    dominance}:
  \[ \forall_i a_{ii}\geq\sum_{j\not=i}|a_{ij}|. \]

  Assume that a matrix $A$ satisfies
  $\forall_{j\not=i}\colon a_{ij}\leq 0$. Show that the matrix is
  diagonally dominant iff there are vectors
  $u,v\geq0$ (meaning that each component is nonnegative) such that
  $Au=v$.

  Show that, after eliminating a variable, for the remaining
  matrix~$\tilde A$ there are again vectors $\tilde u,\tilde v\geq0$
  such that $\tilde A\tilde u=\tilde v$.

  Now finish the argument that (partial) pivoting is not necessary if
  $A$ is symmetric and diagonally dominant. (One can actually prove
  that pivoting is not necessary for any \indexacf{SPD} matrix, and
  diagonal dominance is a stronger condition than \ac{SPD}-ness.)
