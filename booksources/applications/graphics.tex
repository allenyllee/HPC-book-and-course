% -*- latex -*-
%%%%%%%%%%%%%%%%%%%%%%%%%%%%%%%%%%%%%%%%%%%%%%%%%%%%%%%%%%%%%%%%
%%%%%%%%%%%%%%%%%%%%%%%%%%%%%%%%%%%%%%%%%%%%%%%%%%%%%%%%%%%%%%%%
%%%%
%%%% This text file is part of the source of 
%%%% `Introduction to High-Performance Scientific Computing'
%%%% by Victor Eijkhout, copyright 2012-6
%%%%
%%%% This book is distributed under a Creative Commons Attribution 3.0
%%%% Unported (CC BY 3.0) license and made possible by funding from
%%%% The Saylor Foundation \url{http://www.saylor.org}.
%%%%
%%%%
%%%%%%%%%%%%%%%%%%%%%%%%%%%%%%%%%%%%%%%%%%%%%%%%%%%%%%%%%%%%%%%%
%%%%%%%%%%%%%%%%%%%%%%%%%%%%%%%%%%%%%%%%%%%%%%%%%%%%%%%%%%%%%%%%

\index{graphics|(}

Many problems in computer graphics can be
considered as a case of \ac{SIMD} programming:
an image is a square or rectangular array
where each pixel can be manipulated independently,
and often with the same operation.

For instance, an image with $1024\times 1024$
pixels, of 8~bits each, takes $2^{20}$~bytes or 1~megabyte.
In the context of a movie with a framerate of 60~frames,
and a processor with an average instruction rate
of 1~GHz, this means that each operation can take about 16~nanoseconds.
%
(While this sounds like a reasonable operation rate,
of course we also have to wonder about the bandwidth.)

Examples of operations on a single pixel are
thresholding and contrast stretching.

Other operations involve several pixels at once:
smoothing operations such as removing noise
use a \indexterm{difference stencil}.
A~typical averaging stencil would be 
\[
\begin{matrix}
  1&1&1\\ 1&8&1\\ 1&1&1
\end{matrix}
\]
The stencils you saw in Chapter~\ref{ch:odepde} represent
differentiation; in graphics that can be used for operations
such as edge detection. A~popular choice for a differentiation stencil is
\[
\hbox{$x$: }
\begin{matrix}
  -1&0&1\\ -2&0&2\\ -1&0&1
\end{matrix}
\qquad\hbox{$y$: }
\begin{matrix}
  -1&-2&-1\\ 0&0&0\\ -1&-2&-1
\end{matrix}
\]


The sequential code for applying a $3\times 3$ stencil
on an $N\times N$ image
would be
\begin{verbatim}
for (i=0; i<N; i++) {
  for (j=0; j<N; j++) {
    s = 0;
    for (ii=-1; ii<=1; ii++)
      for (jj=-1; jj<=1; jj++)
        s += frame[i+ii][j+jj];
    avg[i,j] = s;
  }
}
\end{verbatim}

As discussed in section~\ref{sec:aos-soa}, this code structure
is advantageous for certain types of parallelism. For instance,
in \indexac{CUDA} one would write a \emph{kernel}\index{kernel!CUDA}
containing the inner two loops, and instantiate this
in parallel on each $[i,j]$ coordinate of the averages array.

On the other hand, this code structure would not be right for
\indextermbus{vector}{instructions} or \emph{pipeline
  instructions}\index{pipeline!instruction} where the parallelism has
to be in the inner loop, and preferably be as large as possible.

\index{graphics|)}
