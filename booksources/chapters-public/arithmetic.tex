%%%%%%%%%%%%%%%%%%%%%%%%%%%%%%%%%%%%%%%%%%%%%%%%%%%%%%%%%%%%%%%%
%%%%%%%%%%%%%%%%%%%%%%%%%%%%%%%%%%%%%%%%%%%%%%%%%%%%%%%%%%%%%%%%
%%%%
%%%% This text file is part of the source of 
%%%% `Introduction to High-Performance Scientific Computing'
%%%% by Victor Eijkhout, copyright 2012
%%%%
%%%% This book is distributed under a Creative Commons Attribution 3.0
%%%% Unported (CC BY 3.0) license and made possible by funding from
%%%% The Saylor Foundation \url{http://www.saylor.org}.
%%%%
%%%%
%%%%%%%%%%%%%%%%%%%%%%%%%%%%%%%%%%%%%%%%%%%%%%%%%%%%%%%%%%%%%%%%
%%%%%%%%%%%%%%%%%%%%%%%%%%%%%%%%%%%%%%%%%%%%%%%%%%%%%%%%%%%%%%%%

Of the various types of data that one normally encounters, the ones we
are concerned with in the context of scientific computing are the
numerical types: integers (or whole numbers)
$\ldots,-2,-1,0,1,2,\ldots$, real numbers $0,1,-1.5,2/3,\sqrt 2,\log
10,\ldots$, and complex numbers $1+2i,\sqrt 3-\sqrt 5i,\ldots$.
Computer memory is organized to give only a certain amount of space to
represent each number, in multiples of \indexterm{bytes}, each
containing 8~\indexterm{bits}. Typical values are 4 bytes for an
integer,  4~or~8 bytes for a real number, and 8~or~16 bytes for a
complex number.

Since only a certain amount of memory is available to store a number,
it is clear that not all numbers of a certain type can be stored. For
instance, for integers only a range is stored. In the case of real
numbers, even storing a range is not possible since any interval $[a,b]$
contains infinitely many numbers. Therefore, any
\indextermsub{representation of}{real numbers} will cause gaps between the
numbers that are stored. As a result, any computation that results in
a number that is not representable will have to be dealt with by issuing an
error or by approximating the result. In this chapter we will look at
the ramifications of such approximations of the `true' outcome of
numerical calculations.

\Level 0 {Integers}

In scientific computing, most operations are on real
numbers. Computations on integers rarely add up to any serious
computation load\footnote{Some computations are done on bit
  strings. We will not mention them at all.}. It is mostly for
completeness that we start with a short discussion of integers.

Integers are commonly stored in 16, 32, or 64~bits, with 16 becoming
less common and 64 becoming more and more so. The main reason for this increase is
not the changing nature of computations, but the fact that integers
are used to index arrays. As the size of data sets grows (in
particular in parallel computations), larger indices are needed. For
instance, in 32~bits one can store the numbers zero through
$2^{32}-1\approx 4\cdot 10^9$. In other words, a 32 bit index can
address 4 gigabytes of memory. Until recently this was enough for most
purposes; these days the need for larger data sets has made 64~bit
indexing necessary.

When we are indexing an array, only positive integers are needed.
In general integer computations, of course, we need to accomodate
the negative integers too. We will now discuss several strategies for
implementing negative integers. Our motivation here will be that
arithmetic on positive and negative integers should be as simple as on
positive integers only: the circuitry that we have for comparing and
operating on bitstrings should be usable for (signed) integers.

There are several ways of implementing
negative integers. The simplest solution is to reserve one bit as a
\indexterm{sign bit}, and use the remaining 31 (or 15 or 63; from now on we will
consider 32 bits the standard) bits to store the
absolute magnitude. By comparison, we will call the straightforward
interpretation of bitstring \indexterm{unsigned} integers.

\[
\begin{array}{|c|rrrrrr|}
  \hline
  \hbox{bitstring}&
  00\cdots0&\ldots&01\cdots1&
  10\cdots0&\ldots&11\cdots1\\ \hline
  \hbox{interpretation as unsigned int}&
  0&\ldots&2^{31}-1&
  2^{31}&\ldots&2^{32}-1\\ \hline
  \hbox{interpretation as signed integer}&
  0&\cdots&2^{31}-1&
  -0&\cdots&-(2^{31}-1)\\
  \hline
\end{array}
\]

This scheme has some disadvantages, one being that
there is both a positive and negative number zero. This means that a test
for equality becomes more complicated than simply testing for equality
as a bitstring. More importantly, in the second half of the
bitstrings, the interpretation as signed integer \emph{decreases},
going to the right. This means that a test for greater-than becomes
complex; also
adding a positive number to a
negative number now has to be treated differently from adding it to a
positive number.

Another solution would be to let an unsigned number $n$ be interpreted
as $n-B$ where $B$ is some plausible base, for instance~$2^{31}$.

\[
\begin{array}{|c|rrrrrr|}
  \hline
  \hbox{bitstring}&
  00\cdots0&\ldots&01\cdots1&
  10\cdots0&\ldots&11\cdots1\\ \hline
  \hbox{interpretation as unsigned int}&
  0&\ldots&2^{31}-1&
  2^{31}&\ldots&2^{32}-1\\ \hline
  \hbox{interpretation as shifted int}&
  -2^{31}&\ldots&-1&
  0&\ldots&-2^{31}+1\\ \hline
\end{array}
\]

This shifted scheme does not suffer from the $\pm0$ problem, and addition now
works properly. However, it would be nice if for positive numbers
\n{int n} and \n{unsigned int n} had the same bit pattern. To get
this, we flip the first and second half of the number line.

The resulting scheme, which is the one that is used most commonly, is
called \indexterm{2's complement}. Using this scheme,
the representation of integers is
formally defined as follows.
\begin{itemize}
\item If $0\leq m\leq 2^{31}-1$, the normal bit pattern for $m$ is
  used.
\item For $-2^{31}\leq n\leq -1$, $n$ is represented by the bit
  pattern for $2^{32}-|n|$.
\end{itemize}
The following diagram shows the correspondence between bitstrings and
their interpretation as 2's complement integer:
\[
\begin{array}{|c|rrrrrr|}
  \hline
  \hbox{bitstring}&
  00\cdots0&\ldots&01\cdots1&
  10\cdots0&\ldots&11\cdots1\\ \hline
  \hbox{interpretation as unsigned int}&
  0&\ldots&2^{31}-1&
  2^{31}&\ldots&2^{32}-1\\ \hline
  \hbox{interpretation as 2's comp. integer}&
  0&\cdots&2^{31}-1&
  -2^{31}&\cdots&-1\\
  \hline
\end{array}
\]

Some observations:
\begin{itemize}
\item There is no overlap between the bit patterns for positive and
  negative integers, in particular, there is only one pattern for zero.
\item The positive numbers have a leading bit zero, the negative
  numbers have the leading bit set.
\end{itemize}

\begin{exercise}
  \label{ex:compare-mn}
  For the `naive' scheme and the 2's complement scheme for negative
  numbers, give pseudocode for the comparison test $m<n$, where $m$
  and~$n$ are integers. Be careful to distinguish between all cases of
  $m,n$ positive, zero, or negative.
\end{exercise}

Adding two numbers with the same sign, or multiplying two numbers of
any sign, may lead to a result that is too
large or too small to represent. This is called \indexterm{overflow}.

\begin{exercise}
  Investigate what happens when you perform such a calculation. What
  does your compiler say if you try to write down a nonrepresentible
  number explicitly, for instance in an assignment statement?
\end{exercise}

In exercise~\ref{ex:compare-mn} above you explored comparing two integers.
Let us know explore how 
subtracting numbers in two's complement is implemented. Consider $0\leq
m\leq 2^{31}-1$ and $1\leq n\leq 2^{31}$ and let us see what happens
in the computation of $m-n$. 

Suppose we have an algorithm for adding and subtracting unsigned
32-bit numbers. Can we use that to subtract two's complement integers?
We start by observing that the integer subtraction $m-n$ becomes the
unsigned addition $m+(2^{32}-n)$.
\begin{itemize}
\item Case: $m<n$. In this case, $m-n$ is negative and $1\leq
  |m-n|\leq 2^{31}$, so the bit pattern for $m-n$ is that of
  $2^{32}-(n-m)$. Now, $2^{32}-(n-m)=m+(2^{32}-n)$, so we can compute
  $m-n$ in 2's complement by adding the bit patterns of $m$ and $-n$ as
  unsigned integers.
\item Case: $m>n$. Here we observe that
  $m+(2^{32}-n)=2^{32}+m-n$. Since $m-n>0$, this is a number $>2^{32}$
  and therefore not a legitimate
  representation of a negative number. However, if we store this
  number in 33 bits, we see that it is
  the correct result $m-n$, plus a single bit in the
  33-rd position. Thus, by performing the unsigned addition, and
  ignoring the \indexterm{overflow bit}, we again get the correct result.
\end{itemize}
In both cases we conclude that we can perform the subtraction $m-n$ by adding
the unsigned number that represent $m$ and $-n$
and ignoring overflow if it occurs.

\Level 0 {Representation of real numbers}
\label{sec:float-representation}

In this section we will look at how various kinds of numbers are
represented in a computer, and the limitations of various schemes. The
next section will then explore the ramifications of this for arithmetic
involving computer numbers.

Real numbers are stored using a scheme that is analogous to what is
known as `scientific notation', where a number is represented as a
\indexterm{significant} and an \indexterm{exponent}, for
instance~$6.022\cdot 10^{23}$, which has a significant \n{6022} with a
\indexterm{radix point} after the first digit, and an exponent~\n{23}.
This number stands for
\[ 6.022\cdot 10^{23}= \left[
    6\times 10^0+0\times 10^{-1}+2\times10^{-2}+2\times10^{-3}
    \right] \cdot 10^{23}. 
\]
We introduce a \indexterm{base}, 
a small integer number, 10~in the preceding example, and 2~in computer
numbers, and write numbers in terms of it as a sum of $t$~terms:
\begin{equation}
    x = \pm 1 \times \left[
       d_1\beta^0+d_2\beta^{-1}+d_3\beta^{-2}+\cdots+d_t\beta^{-t+1}b\right] \times \beta^e =
       \pm \Sigma_{i=1}^t d_i\beta^{1-i}  \times\beta^e
       \label{eq:floatingpoint-def}
\end{equation}
where the components are
\begin{itemize}
\item the \indexterm{sign bit}: a single bit storing whether the
  number is positive or negative;
\item $\beta$ is the base of the number system;
\item $0\leq d_i\leq \beta-1$ the digits of the \indexterm{mantissa}
  or \indexterm{significant} -- the location of the radix point
  (decimal point in decimal numbers) is implicitly assumed to the
  immediately following the first digit;
\item $t$ is the length of the mantissa;
\item $e\in [L,U]$ exponent; typically $L<0<U$ and $L\approx-U$.
\end{itemize}

Note that there is an explicit sign bit for the whole number; the sign
of the exponent is handled differently. 
For reasons of efficiency, $e$~is not a signed number; instead it is
considered as an unsigned number in \indexterm{excess} of a certain minimum
value. For instance, the bit pattern for the number zero is
interpreted as~$e=\nobreak L$.

\Level 1 {Some examples}

Let us look at some specific examples of floating point
representations. Base 10 is the most logical choice for human
consumption, but computers are binary, so base~2 predominates
there. Old IBM\index{IBM} mainframes grouped bits to make for a base~16
representation.

\begin{tabular}{r|r|r|r|r}
  &$\beta$&$t$&$L$&$U$\\ \hline
  IEEE single precision (32 bit)&2&24&-126&127\\
  IEEE double precision (64 bit)&2&53&-1022&1023\\
  Old Cray 64 bit&2&48&-16383&16384\\
  IBM mainframe 32 bit&16&6&-64&63\\
  packed decimal&10&50&-999&999\\
  Setun&3
\end{tabular}

Of these, the single and double precision formats are by far the most
common. We will discuss these in section~\ref{sec:ieee754} and
further.

\Level 2 {Binary coded decimal}

Decimal numbers are not relevant in scientific computing, but they are
useful in financial calculations, where computations involving money
absolutely have to be exact. Binary arithmetic is at a disadvantage
here, since numbers such as $1/10$ are repeating fractions in
binary. With a finite number of bits in the mantissa\index{mantissa},
this means that the number $1/10$ can not be represented exactly in
binary.  For this reason, \indexterm{binary-coded-decimal} schemes
were used in old IBM\index{IBM} mainframes, and are in fact being
standardized in revisions of IEEE754~\cite{ieee754-webpage}; see also
section~\ref{sec:ieee754}.

In BCD schemes, one or more decimal digits are encoded in a number of
bits. The simplest scheme would encode the digits $0\ldots9$ in four
bits. This has the advantage that in a
BCD number each digit is readily identified; it has the disadvantage
that about $1/3$ of all bits are wasted, since 4 bits can
encode the numbers~$0\ldots15$.
%
More efficient encodings would encode $0\ldots999$ in ten bits, which
could in principle store the numbers~$0\ldots1023$. While this is
efficient in the sense that few bits are wasted, identifying
individual digits in such a number takes some decoding. For this
reason, BCD arithmetic needs hardware support from the processor,
which is rarely found these days; one example is the IBM
Power architecture, starting with the \indextermbus{IBM}{Power6}.

\Level 2 {Other number bases for computer arithmetic}

There have been some experiments with \indexterm{ternary arithmetic}
(see~\url{http://en.wikipedia.org/wiki/Ternary_computer}
and~\url{http://www.computer-museum.ru/english/setun.htm}), however,
no practical hardware exists.

\Level 1 {Limitations}
\label{sec:exceptions}

Since we use only a finite number of bits to store floating point
numbers, not all numbers can be represented. The ones that can not be
represented fall into two categories: those that are too large or too
small (in some sense), and those that fall in the gaps. Numbers can be
too large or too small in the following ways.
\begin{itemize}
\item[Overflow] The largest number we can store is
  \[ (\beta-1)\cdot1+(\beta-1)\cdot\beta\inv+\cdots
  +(\beta-1)\cdot\beta^{-(t-1)}=\beta-1\cdot\beta^{-(t-1)},
  \]
  and the smallest number (in an absolute
  sense) is $-(\beta-\beta^{-(t-1)}$; anything larger than the
  former or smaller than the latter causes a condition called
  \indexterm{overflow}.
\item[Underflow]The number closest to zero is $\beta^{-(t-1)}\cdot
  \beta^L$. A computation that has a result less than that (in
  absolute value) causes a condition called \indexterm{underflow}. In
  fact, most computers use \indexterm{normalized floating point
    numbers}: the first digit $d_1$ is taken to be nonzero; see
  section~\ref{sec:machine-eps} for more about this. In this case, any
  number less than $1\cdot\beta^L$ causes underflow. Trying
  to compute a number less than that in absolute value is sometimes
  handled by using \indexterm{unnormalized floating point numbers}
  (a~process known as \indexterm{gradual underflow}), but this is
  typically tens or hundreds of times slower than computing with
  regular floating point numbers\footnote{In real-time applications
    such as audio processing this phenomenon is especially noticable;
    see \url{http://phonophunk.com/articles/pentium4-denormalization.php?pg=3}.}. At
  the time of this writing, only the IBM\index{IBM} Power6 has
  hardware support for gradual underflow.
\end{itemize}
The fact that only a small number of real numbers can be represented
exactly is the basis of the field of round-off error analysis. We will
study this in some detail in the following sections.

For detailed discussions, see the book by
Overton~\cite{Overton:754book}; it is easy to find online copies of
the essay by Goldberg~\cite{goldberg:floatingpoint}. For extensive
discussions of round-off error analysis in algorithms, see the books
by Higham~\cite{Higham:2002:ASN} and Wilkinson~\cite{Wilkinson:roundoff}.

\Level 1 {Normalized numbers}

The general definition of floating point numbers,
equation \eqref{eq:floatingpoint-def}, leaves us with the problem that numbers
have more than one representation. For instance,
$.5\times10^{2}=.05\times 10^3$. Since this would make computer
arithmetic needlessly complicated, for instance in testing equality of
numbers, we use \indexterm{normalized floating point
  numbers}. A~number is normalized if its first digit is nonzero.
The implies that the mantissa\index{mantissa} part is $\beta> x_m\geq 1$.

A practical implication in the case of binary numbers is that the
first digit is always~1, so we do not need to store it explicitly.
In the IEEE 754 standard, this means that every floating point number
is of the form
\[ 1.d_1d_2\ldots d_t\times 2^{\mathrm exp}.\]

\Level 1 {Representation error}

Let us consider a real number that is not representable in a
computer's number system.

An unrepresentable number is approximated either by
\indexterm{rounding} or \indexterm{truncation}.
This means that a machine number~$\tilde x$ is the representation for
all~$x$ in an interval around it.  With $t$ digits in the
mantissa\index{mantissa}, this is the interval of numbers that differ
from~$\bar x$ in the $t+1$st digit. For the mantissa\index{mantissa}
part we get:
\[
\begin{cases}
  x\in \left[\tilde x,\tilde x+\beta^{-t+1}\right)&\hbox{truncation}\\
  x\in \left[\tilde x-\frac12 \beta^{-t+1},\tilde x+\frac12 \beta^{-t+1}\right)
    &\hbox{rounding}
\end{cases}
\]

If $x$ is a number and $\tilde x$ its representation in the computer,
we call $x-\tilde x$ the \emph{representation error} or
\indextermsub{absolute}{representation error}, and $\frac{x-\tilde x}{x}$
the \indextermsub{relative}{representation error}. Often we are not
interested in the sign of the error, so we may apply the terms error
and relative error to $|x-\tilde x|$ and~$|\frac{x-\tilde x}{x}|$
respectively.

Often we are only interested in bounds on the error. If $\epsilon$ is
a bound on the error, we will write
\[ \tilde x = x\pm\epsilon \defined
    |x-\tilde x|\leq\epsilon 
    \Leftrightarrow \tilde x\in[x-\epsilon,x+\epsilon]
\]
For the relative error we note that
\[ \tilde x =x(1+\epsilon) \Leftrightarrow
    \left|\frac{\tilde x-x}{x}\right|\leq \epsilon
\]

Let us consider an example in decimal arithmetic, that is, $\beta=10$,
and with a 3-digit mantissa: $t=3$.  The number $x=1.256$ has a
representation that depends on whether we round or truncate: $\tilde
x_{\mathrm{round}}=1.26$, $\tilde x_{\mathrm{truncate}}=1.25$.
The error is in the 4th digit: if $\epsilon=x-\tilde x$ 
then $|\epsilon|<\beta^{t-1}$.
\begin{exercise}
The number in this example had no exponent part. What are the error
and relative error if there had been one?
\end{exercise}

\Level 1 {Machine precision}
\label{sec:machine-eps}

Often we are only interested in the order of magnitude of the
representation error,
and we will write $\tilde x=x(1+\epsilon)$, where~$|\epsilon|\leq\beta^{-t}$.
This maximum relative error is called the \indexterm{machine
  precision}, or sometimes \emph{machine epsilon}\index{machine
  epsilon|see{machine precision}}. Typical values are:
\[
\begin{cases}
  \epsilon\approx10^{-7}&\hbox{32-bit single precision}\\
  \epsilon\approx10^{-16}&\hbox{64-bit double precision}
\end{cases}
\]
Machine precision can be defined another way: $\epsilon$~is the
smallest number that can be added to~$1$ so that $1+\epsilon$ has a
different representation than~$1$. A~small example shows how aligning
exponents can shift a too small operand so that it is effectively ignored in
the addition operation:
\[
\begin{array}{cll}
   &1.0000&\times 10^0\\
  +&1.0000&\times 10^{-5}\\ \hline
\end{array}
\quad\Rightarrow\quad
\begin{array}{cll}
   &1.0000&\times 10^0\\
  +&0.00001&\times 10^0\\ \hline
  =&1.0000&\times 10^0
\end{array}
\]
Yet another way of looking at this
is to observe that, in the addition $x+y$, if the ratio of $x$ and~$y$
is too large, the result will be identical to~$x$.

The machine precision is the maximum attainable accuracy of
computations: it does not make sense to ask for more than 6-or-so
digits accuracy in single precision, or 15 in double.

\begin{exercise}
  Write a small program that computes the machine epsilon. Does it
  make any difference if you set the
  \indextermbus{compiler}{optimization levels} low or high? 
  %Can you find other ways in which this computation goes wrong?
\end{exercise}
\begin{exercise}
  The number $e\approx 2.72$, the base for the natural logarithm, has
  various definitions. One of them is 
  \[ e=\lim_{n\rightarrow\infty} (1+1/n)^n. \]
  Write a single precision program that tries to compute~$e$ in this
  manner. Evaluate the expression for $n=10^k$ with
  $k=1,\ldots,10$. Explain the output for large~$n$. Comment on the
  behaviour of the error.
\end{exercise}

\Level 1 {The IEEE 754 standard for floating point numbers}
\label{sec:ieee754}

\begin{figure}
  \[
  \begin{array}{| l || l || l |}
    \hline
    \hbox{sign}&\hbox{exponent}&\hbox{mantissa}\\
    \hline
    s&e_1\cdots e_8&s_1\dots s_{23}\\
    \hline
    31&30\cdots 23&22\cdots 0\\
    \hline
  \end{array}\quad
  \begin{array}{|r|r|}
    \hline
    (e_1\cdots e_8)&\hbox{numerical value}\\
    \hline
    (0\cdots0)=0& \pm 0.s_1\cdots s_{23}\times 2^{-126}\\
    \hline
    (0\cdots 01)=1& \pm 1.s_1\cdots s_{23}\times 2^{-126}\\
    \hline
    (0\cdots 010)=2& \pm 1.s_1\cdots s_{23}\times 2^{-125}\\
    \hline
    \cdots&\\
    \hline
    (01111111)=127& \pm 1.s_1\cdots s_{23}\times 2^{0}\\
    \hline
    (10000000)=128& \pm 1.s_1\cdots s_{23}\times 2^{1}\\
    \hline
    \cdots&\\
    \hline
    (11111110)=254& \pm 1.s_1\cdots s_{23}\times 2^{127}\\
    \hline
    (11111111)=255& \hbox{$\pm\infty$ if $s_1\cdots s_{23}=0$, otherwise \n{NaN}} \\
    \hline
  \end{array}
  \]
  \caption{Single precision arithmetic}
  \label{fig:single-precision}
\end{figure}

Some decades ago, issues like the length of the
mantissa\index{mantissa} and the rounding behaviour of operations
could differ between computer manufacturers, and even between models
from one manufacturer. This was obviously a bad situation from a point
of portability of codes and reproducibility of results. The IEEE
standard 754\footnote{IEEE 754 is a standard for binary arithmetic;
  there is a further standard, IEEE 854, that allows decimal
  arithmetic.}\footnote{`` It was remarkable that so many hardware
  people there, knowing how difficult p754 would be, agreed that it
  should benefit the community at large. If it encouraged the
  production of floating-point software and eased the development of
  reliable software, it would help create a larger market for
  everyone's hardware. This degree of altruism was so astonishing that
  MATLAB's creator Dr. Cleve Moler used to advise foreign visitors not
  to miss the country's two most awesome spectacles: the Grand Canyon,
  and meetings of IEEE p754.'' W. Kahan,
  \url{http://www.cs.berkeley.edu/~wkahan/ieee754status/754story.html}.}
codified all this, for instance stipulating 24 and 53 bits for the
mantissa\index{mantissa} in single and double precision arithmetic,
using a storage sequence of sign bit, exponent, mantissa. 
%% \footnote{Computer systems
%%   can still differ as to how to store successive bytes. If the
%%   \indexterm{least significant byte} is stored first, the system is
%%   called \indexterm{little-endian}; if the \indexterm{most significant
%%     byte} is stored first, it is called \indexterm{big-endian}. See
%%   \url{http://en.wikipedia.org/wiki/Endianness} for details.}.

The standard also declared the rounding behaviour
to be \indexterm{correct rounding}: the result of an operation should be the
rounded version of the exact result. There will be much more on the
influence of rounding (and truncation) on numerical computations, below.

Above (section~\ref{sec:exceptions}), we have seen the phenomena of
overflow and underflow, that is, operations leading to unrepresentable
numbers. There is a further exceptional situation that needs to be
dealt with: what result should be returned if the program asks for
illegal operations such as~$\sqrt{-4}$? The IEEE 754 standard has two
special quantities for this: \texttt{Inf} and~\texttt{NaN} for
`infinity' and `not a number'.  Infinity is the result of overflow or
dividing by zero, not-a-number is the result of, for instance,
subtracting infinity from infinity.  If \texttt{NaN} appears in an
expression, the whole expression will evaluate to that value. The rule
for computing with \texttt{Inf} is a bit more
complicated~\cite{goldberg:floatingpoint}.

An inventory of the meaning of all bit patterns in IEEE
754 double precision is given in figure~\ref{fig:single-precision}.
Note that for normalized numbers the first nonzero digit is a~1, which
is not stored, so the bit pattern $d_1d_2\ldots d_t$ is interpreted as
$1.d_1d_2\ldots d_t$.

\begin{exercise}
  Every programmer, at some point in their life, makes the mistake of
  storing a real number in an integer or the other way around. This
  can happen for instance if you call a function differently from how
  it was defined.
\begin{verbatim}
void a(float x) {....}
int main() {
  int i;
  .... a(i) ....
}
\end{verbatim}
What happens when you print \n{x} in the function? Consider the bit
pattern for a small integer, and use the table in
figure~\ref{fig:single-precision} to interpret it as a floating point
number. Explain that it will be an unnormalized number\footnote{This
  is one of those errors you won't forget after you make it. In the
  future, whenever you see a number on the order of $10^{-305}$ you'll
  recognize that you made this error.}.
\end{exercise}

These days, almost all processors adhere to the IEEE 754 standard,
with only occasional exceptions. For instance, Nvidia Tesla
\indexac{GPU}s are not standard-conforming in single precision;
see~\url{http://en.wikipedia.org/wiki/Nvidia_Tesla}. The justification
for this is that single precision is more likely used for graphics,
where exact compliance matters less. For many scientific computations,
double precision is necessary, since the precision of calculations
gets worse with increasing problem size or runtime. This is true for
the sort of calculations in chapter~\ref{ch:odepde}, but not for
others such as Lattice-Boltzmann.

\Level 0 {Round-off error analysis}
\index{round-off error analysis|(}

Numbers that are too large or too small to be represented, leading to
overflow and underflow, are
uncommon: usually computations can be arranged so that this situation
will not occur. By contrast, the case that the result of a computation
between computer numbers
(even something as simple as a single addition) 
is not representable is very common. Thus, looking at the
implementation of an algorithm, we need to analyze the
effect of such small errors propagating through the computation.
This is commonly called
\indexterm{round-off error analysis}.

\Level 1 {Correct rounding}

The IEEE 754 standard, mentioned in section~\ref{sec:ieee754}, does
not only declare the way a floating point number is stored, it also
gives a standard for the accuracy of operations such as addition,
subtraction, multiplication, division. The model for arithmetic in the
standard is that of \indexterm{correct rounding}: the result of an
operation should be as if the following procedure is followed:
\begin{itemize}
\item The exact result of the operation is computed, whether this is
  representable or not;
\item This result is then rounded to the nearest computer number.
\end{itemize}
In short: the representation of the result of an
operation is the rounded exact result of that operation. (Of course,
after two operations it no longer needs to hold that the computed
result is the exact rounded version of the exact result.)

If this statement sounds trivial or self-evident, consider subtraction
as an example. In a decimal number system with two digits in the
mantissa, the computation
$1.0-\fp{9.4}{-1}=1.0-0.94=0.06=\fp{0.6}{-2}$. Note that in an
intermediate step the mantissa $.094$ appears, which has one more
digit than the two we declared for our number system. The extra digit
is called a \indexterm{guard digit}.

Without a guard digit, this operation would have proceeded as
$1.0-\fp{9.4}{-1}$, where $\fp{9.4}{-1}$ would be rounded to~$0.9$,
giving a final result of~$0.1$, which is almost double the correct result.
\begin{exercise}
  Consider the computation $1.0-\fp{9.5}{-1}$, and assume again that
  numbers are rounded to fit the 2-digit mantissa. Why is this
  computation in a way a lot worse than the example?
\end{exercise}
One guard digit is not enough to guarantee correct rounding. An
analysis that we will not reproduce here shows that three extra bits
are needed~\cite{Goldberg:arithmetic}.

\Level 1 {Addition}

Addition of two floating point numbers is done in a couple of steps.
First the exponents are aligned: the smaller of the two numbers is
written to have the same exponent as the larger number. Then the
mantissas\index{mantissa} are added. Finally, the result is adjusted
so that it again is a normalized number.

As an example, consider $1.00+2.00\times 10^{-2}$. Aligning the
exponents, this becomes $1.00+0.02=1.02$, and this result requires no
final adjustment. We note that this computation was exact, but
the sum $1.00+2.55\times 10^{-2}$ has the same result, and here the
computation is clearly not exact: the exact result is $1.0255$, which
is not representable with three digits to the mantissa.

In the example $6.15\times 10^1+3.98\times 10^1=10.13\times 10^1=1.013\times
10^2\rightarrow 1.01\times 10^2$ we see that after addition of the mantissas an
adjustment of the exponent is needed. The error again comes from
truncating or rounding the first digit of the result that does not fit
in the mantissa: if $x$ is the true sum and $\tilde x$ the computed
sum, then $\tilde x=x(1+\epsilon)$ where, with a 3-digit mantissa
$|\epsilon|<10^{-3}$.

Formally, let us consider the computation of
$s=x_1+x_2$, and we assume that the numbers~$x_i$ are represented
as $\tilde x_i=x_i(1+\epsilon_i)$.
Then the sum~$s$ is represented as
\[ 
\begin{array}{rl}
\tilde s&=(\tilde x_1+\tilde x_2)(1+\epsilon_3)\\
&=x_1(1+\epsilon_1)(1+\epsilon_3)+x_2(1+\epsilon_2)(1+\epsilon_3)\\
&\approx x_1(1+\epsilon_1+\epsilon_3)+x_2(1+\epsilon_2+\epsilon_3)\\
&\approx s(1+2\epsilon)
\end{array}
\]
under the assumptions that all~$\epsilon_i$ are small and of roughly
equal size, and that both $x_i>0$.
We see that the relative errors are added under addition.

\Level 1 {Multiplication}

Floating point multiplication, like addition, involves several steps.
In order to multiply two numbers $m_1\times\beta^{e_1}$
and~$m_2\times\beta^{e_2}$, the following steps are needed.
\begin{itemize}
\item The exponents are added: $e\leftarrow e_1+e_2$.
\item The mantissas\index{mantissa} are multiplied: $m\leftarrow
  m_1\times m_2$.
\item The mantissa is normalized, and the exponent adjusted accordingly.
\end{itemize}

For example: $\fp{1.23}{0}\times\fp{5.67}1=\fp{0.69741}1\rightarrow
\fp{6.9741}0\rightarrow\fp{6.97}0$.

\begin{exercise}
  Analyze the relative error of multiplication.
\end{exercise}

\Level 1 {Subtraction}
\label{sec:subtraction}

Subtraction behaves very differently from addition. Whereas in
addition errors are added, giving only a gradual increase of overall
roundoff error, subtraction has the potential for greatly increased
error in a single operation. 

For example, consider subtraction with 3 digits to the mantissa:
$1.24-1.23=.001\rightarrow \fp{1.00}{-2}$. While the result is exact,
it has only one significant digit\footnote {Normally, a number with 3
  digits to the mantissa suggests an error corresponding to rounding
  or truncating the fourth digit. We say that such a number has 3
  \indexterm{significant digits}. In this case, the last two digits have no
  meaning, resulting from the normalization process.}. To see this, consider
the case where the first operand~$1.24$ is actually the rounded result
of a computation that should have resulted in~$1.235$. In that case,
the result of the subtraction should have been~$\fp{5.00}{-3}$, that
is, there is a 100\% error, even though the relative error of the
inputs was as small as could be expected. Clearly, subsequent
operations involving the result of this subtraction will also be
inaccurate.
We conclude that subtracting almost equal numbers is a likely cause of
numerical roundoff.

There are some subtleties about this example. Subtraction of almost
equal numbers is exact, and we have the correct rounding behaviour of
IEEE arithmetic. Still, the correctness of a single operation does not
imply that a sequence of operations containing it will be
accurate. While the addition example showed only modest decrease of
numerical accuracy, the cancellation in this example can have
disastrous effects.

\begin{comment}
Exercise: sine function through power series. How to deal with the
alternating signs?
\end{comment}

\Level 1 {Examples}

From the above, the reader may got the impression that roundoff errors
only lead to serious problems in exceptional circumstances. In this
section we will discuss some very practical examples where the
inexactness of computer arithmetic becomes visible in the result of a
computation. These will be fairly simple examples; more complicated
examples exist that are outside the scope of this book, such as the
instability of matrix inversion. The interested reader is referred
to~\cite{Wilkinson:roundoff,Higham:2002:ASN}.

\Level 2 {The `abc-formula'}

As a practical example, consider the quadratic equation $ax^2+bx+c=0$ 
which has solutions $x=\frac{-b\pm\sqrt{b^2-4ac}}{2a}$.
Suppose $b>0$ and $b^2\gg 4ac$ then $\sqrt{b^2-4ac}\approx b$ and
the `$+$' solution will be
inaccurate. In this case it is better 
to compute $x_-=\frac{-b-\sqrt{b^2-4ac}}{2a}$ and use $x_+\cdot x_-=-c/a$.

\begin{exercise}
  Program a simulator for decimal $d$-digit arithmetic and experiment with the
  accuracy of the two ways of computing the solution of a quadratic
  equation. Simulating $d$-digit decimal arithmetic can be done as
  follows. Let $x$ be a floating point number, then:
  \begin{itemize}
  \item Normalize $x$ by finding an integer $e$ such that
    $x'\mathrel{:=} |x|\cdot 10^e\in[.1,1)$. 
    \item Now truncate this number to $d$ digits by 
      multiplying $x'$ by $10^d$, truncating the result to an integer,
      and multiplying that result again by~$10^{-d}$.
    \item Multiply this truncated number by~$10^{-e}$ to revert the
      normalization.
  \end{itemize}
\end{exercise}

\Level 2 {Summing series}

The previous example was about preventing a large roundoff error in a
single operation. This example shows that even gradual buildup of
roundoff error can be handled in different ways.

Consider the sum $\sum_{n=1}^{10000}\frac{1}{n^2}=1.644834$
and assume we are working with single precision, which on most computers
means a machine precision of~$10^{-7}$. The problem with this example
is that both the ratio between terms, and the ratio of terms to
partial sums, is ever increasing. In section~\ref{sec:machine-eps} we
observed that a too large ratio can lead to one operand of an addition in
effect being ignored.

If we sum the series in the sequence it is given, we observe that
the first term is~1, so all partial sums ($\sum_{n=1}^N$~where $N<10000$)
are at least~1. This means that any term where $1/n^2<10^{-7}$ gets
ignored since it is less than the machine precision.
Specifically, the last 7000 terms are ignored, and
the computed sum is~$1.644725$. The first 4 digits are correct.

However, if we evaluate the sum in reverse order
we obtain the exact result in single precision. We are still adding
small quantities to larger ones, but now the ratio will never be as
bad as one-to-$\epsilon$, so the smaller number is never ignored.
To see this,
consider the ratio of two terms subsequent terms:
\[ \frac{n^2}{(n-1)^2}=\frac{n^2}{n^2-2n+1}=\frac1{1-2/n+1/n^2}
    \approx 1+\frac2n
\]
Since we only sum $10^5$ terms and the machine precision is $10^{-7}$, 
in the addition $1/n^2+1/(n-1)^2$ the second term will not be wholly
ignored as it is when we sum from large to small.
\begin{exercise}
  There is still a step missing in our reasoning. We have shown that
  in adding two subsequent terms, the smaller one is not
  ignored. However, during the calculation we add partial sums to the
  next term in the sequence. Show that this does not worsen the situation.
\end{exercise}

The lesson here is that series that are monotone (or close to
monotone) should be summed from small to large, since the error is
minimized if the quantities to be added are closer in magnitude. Note
that this is the opposite strategy from the case of subtraction, where
operations involving similar quantities lead to larger errors. This
implies that if an application asks for adding and subtracting series
of numbers, and we know a priori which terms are positive and
negative, it may pay off to rearrange the algorithm accordingly.

% with aligned exponents:
% \begin{tabular}{rr|cl}
%   $n-1$:&$.00\cdots0$&$10\cdots00$\\
%   $n$:&  $.00\cdots0$&$00\cdots01$&$0\cdots0$\\
%       &             &$n$ bits
% \end{tabular}
% The last digit in the smaller number is not lost if $n<1/\epsilon$

\Level 2 {Unstable algorithms}

We will now consider an example where we can give a direct argument
that the algorithm can not cope with problems due to inexactly
represented real numbers.

Consider the recurrence $y_n=\int_0^1 \frac{x^n}{x-5}dx =
\frac1n-5y_{n-1}$.%\marginpar{Negative? check this}
This is easily seen to be monotonically decreasing; the first term can
be computed as~$y_0=\ln 6 - \ln 5$.

Performing the computation in 3 decimal digits we get:

\begin{tabular}{lll}
  computation&&correct result\\
  $y_0=\ln 6 - \ln 5=.182|322\times 10^{1}\ldots$&&1.82\\
  $y_1=.900\times 10^{-1}$&&.884\\
  $y_2=.500\times 10^{-1}$&&.0580\\
  $y_3=.830\times 10^{-1}$&going up?&.0431\\
  $y_4=-.165$&negative?&.0343
\end{tabular}

We see that the computed results are quickly not just inaccurate, but
actually nonsensical. We can analyze why this is the case.

If we define the error $\epsilon_n$ in the $n$-th step as
\[ \tilde y_n-y_n=\epsilon_n,\] then
\[ \tilde y_n=1/n-5\tilde y_{n-1}=1/n+5n_{n-1}+5\epsilon_{n-1}
    = y_n+5\epsilon_{n-1} \]
so $\epsilon_n\geq 5\epsilon_{n-1}$. The error made by this
computation shows exponential growth.
%\marginpar{Show stability of backwards computation}

\Level 2 {Linear system solving}

Sometimes we can make statements about the numerical precision of a
problem even without specifying what algorithm we use. Suppose we want
to solve a linear system, that is, we have an $n\times n$ matrix~$A$
and a vector~$b$ of size~$n$, and we want to compute the vector~$x$
such that $Ax=b$. (We will actually considering algorithms for this in
chapter~\ref{ch:linear}.) Since the vector~$b$ will the result of some
computation or measurement, we are actually dealing with a
vector~$\tilde b$, which is some perturbation of the ideal~$b$:
\[ \tilde b =  b+\Delta b. \]
The perturbation vector~$\Delta b$ can be of the order of the machine
precision if it only arises from representation error, or it can be
larger, depending on the calculations that produced~$\tilde b$.

We now ask what the relation is between the exact value of~$x$, which
we would have obtained from doing an exact calculation with $A$
and~$b$, which is clearly impossible, and
the computed value~$\tilde x$, which we get from computing with $A$
and~$\tilde b$. (In this discussion we will assume that $A$ itself is
exact, but this is a simplification.)

Writing $\tilde x= x+\Delta x$, the result of our computation is now
\[ A\tilde x = \tilde b \] or \[ A(x+\Delta x)=b+\Delta b. \]
Since $Ax=b$, we get $A\Delta x=\Delta b$. From this, we get
(see appendix~\ref{app:norms} for details)
\begin{equation}
 \left \{
\begin{array}{rl}
  \Delta x&=A\inv \Delta b\\ Ax&=b
\end{array} \right\} \Rightarrow \left\{
\begin{array}{rl}
  \|A\| \|x\|&\geq\|b\| \\ \|\Delta x\|&\leq \|A\inv\| \|\Delta b\|
\end{array} \right.
\Rightarrow
\frac{\|\Delta x\|}{\|x\|}
\leq 
\|A\| \|A\inv\| \frac{\|\Delta b\|}{\|b\|}
    \label{eq:xbound}
\end{equation}
The quantity $\|A\| \|A\inv\|$ is called the \indexterm{condition
  number} of a matrix. The bound \eqref{eq:xbound} then says that any
perturbation in the right hand side can lead to a perturbation in the
solution that is at most larger by the condition number of the
matrix~$A$. Note that it does not say that the perturbation in~$x$
\emph{needs} to be anywhere close to that size, but we can not rule it
out, and in some cases it indeed happens that this bound is attained.

Suppose that $b$ is exact up to machine precision, and the
condition number of~$A$ is~$10^4$. The bound \eqref{eq:xbound} is
often interpreted as saying that the last 4 digits of~$x$ are
unreliable, or that the computation `loses 4 digits of accuracy'.

Equation~\eqref{eq:xbound} can also be interpreted as follows: when we
solve a linear system $Ax=b$ we get an approximate solution $x+\Delta
x$ which is the \emph{exact} solution of a perturbed system
$A(x+\Delta x)=b+\Delta b$. The fact that the perturbation in the
solution can be related to the perturbation in the system, is
expressed by saying that the algorithm exhibits \indexterm{backwards
  stability}.

The analysis of the accuracy of linear algebra algorithms is a field
of study in itself; see for instance the book by
Higham~\cite{Higham:2002:ASN}.

\Level 1 {Roundoff error in parallel computations}

From the above example of summing a series we saw that addition in
computer arithmetic is not associative. A~similar fact holds for
multiplication. 
This has an interesting consequence for parallel computations: the way
a computation is spread over parallel processors influences the
result. For instance, consider computing the sum of a large number~$N$ of
terms. With $P$~processors at our disposition, we can let each compute
$N/P$ terms, and combine the partial results. We immediately see that
for no two values of~$P$ will the results be identical. This means
that reproducibility of results in a parallel context is elusive.

\index{round-off error analysis|)}

\Level 0 {More about floating point arithmetic}

\Level 1 {Programming languages}

Different languages have different approaches to declaring integers and
floating point numbers.
\begin{itemize}
\item [Fortran]\index{Fortran!declarations in} In Fortran there are
  various ways of specifying the storage format for integer and real
  variables. For instance, it is possible to declare
   the number of bytes that it takes to store a variable:
  \texttt{INTEGER*2, REAL*8}. One advantage of this approach is the
  easy interoperability with other languages, or the MPI library.

  Often it is possible to write a code
  using only \texttt{INTEGER, REAL}, and use
  \indextermbus{compiler}{flags} to indicate the size of an integer
  and real number in bytes.

  More sophisticated, modern versions of Fortran can indicate the
  number of digits of precision a floating point number needs to have:
\begin{verbatim}
integer, parameter :: k9 = selected_real_kind(9)
real(kind=k9) :: r
r = 2._k9; print *, sqrt(r) ! prints 1.4142135623730
\end{verbatim}
  The `kind' values will usually be 4,8,16 but this is compiler
  dependent.
\item [C] In C, the type identifiers have no standard length. For
  integers there is \texttt{short int, int, long int}, and for
  floating point \texttt{float, double}. The \texttt{sizeof()}
  operator gives the number of bytes used to store a datatype.
\item [C99, Fortran2003] Recent standards of the C and Fortran
  languages incorporate the C/Fortran interoperability standard, which
  can be used to declare a type in one language so that it is
  compatible with a certain type in the other language.
\end{itemize}

\Level 1 {Other computer arithmetic systems}

Other systems have been proposed to dealing with the problems of
inexact arithmetic on computers. One solution is extended precision
arithmetic, where numbers are stored in more bits than usual. A common
use of this is in the calculation of inner products of vectors: the
accumulation is internally performed in extended precision, but
returned as a regular floating point number. Alternatively, there are
libraries such as GMPlib~\cite{gmplib} that allow for any calculation
to be performed in higher precision.

Another solution to the imprecisions of computer arithmetic is `interval
arithmetic'~\cite{wikipedia:interval-arithmetic}, where for each
calculation interval bounds are maintained. While this has been
researched for considerable time, it is not practically used other
than through specialized libraries~\cite{boost:interval-arithmetic}.

\Level 1 {Fixed-point arithmetic}

A fixed-point number (for a more thorough discussion than found here,
see~\cite{YatesFixedPoint}) can be represented as $\fxp NF$ where
$N\geq\beta^0$ is the integer part and $F<1$ is the fractional
part. Another way of looking at this, is that a fixed-point number is
an integer stored in $N+F$ digits, with an implied decimal point after
the first $N$ digits.

Fixed-point calculations can overflow, with no possibility to adjust
an exponent. Consider the multiplication $\fxp{N_1}{F_1}\times
\fxp{N_2}{F_2}$, where $N_1\geq \beta^{n_1}$ and $N_2\geq
\beta^{n_2}$. This overflows if $n_1+n_2$ is more than the number of
positions available for the integer part. (Informally, the number of
digits of the product is the sum of the digits of the operands.)
This means that, in a program
that uses fixed-point, numbers will need to have a number of zero
digits, if you are ever going to multiply them, 
which lowers the numerical accuracy.
It also means that the programmer has to think harder about
calculations, arranging them in such a way that overflow will not
occur, and that numerical accuracy is still preserved to a reasonable
extent.

So why would people use fixed-point numbers? One important application
is in embedded low-power devices, think a battery-powered digital
thermometer. Since fixed-point calculations are essentially identical
to integer calculations, they do not require a floating-point unit,
thereby lowering chip size and lessening power demands. Also, many
early video game systems had a processor that either had no
floating-point unit, or where the integer unit was considerably faster
than the floating-point unit. In both cases, implementing non-integer
calculations as fixed-point, using the integer unit, was the key to
high throughput. 

Another area where fixed point arithmetic is still used is in signal
processing. In modern CPUs, integer and floating point operations
are of essentially the same speed, but converting between them is
relatively slow. Now, if the sine function is implemented through
table lookup, this means that in $\sin(\sin x)$ 
the output of a function is used to index the next function
application. Obviously, outputting the sine function in fixed point
obviates the need for conversion between real and integer quantities,
which simplifies the chip logic needed, and speeds up calculations.

\Level 1 {Complex numbers}
\label{sec:complex}

Some programming languages have complex numbers as a native data type,
others not, and others are in between. For instance, in Fortran you
can declare
\begin{verbatim}
COMPLEX z1,z2, z(32)
COMPLEX*16 zz1, zz2, zz(36)
\end{verbatim}
A complex number is a pair of real numbers, the real and imaginary
part, allocated adjacent in memory. The first declaration then uses
8~bytes to store to \n{REAL*4} numbers, the second one has \n{REAL*8}s
for the real and imaginary part.  (Alternatively, use \n{DOUBLE
  COMPLEX} or in Fortran90 \n{COMPLEX(KIND=2)} for the second line.)

By contrast, the \n{C} language does not natively have complex
numbers, but both \n{C99} and \n{C++} have a \n{complex.h} header
file\footnote {These two header files are not identical, and in fact
  not compatible. Beware, if you compile C code with a C++
  compiler~\cite{DobbsComplex}.}. This defines as complex number as in
Fortran, as two real numbers.

Storing a complex number like this is easy, but sometimes it is
computationally not the best solution. This becomes apparent when we
look at arrays of complex numbers.
If a computation often relies
on access to the real (or imaginary) parts of complex numbers
exclusively, striding through an array of complex numbers, has a
stride two, which is disadvantageous (see
section~\ref{sec:stride}). In this case, it is better to allocate one
array for the real parts, and another for the imaginary parts.

\begin{exercise}
  Suppose arrays of complex numbers are stored the Fortran
  way. Analyze the memory access pattern of pairwise multiplying the
  arrays, that is, $\forall_i\colon c_i\leftarrow a_i\cdot b_i$, where
  \texttt{a(), b(), c()} are arrays of complex numbers.
\end{exercise}

\begin{exercise}
  Show that an $n\times n$ linear system $Ax=b$ over the complex numbers
  can be written as a $2n\times 2n$ system over the real
  numbers. Hint: split the matrix and the vectors in their real and
  imaginary parts. Argue for the efficiency of storing arrays of
  complex numbers as separate arrays for the real and imaginary parts.
\end{exercise}

\Level 0 {Conclusions}

Computations done on a computer are invariably beset with numerical error.
In a way, the reason for the error is the imperfection of computer
arithmetic: if we could calculate with actual real numbers there would
be no problem. (There would still be the matter of measurement error
in data, and approximations made in numerical methods; see the next
chapter.) However, if we accept roundoff as a fact of life, then
various observations hold:
\begin{itemize}
\item Mathematically equivalent operations need not behave identically
  from a point of stability; see the `abc-formula' example.
\item Even rearrangements of the same computations do not behave
  identically; see the summing example.
\end{itemize}
Thus it
becomes imperative to analyze computer algorithms with regard to their
roundoff behaviour: does roundoff increase as a slowly growing
function of problem parameters, such as the number of terms evaluted,
or is worse behaviour possible? We will not address such questions in
further detail in this book.

