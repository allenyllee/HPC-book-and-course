%%%%%%%%%%%%%%%%%%%%%%%%%%%%%%%%%%%%%%%%%%%%%%%%%%%%%%%%%%%%%%%%
%%%%%%%%%%%%%%%%%%%%%%%%%%%%%%%%%%%%%%%%%%%%%%%%%%%%%%%%%%%%%%%%
%%%%
%%%% This text file is part of the source of 
%%%% `Introduction to High-Performance Scientific Computing'
%%%% by Victor Eijkhout, copyright 2012
%%%%
%%%% This book is distributed under a Creative Commons Attribution 3.0
%%%% Unported (CC BY 3.0) license and made possible by funding from
%%%% The Saylor Foundation \url{http://www.saylor.org}.
%%%%
%%%%
%%%%%%%%%%%%%%%%%%%%%%%%%%%%%%%%%%%%%%%%%%%%%%%%%%%%%%%%%%%%%%%%
%%%%%%%%%%%%%%%%%%%%%%%%%%%%%%%%%%%%%%%%%%%%%%%%%%%%%%%%%%%%%%%%

In this section we will take a short look at terms such as
\emph{cloud computing}, 
and an earlier term
\emph{distributed computing}.
These are concepts that have a
relation to parallel computing in the scientific sense, but that
differ in certain fundamental ways.

Distributed computing can be traced back
as coming from large database servers, such as
airline reservations systems, which had to be accessed by many travel
agents simultaneously. For a large enough volume of database
accesses a single server will not suffice, so the mechanism of
\indexterm{remote procedure call} was invented, where the central
server would call code (the procedure in question) on a different
(remote) machine. The remote call could involve transfer of data, the
data could be already on the remote machine, or there would be some
mechanism that data on the two machines would stay synchronized. This
gave rise to the \indexac{SAN}. A~generation later than distributed
database systems, web servers had to deal with the same problem of
many simultaneous accesses to what had to act like a single server.

We already see one big difference between distributed computing and
high performance parallel computing. Scientific computing needs
parallelism because a single simulation becomes too big or slow for
one machine; the business applications sketched above deal with many
users executing small programs (that is, database or web queries)
against a large data set. For scientific needs, the processors of a
parallel machine (the nodes in a cluster) have to have a very fast
connection to each other; for business needs no such network is
needed, as long as the central dataset stays coherent.

Both in \indexac{HPC} and in business computing, the server has to stay
available and operative, but in distributed computing there is
considerably more liberty in how to realize this. For a user
connecting to a service such as a database, it does not matter what
actual server executes their request. Therefore, distributed computing
can make use of \indexterm{virtualization}: a virtual server can be
spawned off on any piece of hardware.

An analogy can be made between remote servers, which supply computing
power wherever it is needed, and the electric grid, which supplies
electric power wherever it is needed. This has led to \indexterm{grid
  computing} or \indexterm{utility computing}, with the Teragrid,
owned by the US National Science Foundation, as an example. Grid
computing was originally intended as a way of hooking up computers
connected by a \indexac{LAN} or \indexac{WAN}, often the Internet. The machines
could be parallel themselves, and were often owned by different
institutions. More recently, it has been viewed as a way of sharing
resources, both datasets, software resources, and scientific
instruments, over the network.

The notion of utility computing as a way of making services available,
which you recognize from the above description of distributed
computing, went mainstream
with Google's search engine, which
indexes the whole of the Internet. Another example is the GPS capability
of Android mobile phones, which combines GIS, GPS, and mashup
data. The computing model by which Google's gathers and processes data
has been formalized in
MapReduce~\cite{Google:mapreduce}. It combines a data parallel aspect (the
`map' part) and a central accumulation part (`reduce'). Neither
involves the tightly coupled neighbour-to-neighbour communication that
is common in scientific computing.
%
An open
source framework for MapReduce computing exists in
Hadoop~\cite{Hadoop-wiki}. Amazon offers a commercial Hadoop service.

The concept of having a remote computer serve user needs is attractive
even if no large datasets are involved, since it absolves the user
from the need of maintaining software on their local machine. Thus,
Google Docs\index{Google!Google Docs} offers various `office'
applications without the user actually installing any software. This
idea is sometimes called \indexac{SaaS}, where
the user connects to an `application server', and accesses it through
a client such as a web browser. In the case of Google Docs, there is
no longer a large central dataset, but each user interacts with their
own data, maintained on Google's servers. This of course has the large
advantage that the data is available from anywhere the user has access
to a web browser.

The \ac{SaaS} concept has several connections to earlier
technologies. For instance, after the mainframe and workstation eras,
the so-called \indexterm{thin client} idea was briefly popular. Here,
the user would have a workstation rather than a terminal, yet work on
data stored on a central server. One product along these lines was
Sun's \indextermbus{Sun}{Ray} (circa 1999) where users relied on a smartcard to
establish their local environment on an arbitary, otherwise stateless,
workstation.

The term \indexterm{cloud computing} usually refers to the
internet-based model of \ac{SaaS}
where the data is not maintained by the user.
However, it can span some or all of the above
concepts, depending on who uses the term. Here is a list of
characteristics:
\begin{itemize}
\item A cloud is centrally administered and maintained, involving
  applications running on servers that are not owned by the user. The
  user has local and homogeneous access to services on a subscription
  basis, as pay-as-you-go.
\item Cloud computing is typically associated with large amounts of
  data, either a single central dataset such as on airline database
  server, or many independent datasets such as for Google
  Docs\index{Google!Google Docs}, each of which are used by a single
  user or a small group of users. In the case of large datasets, they
  are stored distributedly, with concurrent access for the clients.
\item Cloud computing, like earlier grid computing, allows the user
  freedom of choice in the client platform. In an extreme example,
  Amazon's Kindle books allow one to read the same book on a PC, and a
  smartphone; the cloud-stored book `remembers' where the reader left
  off, regardless the platform.
\item Cloud computing makes a whole datacenter appear as a single
  computer to the user~\cite{Patterson2008datacenter}.
\item The services offered by cloud computing are typically business
  applications and IT services, rather than scientific
  computing. Cloud service providers are starting to offer clusters
  that allow tightly coupled parallelism, but tests show that this is
  not yet efficient.
\item Computing in a cloud is probably virtualized, or at least the
  client interfaces to an abstract notion of a server. These
  strategies often serve to `move the work to the data'.
\item Server processes are loosely coupled, at best synchronized
  through working on the same dataset.
\item Cloud computing can be interfaced through a web browser; it can
  involve a business model that is `pay as you go'.
\end{itemize}

Cloud computing clearly depends on the following factors:
\begin{itemize}
\item The ubiquity of the internet;
\item Virtualization of servers;
\item Commoditization of processors and hard drives.
\end{itemize}

The model where services are available on demand is attractive for
businesses, which increasingly are using cloud services. The
advantages are that it requires no initial monetary and time
investment, and that no decisions about type and size of equipment
have to be made. At the moment, cloud services are mostly focused on
databases and office applications, but scientific clouds with a high
performance interconnect are under development.

The following is a broad classification of usage scenarios of cloud
resources\footnote{Based on a blog post by Ricky Ho:
  \url{http://blogs.globallogic.com/five-cloud-computing-patterns}.}.
\begin{itemize}
\item Scaling. Here the cloud resources are used as a platform that
  can be expanded based on user demand. This can be considered
  Platform-as-a-Service (PaaS): the cloud provides software and development platforms,
  eliminating the administration and maintenance for the user. 

  We can distinguish between two cases: if the user is running single
  jobs and is actively waiting for the output, resources can be added
  to minimize the wait time for these jobs (capability computing). On
  the other hand, if the user is submitting jobs to a queue and the
  time-to-completion of any given job is not crucial (capacity
  computing), resouces can be added as the queue grows.

  In HPC applications, users can consider the cloud resources as a
  cluster; this falls under Infrastructure-as-a-Service (IaaS): the
  cloud service is a computing platforms allowing 
  customization at the operating system level.
\item Multi-tenancy. Here the same software is offered to multiple
  users, giving each the opportunity for individual
  customizations. This falls under Software-as-a-Service (SaaS):
  software is provided on demand; the customer does not purchase
  software, but only pays for its use.
\item Batch processing. This is a limited version of one of the Scaling
  scenarios above: the user has a large amount of data to process in
  batch mode. The
  cloud then becomes a batch processor. This model is a good candidate
  for MapReduce computations; section~\ref{sec:mapreduce}.
\item Storage. Most cloud providers offer database services, so this
  model absolves the user from maintaining their own database, just
  like the Scaling and Batch processing models take away the user's
  concern with maintaining cluster hardware.
\item Synchronization. This model is popular for commercial user
  applications. Netflix and Amazon's Kindle allow users to consume
  online content (streaming movies and ebooks\index{ebook}
  respectively); after pausing the content they can resume from any
  other platform. Apple's recent iCloud\index{Apple!iCloud} provides
  synchronization for data in office applications, but unlike Google
  Docs\index{Google!Google Docs} the applications are not `in the
  cloud' but on the user machine.
\end{itemize}

The first Cloud to be publicly accessible was Amazon's Elastic Compute 
cloud (EC2), launched in 2006. EC2 offers a variety of different 
computing platforms and storage facilities. Nowadays
more than a hundred companies provide cloud based services, well beyond 
the initial concept of computers-for-rent.

The infrastructure for cloud computing can be interesting from a
computer science point of view, involving distributed file systems,
scheduling, virtualization, and mechanisms for ensuring high
reliability.

An interesting project, combining aspects of grid and cloud computing
is the Canadian Advanced Network For Astronomical
Research\cite{canfar-lecture}. Here large central datasets are being
made available to astronomers as in a grid, together with compute
resources to perform analysis on them, in a cloud-like
manner. Interestingly, the cloud resources even take the form of
user-configurable virtual clusters.
