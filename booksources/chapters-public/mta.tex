%%%%%%%%%%%%%%%%%%%%%%%%%%%%%%%%%%%%%%%%%%%%%%%%%%%%%%%%%%%%%%%%
%%%%%%%%%%%%%%%%%%%%%%%%%%%%%%%%%%%%%%%%%%%%%%%%%%%%%%%%%%%%%%%%
%%%%
%%%% This text file is part of the source of 
%%%% `Introduction to High-Performance Scientific Computing'
%%%% by Victor Eijkhout, copyright 2012
%%%%
%%%% This book is distributed under a Creative Commons Attribution 3.0
%%%% Unported (CC BY 3.0) license and made possible by funding from
%%%% The Saylor Foundation \url{http://www.saylor.org}.
%%%%
%%%%
%%%%%%%%%%%%%%%%%%%%%%%%%%%%%%%%%%%%%%%%%%%%%%%%%%%%%%%%%%%%%%%%
%%%%%%%%%%%%%%%%%%%%%%%%%%%%%%%%%%%%%%%%%%%%%%%%%%%%%%%%%%%%%%%%

The architecture of modern CPUs is largely dictated by the fact that
getting data from memory is much slower than processing it. Hence, a
hierarchy of ever faster and smaller tries to keep data as close to
the processing unit as possible, mitigating the long latency and small
bandwidth of main memory. The \ac{ILP} in the processing unit 
also helps to hide the latency and more fully utilize the available
bandwidth. 

However, finding \ac{ILP} is a job for the compiler and
there is a limit to what it can practically find. On the other hand,
scientific codes are often very \indexterm{data parallel} in a sense
that is obvious to the programmer, though not necessarily to the
compiler. Would it be possible for the programmer to specify this
parallelism explicitly and for the processor to use it?

In section~\ref{sec:simd} you saw that \ac{SIMD} architectures can be
programmed in an explicitly data parallel way. What if we have a great
deal of data parallelism but not that many processing units? In that
case, we could turn the parallel instruction streams into threads (see
section~\ref{sec:threads}) and have multiple threads be executed on
each processing unit. Whenever a tread would stall because of an
outstanding memory request, the processor could switch to another
thread for which all the necessary inputs are available. This is
called \indexterm{multi-threading}. While it sounds like a way of
preventing the processor from waiting for memory, it can also be
viewed as a way of keeping memory maximally occupied.

\begin{exercise}
  If you consider the long latency and limited bandwidth of memory as
  two separate problems, does multi-threading address them both?
\end{exercise}

The problem here is that most CPUs are not good
at switching quickly between threads. A~\indextermbus{context}{switch}
(switching between one thread and another) takes a large number of
cycles, comparable to a wait for data from main memory. In a so-called
\indexacf{MTA} a context-switch is very efficient, sometimes as little
as a single
cycle, which makes it possible for one processor to work on many
threads simultaneously.

The multi-threaded concept was explored in the \indextermbus{Tera
  Computer}{MTA} machine, which evolved into the current
\indextermbus{Cray}{XMT}\footnote{Tera Computer renamed itself
  \indexterm{Cray Inc.} after acquiring \indexterm{Cray Research} from
  \indexterm{SGI}.}.

The other example of an \ac{MTA} is the \ac{GPU}, where the processors
work as \ac{SIMD} units, while being themselves multi-threaded; see
section~\ref{sec:gpu}.
