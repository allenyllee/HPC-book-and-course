\label{sec:sorting}
\index{sorting|(}

Sorting is not a common operation in scientific computing: one expects
it to be more important in databases, whether these be financial or
biological (for instance in sequence alignment). However, it sometimes
comes up, for instance in \indexac{AMR} and other applications where
significant manipulations of data structures occurs.

In this section we will briefly look at the QuickSort
algorithm and how it can be done in parallel. For more details,
see~\cite{Kumar:parcomp-book} and the references therein.

\Level 0 {Brief introduction to sorting}

There are many sorting algorithms. One way to distinguish them is by
their computational complexity, that is, given an array of $n$
elements, how many operations does it take to sort them, as a function
of~$n$. For some sorting algorithms, the answer to this question is
not simple to give. While some algorithms work largely independent of
the state of the input data, for others, the operation count does
depend on it. One could imagine that a sorting algorithm would make a
pass over the data to check if it was already sorted. In that case,
the complexity is~$n$ operations for a sorted list, and something
higher for an unsorted list.

The so-called \indexterm{bubble sort} algorithm has a complexity
independent of the data to be sorted. This algorithm is given by:

\begin{displayalgorithm}
  \For{$t$ from $n-1$ down to $1$}{
    \For{$e$ from 1 to $t$}{
      \If{elements $e$ and $e+1$ are ordered the wrong way}{exchange
      them}
    }
  }
  \caption{The bubble sort algorithm}
\end{displayalgorithm}

It is easy to see that this algorithm has a complexity of~$O(n^2)$:
the inner loop does $t$ comparisons and up to $t$ exchanges. Summing
this from $1$ to $n-1$ gives approximately $n^2/2$ comparisons and a
at most the same number of exchanges.

Theoretically one can show that a sorting algorithm has to have at
least complexity~$O(n\log n)$\footnote{One can consider a sorting
  algorithm as a decision tree: a first comparison is made, depending
  on it two other comparisons are made, et cetera. Thus, an actual
  sorting becomes a path through this decision tree. If every path has
  running time~$h$, the tree has $2^h$ nodes. Since a sequence of $n$
  elements can be ordered in $n!$ ways, the tree needs to have enough
  paths to accomodate all of these; in other words, $2^h\geq
  n!$. Using Stirling's formula, this means that $n\geq O(n\log
  n)$}. There are indeed several algorithms that are guaranteed to
attain this complexity, but a very popular algorithm, called
\indexterm{quicksort} has only an `expected' complexity of~$O(n\log
n)$, and a worst case complexity of~$O(n^2)$.

\Level 0 {Quicksort} 
\index{quicksort|(}

Quicksort is a recursive algorithm, that, unlike bubble sort, is not
deterministic. It is a two step procedure, based on a reordering of
the sequence\footnote{The name is explained by its origin with the
  Dutch computer scientist Edsger Dijkstra; see
  \url{http://en.wikipedia.org/wiki/Dutch_national_flag_problem}.}:

\begin{displayalgorithm}
  \TitleOfAlgo{Dutch National Flag ordering of an array}
  \Input{An array of elements, and a `pivot' value}
  \Output{The input array with elements ordered as red-white-blue,
    where red elements are larger than the pivot, white elements are
    equal to the pivot, and blue elements are less than the pivot}
\end{displayalgorithm}

We state without proof that this can be done in $O(n)$ operations.
With this, quicksort becomes:

\begin{displayalgorithm}
  \TitleOfAlgo{Quicksort}
  \Input{An array of elements}
  \Output{The input array, sorted}
  \While{The array is longer than one element}{
    pick an arbitrary value as pivot \;
    apply the Dutch National Flag reordering to this array \;
    Quicksort( the blue elements ) \; Quicksort( the red elements ) \;
  }
\end{displayalgorithm}

The indeterminacy of this algorithm, and the variance in its
complexity, stems from the pivot choice. In the worst case, the pivot
is always the (unique) smallest element of the array. There will then
be no blue elements, the only white element is the pivot, and the
recursive call will be on the array of $n-1$ red elements. It is easy
to see that the running time will then be~$O(n^2)$. On the other hand,
if the pivot is always (close to) the median, that is, the element
that is intermediate in size, then the recursive calls will have an
about equal running time, and we get a recursive formula for the
running time:
\[ T_n = 2T_{n/2} + O(n) \]
which  is (again without proof) $O(n\log n)$.

We will now consider parallel implementations of quicksort.

\Level 0 {Quicksort in shared memory}

A simple parallelization of the quicksort algorithm can be achieved by
executing the two recursive calls in parallel. This is easiest
realized with a shared memory model, and threads
(section~\ref{sec:threads}) for the recursive calls. However, this
implementation is not efficient. 

On an array of length~$n$, and with perfect pivot choice, there will
be $n$~threads active in the final stage of the algorithm. Optimally,
we would want a parallel algorithm to run in $O(\log n)$ time, but
here the time is dominated by the initial reordering of the array by
the first thread.

\begin{exercise}
  Make this argument precise. What is the total running time, the
  speedup, and the efficiency of parallelizing the quicksort algorithm
  this way?
\end{exercise}

Since shared memory is not the most interesting case, we will forego
trying to make the thread implementation more efficient, and we will
move on straight away to distributed memory parallelization.

\Level 0 {Quicksort on a hypercube}

As was apparent from the previous section, for an efficient
parallelization of the quicksort algorithm, we need to make the Dutch
National Flag reordering parallel too. Let us then assume that the
array has been partitioned over the $p$ processors of a hypercube of
dimension~$d$ (meaning that $p=2^d$).

In the first step of the parallel algorithm, we choose a pivot, and
broadcast it to all processors. All processors will then apply the
reordering independently on their local data. 

In order to bring together the red and blue elements in this first
level, every processor is now paired up with one that has a binary
address that is the same in every bit but the most significant one. In
each pair, the blue elements are sent to the processor that has a
1~value in that bit; the red elements go to the processor that has a
0~value in that bit.

After this exchange (which is local, and therefore fully parallel),
the processors with an address $1xxxxx$ have all the red elements, and
the processors with an address $0xxxxx$ have all the blue
elements. The previous steps can now be repeated on the subcubes.

This algorithm keeps all processors working in every step; however, it
is susceptible to load imbalance if the chosen pivots are far from the
median. Moreover, this load imbalance is not lessened during the sort
process.

\Level 0 {Quicksort on a general parallel processor}

Quicksort can also be done on any parallel machine that has a linear
ordering of the processors. We assume at first that every processor
holds exactly one array element, and, because of the flag reordering,
sorting will always involve a consecutive set of processors.

Parallel quicksort of an array (or subarray in a recursive call)
starts by constructing a binary tree on the processors storing the
array. A~pivot value is chosen and broadcast through the tree. The
tree structure is then used to count on each processor how many
elements in the left and right subtree are less than, equal to, or
more than the pivot value. 

With this information, the root processor can compute where the
red/white/blue regions are going to be stored. This information is
sent down the tree, and every subtree computes the target locations
for the elements in its subtree.

If we ignore network contention, the reordering can now be done in
unit time, since each processor sends at most one element. This means
that each stage only takes time in summing the number of blue and red
elements in the subtrees, which is $O(\log n)$ on the top level,
$O(\log n/2)$~on the next, et cetera. This makes for almost perfect
speedup.

\index{quicksort|)}
\index{sorting|)}

