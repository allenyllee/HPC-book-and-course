\acfp{LBM}\footnote{This chapter owes much to the presentation
  in~\cite{ChenDoolen:LBM}.}  offer a different way of computing a
discretized solution to a \ac{PDE}, based on dividing a domain in
cells. Its basic idea is then to wonder `if there is a particle in
this cell, and given the force on this cell, where does he particle
move to?'. Rather than reasoning about actual particles, the \ac{LBM}
considers the probability that there is a particle in a cell, and how
that probability is updated.

Distribution function~$f$ (or particle velocity distribution, just
under equation~2?).  For each cell~$i$
\[ f_i(\mathbf{x}+\mathbf{e}_i\Delta x,t+\Delta t) 
    = f_i(\mathbf{x},t)+\Omega_i(f(\mathbf{x},t)).
\]
where $\Omega_i$ is the collision operator giving
the rate of change of the distribution function~$f_i$.

Density: $\rho=\sum_i f_i$,
momentum density: $\rho\mathbf{u}=\sum_if_i\mathbf{e}_i$.

Conservation of mass $\sum_i\Omega_i=0$
conservation of momentum $\sum_i\Omega_i\mathbf{e}_i=0$


Basic equation, second order in~$\epsilon$:
\[
  \frac{\partial f_i}{\partial t} + \mathbf{e}_i\cdot \nabla f_i +
  \epsilon\left(
    \frac12 \mathbf{e}_e\mathbf{e}_i\colon \nabla\nabla f_i +
    \mathbf{e}_i\cdot\frac{\partial f_i}{\partial t} +
    \frac12 \frac{\partial^2f_i}{\partial t^2}
  \right) = \frac{\Omega_i}{\epsilon}
\]
