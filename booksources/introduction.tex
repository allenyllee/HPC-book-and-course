%%%%%%%%%%%%%%%%%%%%%%%%%%%%%%%%%%%%%%%%%%%%%%%%%%%%%%%%%%%%%%%%
%%%%%%%%%%%%%%%%%%%%%%%%%%%%%%%%%%%%%%%%%%%%%%%%%%%%%%%%%%%%%%%%
%%%%
%%%% This text file is part of the source of 
%%%% `Introduction to High-Performance Scientific Computing'
%%%% by Victor Eijkhout, copyright 2012
%%%%
%%%% This book is distributed under a Creative Commons Attribution 3.0
%%%% Unported (CC BY 3.0) license and made possible by funding from
%%%% The Saylor Foundation \url{http://www.saylor.org}.
%%%%
%%%%
%%%%%%%%%%%%%%%%%%%%%%%%%%%%%%%%%%%%%%%%%%%%%%%%%%%%%%%%%%%%%%%%
%%%%%%%%%%%%%%%%%%%%%%%%%%%%%%%%%%%%%%%%%%%%%%%%%%%%%%%%%%%%%%%%

\thispagestyle{empty}
\section*{Preface}

The field of high performance scientific computing lies at the
crossroads of a number of disciplines and skill sets, and
correspondingly, for someone to be successful at using high performance
computing in science requires at least elementary knowledge of and
skills in all these areas.  Computations stem from an application
context, so some acquaintance with physics and engineering sciences is
desirable. Then, problems in these application areas are typically
translated into linear algebraic, and sometimes combinatorial,
problems, so a computational scientist needs knowledge of several
aspects of numerical analysis, linear algebra, and discrete
mathematics. An efficient implementation of the practical formulations
of the application problems requires some understanding of computer
architecture, both on the CPU level and on the level of parallel
computing. Finally, in addition to mastering all these sciences, a
computational scientist needs some specific skills of software
management.

%The authors of this book felt that, 
While good texts exist on numerical
modeling, numerical linear algebra, computer architecture, parallel
computing, performance optimization, no book brings together these
strands in a unified manner. The need for a book such as the present
became apparent to the author working at a computing center:
users are domain experts who not necessarily have mastery
of all the background that would make them
efficient computational scientists. This book, then, teaches those
topics that seem indispensible for scientists
engaging in large-scale computations.

The contents of this book are a combination of theoretical material and
self-guided tutorials on various practical skills. The theory chapters
have exercises that can be assigned in a classroom, however, their
placement in the text is such that a reader not inclined to do
exercises can simply take them as statement of fact.

The tutorials
should be done while sitting at a computer. Given the practice of
scientific computing, they have a clear Unix bias.

\begin{notlulu}
\paragraph*{\bf Public draft}

This book is 
open for comments.
What is missing or incomplete or unclear? Is material
presented in the wrong sequence? Kindly mail me with any comments you
may have.
\end{notlulu}

\begin{download}
You may have found this book in any of a number of places; the
authoritative download location is 
\url{http://www.tacc.utexas.edu/~eijkhout/istc/istc.html}.
It is also possible to get a nicely printed copy from lulu.com:\\
\url{http://www.lulu.com/shop/victor-eijkhout/introduction-to-high-performance-scientific-computing/paperback/product-20452679.html}
\end{download}
\begin{lulu}
Thank you for buying a printed copy of this book. You can also download a pdf 
version, as well as lecture slides from
\url{http://www.tacc.utexas.edu/~eijkhout/istc/istc.html}.
\end{lulu}

\bigskip
\noindent
Victor Eijkhout {\tt eijkhout@tacc.utexas.edu}\\
Research Scientist\\
Texas Advanced Computing Center\\
The University of Texas at Austin

\paragraph*{\bf Acknowledgement}

Helpful discussions with Kazushige Goto and John McCalpin are gratefully
acknowledged. Thanks to Dan Stanzione for his notes on cloud computing,
Ernie Chan for his notes on scheduling of block algorithms, and John
McCalpin for his analysis of the top500.
Thanks to Elie de Brauwer, Susan Lindsey, and Lorenzo Pesce
for proofreading and many comments.

\begin{lulu}
\paragraph*{\bf About the cover}

Have you wondered about the cover of this book? It can stand for any number of things; just take your pick. Maybe the pattern of the cobblestones stands for a processor grid, maybe it stands for a block matrix. Maybe the cobblestones themselves refer to the author's Dutch background. And maybe it symbolizes that the book teaches many topics that normally fall between the cracks\ldots
\end{lulu}

\pagebreak
\thispagestyle{empty}
\section*{Introduction}

Scientific computing is the cross-disciplinary field at the
intersection of modeling scientific processes, and the use of
computers to produce quantitative results from these models.
It is what takes a domain science and turns it into a computational
activity.
%
As a definition, we may posit
\begin{quote}
  The efficient computation of constructive methods in applied mathematics.
\end{quote}
This clearly indicates the three branches of science that scientific
computing touches on:
\begin{itemize}
\item Applied mathematics: the mathematical modeling of real-world
  phenomena. Such modeling often leads to implicit descriptions, for
  instance in the form of partial differential equations. In order to
  obtain actual tangible results we need a constructive approach.
\item Numerical analysis provides algorithmic thinking about
  scientific models. It offers a constructive approach to solving the
  implicit models, with an analysis of cost and stability.
\item Computing takes numerical algorithms and analyzes the efficacy
  of implementing them on actually existing, rather than hypothetical,
  computing engines.
\end{itemize}

One might say that `computing' became a scientific field in its own
right, when the mathematics of real-world phenomena was asked to be
constructive, that is, to go from proving the existence of solutions
to actually obtaining them. At this point, algorithms become an object
of study themselves, rather than a mere tool.

The study of algorithms became especially important when computers were
invented. Since mathematical operations now were endowed with a
definable time cost, complexity of algoriths became a field of study;
since computing was no longer performed in `real' numbers but in
representations in finite bitstrings, the accuracy of algorithms
needed to be studied. Some of these considerations in fact predate the
existence of computers, having been inspired by computing with
mechanical calculators.

A prime concern in scientific computing is efficiency. While to some
scientists the abstract fact of the existence of a solution is enough,
in computing we actually want that solution, and preferably
yesterday. For this reason, in this book we will be quite specific about the
efficiency of both algorithms and hardware. It is important not to
limit the concept of efficiency to that of efficient use of
hardware. While this is important, the difference between two
algorithmic approaches can make optimization for specific hardware a
secondary concern.

This book aims to cover the basics of this gamut of knowledge that a
successful computational scientist needs to master. It is set up as a
textbook for graduate students or advanced undergraduate students;
others can use it as a reference text, reading the exercises for their
information content.

