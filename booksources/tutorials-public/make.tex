{applications,approximation,discrete,namd,performance,programming}.texThe \indextermstart{Make} utility helps you manage the building of
projects: its main task is to facilitate rebuilding only those parts 
of a multi-file project that need to be recompiled or rebuilt.
This can save lots of time, since it
can replace a minutes-long full installation by a single file
compilation. \emph{Make} can also help maintaining multiple
installations of a program on a single machine, for instance compiling
a library with more than one compiler, or compiling a program in debug
and optimized mode.

\emph{Make} is a Unix utility with a long history, and traditionally
there are variants with slightly different behaviour, 
for instance on the various
flavours of Unix such as HP-UX, AUX, IRIX. 
These days, it is advisable, no
matter the platform, to use the GNU version of Make which has some
very powerful extensions; it is available on all Unix platforms
(on Linux it is the only available variant), and it is a {\it de
  facto} standard. The manual is available at
\url{http://www.gnu.org/software/make/manual/make.html}, or you can
read the book~\cite{OReilly-GnuMake}.

There are other build systems, most notably Scons and Bjam. We will
not discuss those here. The examples in this tutorial will be for the
C and Fortran languages, but \emph{Make} can work with any language,
and in fact with things like \TeX\ that are not really a language at
all; see section~\ref{sec:latex-make}.

\Level 0 {A simple example}

\begin{purpose}
In this section you will see a simple example, just to give the flavour of
\emph{Make}.
\end{purpose}

\Level 1 {C}

Make the following files:

\listing{foo.c}{tutorials-public/makefiles/1c/foo.c}
\listing{bar.c}{tutorials-public/makefiles/1c/bar.c}
\listing{bar.h}{tutorials-public/makefiles/1c/bar.h}
and a makefile:
\listing{Makefile}{tutorials-public/makefiles/1c/Makefile}

The makefile has a number of rules like
\begin{verbatim}
foo.o : foo.c
<TAB>cc -c foo.c
\end{verbatim}
which have the general form
\begin{verbatim}
target : prerequisite(s)
<TAB>rule(s)
\end{verbatim}
where the rule lines are indented by a \n{TAB} character.

A rule, such as above, states that a `target' file \n{foo.o} is made
from a `prerequisite' \n{foo.c}, namely by executing the command \n{cc
  -c foo.c}. The precise definition of the rule is:
\begin{itemize}
\item if the target \n{foo.o} does not exist or is older than the
  prerequisite \n{foo.c},
\item then the command part of the rule is executed: \n{cc -c foo.c}
\item If the prerequisite is itself the target of another rule, than that
  rule is executed first.
\end{itemize}
Probably the best way to interpret a rule is:
\begin{itemize}
\item if any prerequisite has changed,
\item then the target needs to be remade,
\item and that is done by executing the commands of the rule.
\end{itemize}

If you call \n{make} without any arguments,
the first rule in the makefile is evaluated. You can execute other
rules by explicitly invoking them, for instance \n{make foo.o} to
compile a single file.

\practical{Call \n{make}.}
  {The above rules are applied: \n{make} without arguments tries to
    build the first target, \n{fooprog}. In order to build this, it
    needs the prerequisites \n{foo.o} and \n{bar.o}, which do not
    exist. However, there are rules for making them, which \n{make}
    recursively invokes. Hence you see two compilations, for \n{foo.o}
    and \n{bar.o}, and a link command for \n{fooprog}.}
  {Typos in the makefile or in file names can cause various
    errors. In particular, make sure you use tabs and not spaces for
    the rule lines. Unfortunately, debugging a makefile is not simple. 
    \emph{Make}'s error message will usually give 
    you the line number in the make file where the error was detected.}

\practical{Do \n{make clean}, followed by \n{mv foo.c boo.c} and
  \n{make} again. Explain the error message. Restore the original file
  name.}
  {\emph{Make} will complain that there is no rule to make
    \n{foo.c}. This error was caused when \n{foo.c} was a
    prerequisite, and was found not to exist. \emph{Make} then went
    looking for a rule to make it.}{}

Now add a second argument to the function \n{bar}. This requires you
to edit \n{bar.c} and \n{bar.h}: go ahead and make these
edits. However, it also requires you to edit \n{foo.c}, but let us for
now `forget' to do that. We will see how \emph{Make} can help you find
the resulting error.

\practical{Call \n{make} to recompile your program. Did it recompile
  \n{foo.c}?}{Even through conceptually \n{foo.c} would need to be
  recompiled since it uses the \n{bar} function,
  \emph{Make} did not do so because the makefile had no
  rule that forced it.}{}

In the makefile, change the line
\begin{verbatim}
foo.o : foo.c
\end{verbatim}
to
\begin{verbatim}
foo.o : foo.c bar.h
\end{verbatim}
which adds \n{bar.h} as a prerequisite for \n{foo.o}. This means that,
in this case where \n{foo.o} already exists, \emph{Make} will check
that \n{foo.o} is not older than any of its prerequisites. Since
\n{bar.h} has been edited, it is younger than \n{foo.o}, so \n{foo.o}
needs to be reconstructed.

\practical{Confirm that the new makefile indeed causes \n{foo.o} to be
  recompiled if \n{bar.h} is changed. This compilation will now give
  an error, since you `forgot' to edit the use of the \n{bar} function.}{}{}

\Level 1 {Fortran}

Make the following files:

\listing{foomain.F}{tutorials-public/makefiles/1f/foomain.F}
\listing{foomod.F}{tutorials-public/makefiles/1f/foomod.F}
and a makefile:
\listing{Makefile}{tutorials-public/makefiles/1f/Makefile}
If you call \n{make}, the first rule in the makefile is executed. Do
this, and explain what happens.

\practical{Call \n{make}.}
  {The above rules are applied: \n{make} without arguments tries to
    build the first target, \n{foomain}. In order to build this, it
    needs the prerequisites \n{foomain.o} and \n{foomod.o}, which do not
    exist. However, there are rules for making them, which \n{make}
    recursively invokes. Hence you see two compilations, for \n{foomain.o}
    and \n{foomod.o}, and a link command for \n{fooprog}.}
  {Typos in the makefile or in file names can cause various
    errors. Unfortunately, debugging a makefile is not simple. You
    will just have to understand the errors, and make the
    corrections.}

\practical{Do \n{make clean}, followed by \n{mv foo.c boo.c} and
  \n{make} again. Explain the error message. Restore the original file
  name.}
  {\emph{Make} will complain that there is no rule to make
    \n{foo.c}. This error was caused when \n{foo.c} was a
    prerequisite, and was found not to exist. \emph{Make} then went
    looking for a rule to make it.}{}

Now add an extra
parameter to \n{func} in \n{foomod.F} and recompile. 

\practical{Call \n{make} to recompile your program. Did it recompile
  \n{foomain.F}?}{Even through conceptually \n{foomain.F} would need to be
  recompiled, \emph{Make} did not do so because the makefile had no
  rule that forced it.}{}

Change the line
\begin{verbatim}
foomain.o : foomain.F
\end{verbatim}
to
\begin{verbatim}
foomain.o : foomain.F foomod.F
\end{verbatim}
which adds \n{foomod.F} as a prerequisite for \n{foomain.o}. This
means that, in this case where \n{foomain.o} already exists,
\emph{Make} will check that \n{foomain.o} is not older than any of its
prerequisites. Since \n{foomod.F} has ben edited, it is younger than
\n{foomain.o}, so \n{foomain.o} needs to be reconstructed.

\practical{Confirm that the corrected makefile indeed causes
  \n{foomain.F} to be recompiled.}{}{}

\Level 0 {Variables and template rules}

\begin{purpose}
  In this section you will learn various work-saving mechanism in
  \emph{Make}, such as the use of variables, and of template rules.
\end{purpose}

\Level 1 {Makefile variables}

It is convenient to introduce variables in your makefile.
For instance, 
instead of spelling out the compiler explicitly every time, introduce a
variable in the makefile:
\begin{verbatim}
CC = gcc
FC = gfortran
\end{verbatim}
and use \verb+${CC}+ or \verb+${FC}+ on the compile lines:
\begin{verbatim}
foo.o : foo.c
        ${CC} -c foo.c
foomain.o : foomain.F
        ${FC} -c foomain.F
\end{verbatim}
\practical{Edit your makefile as indicated. First do \n{make clean},
  then \n{make foo} (C) or \n{make fooprog} (Fortran).}
  {You should see the exact same compile and link lines as before.}
  {Unlike in the shell, where braces are optional, variable names in a
    makefile have to be in
    braces or parentheses. Experiment with what hapens if you forget
    the braces around a variable name.}

One
advantage of using variables is that you can now change the compiler
from the commandline:
\begin{verbatim}
make CC="icc -O2"
make FC="gfortran -g"
\end{verbatim}

\practical{Invoke \emph{Make} as suggested (after \n{make clean}). Do
  you see the difference in your screen output?}
  {The compile lines now show the added compiler option \n{-O2} or \n{-g}.}{}

\emph{Make} also has built-in variables:
\begin{itemize}
\item [\n{\$@}] The target. Use this in the link line for the main
  program. % (See section \ref{sec:target-prereq} for some trickery
  % with~\verb+$@+.)
\item [\n{\$\char`\^}] The list of prerequisites. Use this also in the link
  line for the program.
\item [\n{\$<}] The first prerequisite. Use this in the compile
  commands for the individual object files.
\end{itemize}
Using these variables, the rule for \n{fooprog} becomes
\begin{verbatim}
fooprog : foo.o bar.o
        ${CC} -o $@ $^
\end{verbatim}
and a typical compile line becomes
\begin{verbatim}
foo.o : foo.c bar.h
        ${CC} -c $<
\end{verbatim}

You can also declare a variable
\begin{verbatim}
THEPROGRAM = fooprog
\end{verbatim}
and use this variable instead of the program name in your
makefile. This makes it easier to change your mind about the name of
the executable later. 

\practical{Construct a commandline so that your makefile will build the
  executable \n{fooprog\_v2}.}{You need to specify the \n{THEPROGRAM}
  variable on the commandline using the syntax \n{make VAR=value}.}
  {Make sure that there are no spaces around the equals sign in your
    commandline.}

\Level 1 {Template rules}

In your makefile, the rules for the object files are practically
identical:
\begin{itemize}
\item the rule header (\n{foo.o : foo.c}) states that a source file is a
  prerequisite for the object file with the same base name;
\item and the instructions for compiling (\n{\$\{CC\} -c \$<}) 
  are even character-for-character the
  same, now that you are using \emph{Make}'s built-in variables;
\item the only rule with a difference is 
\begin{verbatim}
foo.o : foo.c bar.h
        ${CC} -c $<
\end{verbatim}
  where the object file depends on the source file and another file.
\end{itemize}
We can take the commonalities and summarize them in one rule\footnote
{This mechanism is the first instance you'll see that only exists in
  GNU make, though in this particular case there is a similar
  mechanism in standard make. That will not be the case for the
  wildcard mechanism in the next section.}:
\begin{verbatim}
%.o : %.c
        ${CC} -c $<
%.o : %.F
        ${FC} -c $<
\end{verbatim}
This states that any object file depends on the C or Fortran file with
the same base name. To regenerate the object file, invoke the C or
Fortran compiler with the \n{-c} flag.
These template rules can function as a replacement for the multiple
specific targets in the makefiles above, except for the rule for \n{foo.o}.

The dependence of \n{foo.o} on \n{bar.h} can be handled by adding a rule
\begin{verbatim}
foo.o : bar.h
\end{verbatim}
with no further instructions. This rule states, `if the prerequisite file
\n{bar.h} changed, file \n{foo.o} needs updating'. \emph{Make} will
then search the makefile for a different
rule that states how this updating is done. 

\practical{Change your makefile to incorporate these ideas, and test.}{}{}

\Level 0 {Wildcards}

Your makefile now uses one general rule for compiling all your source
files. Often, these source files will be all the \n{.c} or \n{.F}
files in your directory, so is there a way to state `compile
everything in this directory'? Indeed there is. Add the following lines
to your makefile, and use the variable \n{COBJECTS} or \n{FOBJECTS}
wherever appropriate.
\begin{verbatim}
# wildcard: find all files that match a pattern
CSOURCES := ${wildcard *.c}
# pattern substitution: replace one pattern string by another
COBJECTS := ${patsubst %.c,%.o,${SRC}}

FSOURCES := ${wildcard *.F}
FOBJECTS := ${patsubst %.F,%.o,${SRC}}
\end{verbatim}

\Level 0 {Miscellania}

\Level 1 {What does this makefile do?}

Above you learned that issuing the \n{make} command will automatically
execute the first rule in the makefile. This is convenient in one
sense\footnote {There is a convention among software developers that a
  package can be installed by the sequence \n{./configure ; make ;
    make install}, meaning: Configure the build process for this
  computer, Do the actual build, Copy files to some system directory
  such as \n{/usr/bin}.}, and inconvenient in another: the only way to
find out what possible actions a makefile allows is to read the
makefile itself, or the --~usually insufficient~-- documentation.

A better idea is to start the makefile with a target
\begin{verbatim}
info :
        @echo "The following are possible:"
        @echo "  make"
        @echo "  make clean"
\end{verbatim}
Now \n{make} without explicit targets informs you of the capabilities
of the makefile. The at-sign at the start of the commandline means `do
not echo this command to the terminal', which makes for cleaner
terminal output; remove the at signs and observe the difference in
behaviour.

\Level 1 {Phony targets}

The example makefile contained a target \n{clean}. This uses
the \emph{Make} mechanisms to accomplish some actions that are not
related to file creation: calling \n{make clean} causes \emph{Make} to
reason `there is no file called \n{clean}, so the following
instructions need to be performed'. However, this does not actually
cause a file \n{clean} to spring into being, so calling \n{make clean}
again will make the same instructions being executed.

To indicate that this rule does not actually make the target, declare
\begin{verbatim}
.PHONY : clean
\end{verbatim}
One benefit of declaring a target to be phony, is that the \emph{Make}
rule will still work, even if you have a file named \n{clean}.

\Level 1 {Predefined variables and rules}

Calling \n{make -p yourtarget} causes make to print out all its
actions, as well as the values of all variables and rules, both in
your makefile and ones that are predefined. If you do this in a
directory where there is no makefile, you'll see that make actually
already knows how to compile \n{.c} or \n{.F} files. Find this rule
and find the definition of the variables in it.

You see that you can customize make by setting such variables as
\n{CFLAGS} or \n{FFLAGS}. Confirm this with some experimentation. If
you want to make a second makefile for the same sources, you can call
\n{make -f othermakefile} to use this instead of the default
\n{Makefile}.

\Level 1 {Using the target as prerequisite}

Suppose you have two different targets that are treated largely the
same. You would want to write:
\begin{verbatim}
PROGS = myfoo other
${PROGS} : $@.o
        ${CC} -o $@ $@.o ${list of libraries goes here}
\end{verbatim}
and saying \n{make myfoo} would cause 
\begin{verbatim}
cc -c myfoo.c
cc -o myfoo myfoo.o ${list of libraries}
\end{verbatim}
and likewise for \n{make other}. What goes wrong here is the use of
\verb+$@.o+ as prerequisite. In Gnu Make, you can repair this as
follows:
\begin{verbatim}
.SECONDEXPANSION:
${PROGS} : $$@.o
\end{verbatim}
%% (Make will now look at lines twice: the first time \verb+$$+ gets
%% converted to a single~\verb+$+, and in the second pass \verb+$@+
%% becomes the name of the target.)

\Level 0 {Shell scripting in a Makefile}

\begin{purpose}
  In this section you will see an example of a longer shell script
  appearing in a makefile rule.
\end{purpose}

In the makefiles you have seen so far, the command part was a single
line. You can actually have as many lines there as you want.
For example, let us make a rule for making backups of the program you
are building.

Add a \n{backup}  rule to your makefile. The first thing it needs to
do is make a backup directory:
\begin{verbatim}
.PHONY : backup
backup :
        if [ ! -d backup ] ; then 
          mkdir backup
        fi
\end{verbatim}
Did you type this? Unfortunately it does not work: every line in the
command part of a makefile rule gets executed as a single
program. Therefore, you need to write the whole command on one line:
\begin{verbatim}
backup :
        if [ ! -d backup ] ; then mkdir backup ; fi
\end{verbatim}
or if the line gets too long:
\begin{verbatim}
backup :
        if [ ! -d backup ] ; then \
          mkdir backup ; \
        fi
\end{verbatim}
Next we do the actual copy:
\begin{verbatim}
backup :
        if [ ! -d backup ] ; then mkdir backup ; fi
        cp myprog backup/myprog
\end{verbatim}
But this backup scheme only saves one version. Let us make a version
that has the date in the name of the saved program. 

The Unix \n{date} command can customize its output by accepting a
format string. Type the following: 
%
\verb+date +%Y%m%d+. 
%
This can be used in the makefile.

\practical {Edit the {\tt cp} command line so that the name of the
  backup file includes the current date.}{Hint: you need the
  backquote. Consult the Unix tutorial if you do not remember what
  backquotes do.}{}

If you are defining shell variables in the command section of a
makefile rule, you need to be aware of the following. Extend your
\n{backup} rule with a loop to copy the object files:
\begin{verbatim}
backup :
        if [ ! -d backup ] ; then mkdir backup ; fi
        cp myprog backup/myprog
        for f in ${OBJS} ; do \
          cp $f backup ; \
        done
\end{verbatim}
(This is not the best way to copy, but we use it for the purpose of
demonstration.) This leads to an error message, caused by the fact
that \emph{Make} interprets \verb+$f+ as an environment variable of
the outer process. What works is:
\begin{verbatim}
backup :
        if [ ! -d backup ] ; then mkdir backup ; fi
        cp myprog backup/myprog
        for f in ${OBJS} ; do \
          cp $$f backup ; \
        done
\end{verbatim}
(In this case \emph{Make} replaces the double dollar by a single one
when it scans the commandline. During the execution of the
commandline, \n{\$f} then expands to the proper filename.)

\Level 0 {A Makefile for \LaTeX}
\label{sec:latex-make}


\indextermend{Make}
