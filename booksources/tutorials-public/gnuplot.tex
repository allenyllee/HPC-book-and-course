{applications,approximation,discrete,namd,performance,programming}.texThe \indexterm{gnuplot}\index{GNU!gnuplot|see{gnuplot}} utility is a
simple program for plotting sets of points or curves. This very short
tutorial will show you some of the basics. For more commands and
options, see the manual
\url{http://www.gnuplot.info/docs/gnuplot.html}.

\Level 0 {Usage modes}

The two modes for running \n{gnuplot} are \emph{interactive} and
\emph{from file}. In interactive mode, you call \n{gnuplot} from the
command line, type commands, and watch output appear; you terminate an
interactive session with \n{quit}.  If you want to save the results of
an interactive session, do \n{save "name.plt"}. This file can be
edited, and loaded with \n{load "name.plt"}.

Plotting non-interactively, you call \n{gnuplot <your file>}.

The output of \n{gnuplot} can be a picture on your screen, or drawing
instructions in a file. Where the output goes depends on the setting
of the \emph{terminal}. By default, \n{gnuplot} will try to draw a
picture. This is equivalent to declaring
\begin{verbatim}
set terminal x11
\end{verbatim}
or \n{aqua}, \n{windows}, or any choice of graphics hardware.

For output to file, declare
\begin{verbatim}
set terminal pdf
\end{verbatim}
or \n{fig}, \n{latex}, \n{pbm}, et cetera. Note that this will only
cause the pdf commands to be written to your screen: you need to
direct them to file with
\begin{verbatim}
  set output "myplot.pdf"
\end{verbatim}
or capture them with
\begin{verbatim}
  gnuplot my.plt > myplot.pdf
\end{verbatim}

\Level 0 {Plotting}

The basic plot commands are \n{plot} for 2D, and \n{splot} (`surface
plot') for 3D plotting.

\Level 1 {Plotting curves}

By specifying
\begin{verbatim}
plot x**2
\end{verbatim}
you get a plot of $f(x)=x^2$; \n{gnuplot} will decide on the range
for~$x$.
With
\begin{verbatim}
set xrange [0:1]
plot 1-x title "down", x**2 title "up"
\end{verbatim}
you get two graphs in one plot, with the $x$~range limited to~$[0,1]$,
and the appropriate legends for the graphs. The variable~\n{x} is the
default for plotting functions.

Plotting one function against another --~or equivalently, plotting a
parametric curve~-- goes like this:
\begin{verbatim}
set parametric
plot [t=0:1.57] cos(t),sin(t)
\end{verbatim}
which gives a quarter circle.

To get more than one graph in a plot, use the command \n{set multiplot}.

\Level 1 {Plotting data points}

It is also possible to plot curves based on data points. The basic
syntax is \n{plot 'datafile'}, which takes two columns from the data
file and interprets them as $(x,y)$ coordinates. Since data files can often have
multiple columns of data, the common syntax is \n{plot 'datafile'
  using 3:6} for columns 3 and~6. Further qualifiers like \n{with
  lines} indicate how points are to be connected.

Similarly, \n{splot "datafile3d.dat" 2:5:7} will interpret three
columns as specifying $(x,y,z)$ coordinates for a 3D plot.

If a data file is to be interpreted as level or height values on a
rectangular grid, do \n{splot "matrix.dat" matrix} for data points;
connect them with
\begin{verbatim}
  split "matrix.dat" matrix with lines
\end{verbatim}

\Level 1 {Customization}

Plots can be customized in many ways. Some of these customizations use
the \n{set} command. For instance,
\begin{verbatim}
  set xlabel "time"
  set ylabel "output"
  set title "Power curve"
\end{verbatim}
You can also change the default drawing style with
\begin{verbatim}
set style function dots
\end{verbatim}
(\n{dots}, \n{lines}, \n{dots}, \n{points}, et cetera), or
change on a single plot with
\begin{verbatim}
plot f(x) with points
\end{verbatim}

\Level 0 {Workflow}

Imagine that your code produces a dataset that you want to plot, and
you run your code for a number of inputs. It would be nice if the
plotting can be automated. Gnuplot itself does not have the facilities
for this, but with a little help from shell programming this is not
hard to do.

Suppose you have data files
\begin{verbatim}
  data1.dat data2.dat data3.dat
\end{verbatim}
and you want to plot them with the same gnuplot commands. You could
make a file \n{plot.template}:
\begin{verbatim}
set term pdf
set output "FILENAME.pdf"
plot "FILENAME.dat"
\end{verbatim}
The string \n{FILENAME} can be replaced by the actual file names
using, for instance \n{sed}:
\begin{verbatim}
for d in data1 data2 data3 ; do
  cat plot.template | sed s/FILENAME/$d/ > plot.cmd
  gnuplot plot.cmd
done
\end{verbatim}
Variations on this basic idea are many.
