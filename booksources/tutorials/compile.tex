% -*- latex -*-
%%%%%%%%%%%%%%%%%%%%%%%%%%%%%%%%%%%%%%%%%%%%%%%%%%%%%%%%%%%%%%%%
%%%%%%%%%%%%%%%%%%%%%%%%%%%%%%%%%%%%%%%%%%%%%%%%%%%%%%%%%%%%%%%%
%%%%
%%%% This text file is part of the source of 
%%%% `Introduction to High-Performance Scientific Computing'
%%%% by Victor Eijkhout, copyright 2012-9
%%%%
%%%% This book is distributed under a Creative Commons Attribution 3.0
%%%% Unported (CC BY 3.0) license and made possible by funding from
%%%% The Saylor Foundation \url{http://www.saylor.org}.
%%%%
%%%%
%%%%%%%%%%%%%%%%%%%%%%%%%%%%%%%%%%%%%%%%%%%%%%%%%%%%%%%%%%%%%%%%
%%%%%%%%%%%%%%%%%%%%%%%%%%%%%%%%%%%%%%%%%%%%%%%%%%%%%%%%%%%%%%%%

\Level 0 {What is executed when you run a program?}

\begin{purpose}
  In this section you will be introduced to the different types of
  files that you encounter while programming.
\end{purpose}

\Level 1 {Source versus program}

There are two types of programming languages:
\begin{enumerate}
\item In an \indextermsubdef{interpreted}{language} you write
  human-readable source code and you execute it directly: the computer
  translates your source line by line as it encounters it. 
\item In a \indextermsubdef{compiled}{language} your code whole source
  is first compiled to a program, which you then execute.
\end{enumerate}
Examples of interpreted languages are \indexterm{Python},
\indexterm{Matlab}, \indexterm{Basic}, \indexterm{Lisp}.
Interpreted languages have some advantages: often you can write them
in an interactive environment that allows you to test code very
quickly. They also allow you to construct code dynamically, during
runtime. However, all this flexibility comes at a price: if a source
line is executed twice, it is translated twice. In the context of this
book, then, we will focus on compiled languages, using \indexterm{C}
and \indexterm{Fortran} as prototypical examples.

So now you have
a distinction between the readable source code, and the
unreadable, but executable, program code. In this tutorial you will
learn about the translation process from the one to the other. The
program doing this translation is known as a \indextermdef{compiler}.
This tutorial will be a `user manual' for compilers, as it were; what
goes on inside a compiler is a different branch of computer science.

\Level 1 {Binary files}

Programs can get very large, so  you don't want to recompile
everything for every change you make: typically you divide your source
in multiple files, and compile them separately.

First of all, a source file can be compiled to an \emph{object file}, which is
  a bit like a piece of an executable: by itself it does nothing, but
  it can be combined with other object files to form an executable.

If you have a collection of object files that you need for more than
one program, it is usually a good idea to make a 
\emph{library}: a bundle of object files that can be used to
  form an executable. Often, libraries are written by an expert and
  contain code for specialized purposes such as linear algebra
  manipulations. Libraries are important enough that they can be
  commercial, to be bought if you need expert code for a certain purpose.

You will now learn how these types of files are created and used.

\Level 0 {Simple compilation}

\begin{purpose}
  In this section you will learn about executables and object files.
\end{purpose}

\Level 1 {Compilers}

Your main tool for turning source into a program is the
\indexterm{compiler}. Compilers are specifically to a language: you
use a different compiler for C than for Fortran.
You can also have two compilers for the same language, but from
different `vendors'. For instance, while many people use the open
source \indexterm{gcc} or \indexterm{clang} compiler families,
companies like \emph{Intel}\index{Intel!compiler} and
\emph{IBM}\index{IBM!compiler} offer compilers that may give more
efficient code on their processors.

\Level 1 {Compile a single file}

Let's start with a simple program that has the whole source in one
file.
\srclisting{One file: \n{hello.c}}{tutorials/compile/c/hello.c}
Compile this program with your favourite compiler; we will use \n{gcc}
in this tutorial, but substitute your own as desired. As a result of
the compilation, a file \n{a.out} is created, which is the executable.
\begin{verbatim}
%% gcc hello.c
%% ./a.out
hello world
\end{verbatim}
You can get a more sensible program name with the \n{-o} option:
\begin{verbatim}
%% gcc -o helloprog hello.c
%% ./helloprog
hello world
\end{verbatim}

\Level 1 {Multiple files: compile and link}

Now we move on to a program that is in more than one source file.

\begin{multicols}{2}
  Main program: \n{fooprog.c}
  \lstinputlisting{code/compile/c/fooprog.c}
  \columnbreak
  Subprogram: \n{fooprog.c}
  \lstinputlisting{code/compile/c/foosub.c}
\end{multicols}

As before, you can make the program with one command.

\snippetoutput{compile/c}{makeoneprogram}
\begin{comment}
\begin{verbatim}
clang -o oneprogram fooprog.c foosub.c
./oneprogram
hello world
\end{verbatim}
\end{comment}

However, you can also do it in steps, compiling each file separately
and then linking them together.

\snippetoutput{compile/c}{makeseparatecompile}

\begin{comment}
\begin{verbatim}
clang -c -fPIC fooprog.c
clang -c -fPIC foosub.c
clang -o oneprogram fooprog.o foosub.o
./oneprogram
hello world
\end{verbatim}
\end{comment}

The \n{-c} option tells the compiler to compile the source file,
giving an \indexterm{object file}. The third command than acts as the
\indexterm{linker}, tieing together the object files into an
executable. (With programs that are spread over several files there is
always the danger of editing a subroutine definition and then
forgetting to update all the places it is used. See the `make'
tutorial, section~\ref{tut:gnumake}, for a way of dealing with this.) 

Each object file defines some routine names, and uses some others that
are undefined in it, but that will be defined in other object files or
system libraries. Use the \indextermunixdef{nm} command to display
this:
\begin{verbatim}
[c:264] nm foosub.o
0000000000000000 T _bar
                 U _printf
\end{verbatim}
Lines with \n{T} indicate routines that are defined; lines with \n{U}
indicate routines that are used but not define in this file. In this
case, \n{printf} is a system routine that will be supplied in the
linker stage.

\Level 0 {Libraries}
\index{libraries!creating and using|(}

\begin{purpose}
  In this section you will learn about libraries.
\end{purpose}

If you have written some subprograms, and you want to share them with
other people (perhaps by selling them), then handing over individual
object files is inconvenient. Instead, the solution is to combine them
into a library. First we look at \indexterm{static libraries}, for
which the \indexterm{archive utility} \n{ar} is used. A~static library
is linked into your executable, becoming part of it. This may lead to
large executables; you will learn about shared libraries
next, which do not suffer from this problem.

Create a directory to contain your library (depending on what your
library is for this can be a
system directory such as \n{/usr/bin}), and create the library file
there. 

\snippetoutput{compile/c}{makestaticlib}

\begin{comment}
\begin{verbatim}
%% mkdir ../lib
%% ar cr ../lib/libfoo.a foosub.o
\end{verbatim}
\end{comment}

The \indextermunix{nm} command tells you what's in the library, just
like it did with object files, but now it also tells you what object
files are in the library:
\begin{verbatim}
%% nm ../lib/libfoo.a 

../lib/libfoo.a(foosub.o):
00000000 T _bar
         U _printf
\end{verbatim}

The library can be linked into your executable by explicitly giving
its name, or by specifying a library path:

\snippetoutput{compile/c}{staticprogram}

\begin{comment}
\begin{verbatim}
%% gcc -o foo fooprog.o ../lib/libfoo.a
# or
%% gcc -o foo fooprog.o -L../lib -lfoo
%% ./foo
hello world
\end{verbatim}
\end{comment}

Although they are somewhat more complicated to use, 
\indexterm{shared libraries}
have several advantages. For
instance, since they are not linked into the executable but only
loaded at runtime, they lead to (much) smaller executables. They are
not created with \n{ar}, but through the compiler. For instance:

\snippetoutput{compile/c}{makedynamiclib}

You can again use \indextermunix{nm}:
\begin{verbatim}
%% nm ../lib/libfoo.so 

../lib/libfoo.so(single module):
00000fc4 t __dyld_func_lookup
00000000 t __mh_dylib_header
00000fd2 T _bar
         U _printf
00001000 d dyld__mach_header
00000fb0 t dyld_stub_binding_helper
\end{verbatim}

Shared libraries are not actually linked into the executable;
instead, the executable needs the information where the library
is to be found at execution time. One way to do this is with
\indextermunixdefus{LD_LIBRARY_PATH}:

\snippetoutput{compile/c}{dynamicprogram}

Another solution is to have the path be included in the executable:
\begin{verbatim}
%% gcc -o foo fooprog.o -L../lib -Wl,-rpath,`pwd`/../lib -lfoo
%% ./foo
hello world
\end{verbatim}
The link line now contains the library path twice:
\begin{enumerate}
\item once with the \n{-L} directive so that the linker can resolve
  all references, and
\item once with the linker directive \verb+-Wl,-rpath,`pwd`/../lib+ which
  stores the path into the executable so that it can be found at runtime.
\end{enumerate}

Use the command \indextermunixdef{ldd} to get information about what shared libraries
your executable uses. (On Mac OS~X, use \n{otool -L} instead.)

\index{libraries!creating and using|)}

