% -*- latex -*-
%%%%%%%%%%%%%%%%%%%%%%%%%%%%%%%%%%%%%%%%%%%%%%%%%%%%%%%%%%%%%%%%
%%%%%%%%%%%%%%%%%%%%%%%%%%%%%%%%%%%%%%%%%%%%%%%%%%%%%%%%%%%%%%%%
%%%%
%%%% This text file is part of the source of 
%%%% `Introduction to High-Performance Scientific Computing'
%%%% by Victor Eijkhout, copyright 2012/3/4/5/
%%%%
%%%% This book is distributed under a Creative Commons Attribution 3.0
%%%% Unported (CC BY 3.0) license and made possible by funding from
%%%% The Saylor Foundation \url{http://www.saylor.org}.
%%%%
%%%%
%%%%%%%%%%%%%%%%%%%%%%%%%%%%%%%%%%%%%%%%%%%%%%%%%%%%%%%%%%%%%%%%
%%%%%%%%%%%%%%%%%%%%%%%%%%%%%%%%%%%%%%%%%%%%%%%%%%%%%%%%%%%%%%%%

Let $y(t)$ be a function defined on $(-\infty,\infty)$. If this
function is built up out of sine and cosine functions
\[ y(t) = \sum_{k=-\infty}^\infty \alpha_k \sin 2\pi kt
                + \beta_k \cos 2\pi kt
\] 
then we can recover the $\alpha_k,\beta_k$ coefficients from taking
the inner product with sine and cosine functions:
\[ (\sin 2\pi nt,\sin 2\pi mt)=0 \quad \hbox{if $n\not=m$} \]
where the inner product is defined as
\[ (f,g) = \int_{\-infty}^\infty f(t)g(t)dt \]
This inspires us to make a transform function~$Y(\omega)$ that
describes the frequency content in the function~$y(t)$:
\[ Y(\omega) = \frac 1{\sqrt{2\pi}}
    \int_{-\infty}^\infty e^{-i\omega T} y(t) dt
\]
If we limit $y(t)$ to a measurement interval $(0,T)$ (or we assume
that the function is periodic on this interval) and we use a discrete
interval $h=1/N$, giving measurement point $t_k=kh$ and frequencies
$\omega_n = 2\pi n/Nh$,
we can approximate
\[ 
\begin{array}{r@{=}l}
  Y(\omega) 
  & \frac 1{\sqrt{2\pi}} \int_{-\infty}^\infty e^{-i\omega t} y(t) dt\\
  & \frac 1{\sqrt{2\pi}} \int_0^T e^{-i\omega t} y(t) dt\\
  & \frac 1{\sqrt{2\pi}} \sum_{k=1}^N  e^{-i\omega_nt_k} y(t_k) h\\
  & \frac 1{\sqrt{2\pi}} \sum_{k=1}^N  e^{-2\pi ikn/N} y(t_k) h\\
\end{array}
\]
Dividing out $h$ we define
\[ Y_n = Y(\omega_n)/h = \frac1{\sqrt{2\pi}} 
        \sum_{k=1}^N y_k e^{-2\pi ikn/N}
\]
and the inverse transform
\[ y(t) = \frac1{\sqrt{2\pi}} \int_{-\infty}^\infty e^{i\omega t}
    \approx \sum_{n=1}^N \frac{2\pi}{Nh} \frac{e^{i\omega_n
        t}}{\sqrt{2\pi}}
    Y(\omega_n).
\]
With a complex variable $Z=e^{-2\pi i/N}$ this becomes
\[ y_k = \frac{\sqrt{2\pi}}N \sum_{n=1}^N Z^{-nk} Y_n, \quad
    Y_n = \frac1{\sqrt{2\pi}} \sum_{k=1}^N Z^{nk} y_k.
\]

To motivate the \acf{FFT} algorithm let us consider a concrete example
with $N=8$. With $Z_n=e^{-2\pi n/8}$ we have that $Z_n=Z_{n+8k}$ and
$Z_{n+4}=-Z_n$, so we only have to compute 4 $Z_n$ values. The
computation
\[ 
\begin{array}{l}
  Y_0 = Z_0y_0+Z_0y_1+\cdots\\
  Y_1 = Z_0y_0+Z_1y_1+Z_2y_2+\cdots\\
  Y_2 = Z_0y_0+Z_2y_1+Z_4y_2+Z_6y_3+\cdots\\
  Y_3 = Z_0y_0+Z_3y_1+Z_6y_3+\cdots\\
  \cdots\\
  y_7 = Z_0y_1+Z_7y_1+Z_{14}y_2+\cdots
\end{array}
\]
reduces to 
\[
\begin{array}{l}
  Y_0=Z_0y_0+Z_0y_1+Z_0y_2+Z_0y_3+Z_0y_4+Z_0y_5+Z_0y_6+Z_0y_7 \\
  Y_1=Z_0y_0+Z_1y_1+Z_2y_2+Z_3y_3-Z_0y_4-Z_1y_5-Z_2y_6-Z_3y_7 \\
  Y_2=Z_0y_0+Z_2y_2-Z_0Y-2-Z_2y_3+Z_0y_4+Z_2y_5-Z_0y_6-Z_0y_7 \\
  Y_3=Z_0y_0+Z_3y_1-Z_2y_2+Z_1y_3-Z_0y_4-Z_3y_5+Z_2y_6-Z_1y_7 \\
  Y_4=Z_0y_0-Z_0y_1+Z_0y_2-Z_0y_3+Z_0y_4-Z_0y_5+Z_0y_6-Z_0y_7 \\
  Y_5=Z_0y_0-Z_1y_1+Z_2y_2-Z_3y_3-Z_0y_4+Z_1y_5-Z_2y_6+Z_3y_7 \\
  Y_6=Z_0y_0-Z_2y_2-Z_0Y_2+Z_2y_3+Z_0y_4-Z_2y_5-Z_0y_6+Z_0y_7 \\
  Y_7=Z_0y_0-Z_3y_1-Z_2y_2-Z_1y_3-Z_0y_4+Z_3y_5+Z_2y_6+Z_1y_7 \\
\end{array}
\]
There is a large number of common subexpressions and `almost
subexpressions'. For instance, note that $Y_0,Y_1$ contain $Z_0y_0\pm
Z_0y_4$. Next, $Y_0,Y_2$ and $Y_4,Y_6$ contain $(Z_0y_0\pm Z_2y_2) \pm
(Z_0y_4\pm Z_2y_6)$. Et cetera. Thus, the \ac{FFT} algorithm can be
built out of an elementary `butterfly'
\[
  \left. \begin{array}{r}x\\y\end{array} \right\} 
    \Rightarrow
  \left\{ \begin{array}{r}x+y\\x-y\end{array} \right.
\]

Let $\omega_n$ be the $n$-th root of unity: \[ \omega_n=e^{-2\pi i/n} \]
which has the property that $\omega_{n/2}=\omega_n^2$.

The Fourier sum
\[ Y_k=\sum_{j=0}^{n-1} \omega_n^{jk} y_j \]
If $n$ is even and $k\leq n/2-1$:
\[
\begin{array}{r@{{}={}}l}
Y_k
&\sum_{\mathrm{even}\,j} \omega_n^{jk} y_j + \sum_{\mathrm{odd}\,j} \omega_n^{jk} y_j \\
&\sum_{j=0}^{n/2-1} \omega_n^{2jk} y_{2j} + \omega_n \sum_{j=0}^{n/2-1} \omega_n^{2jk} y_{2j+1} \\
&\sum_{j=0}^{n/2-1} \omega_{n/2}^{jk} y_{2j} + \omega_n \sum_{j=0}^{n/2-1} \omega_{n/2}^{jk} y_{2j+1}
\end{array}
\]
so the length-$n$ Fourier transform consists of two length-$n/2$ transforms and a single multiplication
by~$\omega_n$.

\endinput

Define 
\[
\begin{cases}
  G_N=e^{\frac{ik}N  2\pi}=\omega_N^k&\omega_N=e^{\frac iN2\pi}\\
  G_{2N}=e^{\frac{ik}{2N} 2\pi}=\omega_{2N}^k&\omega_{2N}=e^{\frac iN\pi}\\
\end{cases}
\]
For $f\in C(G_{2N})$,
define $F_a,F_b\in C(G_N)$ by
\[
    F_a(\omega^r_N) = f(\omega^{2r}_{2N}),\qquad
    F_b(\omega^r_N) = f(\omega^{2r+1}_{2N}),
\]
Also
\[
\begin{cases}
  E_j(\omega_N^r) = \omega_N^{jr} & E_{j+N}=E_j\\
  e_k(\omega_{2N}^s) = \omega_{2N}^{sk}\in C(G_{2N})
\end{cases}
\]
\begin{enumerate}
\item Computing $\omega_{2N}^s$ for $0\leq s<2N$ can be done in $2N$
  multiplications.
\item Assuming that the $N$ values of $F_a$ and $F_b$ can be computed
  in $M$ operations, then the coefficients of $F_a,F_b$ can be
  computed in $2M$.
\end{enumerate}
$O(N)$ time, then:
\[
\begin{array}{r@{=}l}
  \hat f(e_k) = \hat f(\omega^k_{2N})
    &(2N)\inv \sum_{r=0}^{N-1} f(\omega^r_{2N})e_k(\omega_{2N}^r)\\
    &(2N)\inv \sum_{r=0}^{N-1} f(\omega^r_{2N})\omega^{rk}_{2N}\\
    &1/2\left[
      N\inv\sum_{r=0}^{N-1}f(\omega^{2r}_{2N})\omega^{2rk}_{2N}+
      N\inv\sum_{r=0}^{N-1}f(\omega^{2r+1}_{2N})\omega^{(2r+1)k}_{2N}
      \right]\\
    &1/2\left[ \hat F_a(\omega^k_N) + \omega^k_{2N}\hat
      F_b(\omega^k_N) \right]\\    
\end{array}
\]
so the $2N$ coefficients of~$f(\omega^k_{2N})$ can also be computed
in~$O(N)$. Inductively, computing all top level coefficients can be
computed in $O(N\log N)$ time.


