  Take another look at figure~\ref{fig:par-sum-graph} of a parallel
  reduction. The basic actions are:
  \begin{itemize}
  \item receive data from a neighbour
  \item add it to your own data
  \item send the result on.
  \end{itemize}
  As you see in the diagram, there is at least one processor who does
  not send data on, and others may do a variable number of receives
  before they send their result on.

  Write node code so that an \ac{SPMD} program realizes the
  distributed reduction. Hint: write each processor number in
  binary. The algorithm uses a number of steps that is equal to the
  length of this bitstring.
  \begin{itemize}
  \item Assuming that a processor receives a message, express the
    distance to the origin of that message in the step number.
  \item Every processor sends at most one message. Express the step
    where this happens in terms of the binary processor number.
  \end{itemize}
