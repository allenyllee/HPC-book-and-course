  With the limited connections of a linear array, you may have to be
  clever about how to program parallel algorithms. For instance,
  consider a `broadcast' operation: processor~$0$ has a data item that
  needs to be sent to every other processor.

  We make the following simplifying assumptions:
  \begin{itemize}
  \item a processor can send any number of messages simultaneously,
  \item but a wire can can carry only one message at a time; however,
    \item communication between any two processors takes unit time,
      regardless the number of processors in between them.
  \end{itemize}

  In a fully connected network or a star network
  you can simply write
  \begin{tabbing}
    for \=$i=1\ldots N-1$:\\ \>send the message to processor~$i$
  \end{tabbing}
  With the assumption that a processor can send multiple messages,
  this means that the operation is done in one step.

  Now consider a linear array. Show that, even with this unlimited capacity for
  sending, the above algorithm runs into trouble because of congestion.

  Find a better way to organize the send operations. Hint: pretend
  that your processors are connected as a binary tree. Assume that
  there are $N=2^n-1$ processors.
  Show that the broadcast can be done in $\log N$ stages, and that
  processors only need to be able to send a single message simultaneously.
