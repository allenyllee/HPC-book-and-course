  Consider the \ac{CG} method of section~\ref{sec:cg},
  figure~\ref{fig:pcg}, applied to the
  matrix from a 2D \ac{BVP}; equation~\eqref{eq:5starmatrix}, First
  consider the unpreconditioned case $M=I$. Show that there is a
  roughly equal number of floating point
  operations are performed in the matrix-vector product and
  in the vector operations. Express everything in the matrix size~$N$ and
  ignore lower order terms. How would this balance be if the matrix
  had 20 nonzeros per row?

  Next, investigate this balance between vector and matrix operations
  for the \ac{FOM} scheme in section~\ref{sec:fom}. Since the number
  of vector operations depends on the iteration, consider the first 50
  iterations and count how many floating point operations are done in
  the vector updates and inner product versus the matrix-vector
  product. How many nonzeros does the matrix need to have for these
  quantities to be equal?
