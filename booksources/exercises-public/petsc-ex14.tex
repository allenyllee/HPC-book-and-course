Write a PETSc program that does the following:
\begin{itemize}
\item Construct the matrix
  \[
  A=
  \begin{pmatrix}
    2&-1\\ -1&2&-1\\ &\ddots&\ddots&\ddots\\ &&-1&2&-1\\ &&&-1&2\\
  \end{pmatrix}
  \]
\item Compute the sequence
  \[
  x_0=(1,0,\ldots,0)^t,\quad y_{i+1}=Ax_i,\quad x_i=y_i/\|y_i\|_2.
  \]
  This is the power method (section~\ref{app:power-method}), which is
  expected to converge to the dominent eigenvector.
\item In each iteration of this process, print out the norm of~$y_i$
  and for $i>0$ the norm of the difference~$x_i-x_{i-1}$. Do this for
  some different problem sizes. What do you observe?
\item The number of iterations and the size of the problem should be
  specified through commandline options. Use the routine
  \n{PetscOptionsGetInt}.
\end{itemize}
For a small problem (say, $n=10$) print out the first couple $x_i$
vectors. What do you observe? Explanation?
