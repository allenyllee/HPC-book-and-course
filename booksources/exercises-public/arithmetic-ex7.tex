  Every programmer, at some point in their life, makes the mistake of
  storing a real number in an integer or the other way around. This
  can happen for instance if you call a function differently from how
  it was defined.
\begin{verbatim}
void a(float x) {....}
int main() {
  int i;
  .... a(i) ....
}
\end{verbatim}
What happens when you print \n{x} in the function? Consider the bit
pattern for a small integer, and use the table in
figure~\ref{fig:single-precision} to interpret it as a floating point
number. Explain that it will be an unnormalized number\footnote{This
  is one of those errors you won't forget after you make it. In the
  future, whenever you see a number on the order of $10^{-305}$ you'll
  recognize that you probably made this error.}.
