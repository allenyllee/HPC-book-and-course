  A condition of the type $u(0)=u_0$ is called a \indexterm{Dirichlet
    boundary condition}. Physically, this corresponds to knowing the
  temperature at the end point of a rod. Other boundary conditions
  exist. Specifying a value for the derivative, $u'(0)=u'_0$, rather
  than for the function value,would be appropriate if we are modeling
  fluid flow and the outflow rate at $x=0$ is known. This is known as
  a \indexterm{Neumann boundary condition}.

  A Neumann boundary condition $u'(0)=u'_0$ can be modeled by stating
  \[ \frac{u_0-u_1}h=u'_0. \]
  Show that, unlike in the case of the Direchlet boundary condition,
  this affects the matrix of the linear system.

  Show that having a
  Neumann boundary condition at both ends gives a singular
  matrix, and therefore no unique solution to the linear system.
  (Hint: guess the vector that has eigenvalue zero.)

  Physically this makes sense. For instance, in an elasticity problem,
  Dirichlet boundary conditions state that the rod is clamped at a
  certain height; a Neumann boundary condition only states its angle
  at the end points, which leaves its height undetermined.
