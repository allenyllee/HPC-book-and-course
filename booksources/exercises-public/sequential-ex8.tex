  Write a small cache simulator in your favourite language. Assume a
  $k$-way associative cache of 32 entries and an architecture with 16
  bit addresses. Run the following
  experiment for $k=1,2,4,\ldots$:
  \begin{enumerate}
  \item Let $k$ be the associativity of the simulated cache.
  \item Write the translation from 16 bit memory addresses to $32/k$
    cache addresses.
  \item\label{step:random} Generate 32 random machine addresses, and
    simulate storing them in cache.
  \end{enumerate}
  Since the cache has 32 entries, optimally the 32 addresses can all
  be stored in cache. The chance of this actually happening is small,
  and often the data of one address will be evicted from the cache
  (meaning that it is overwritten) when another address conflicts with
  it. Record how many addresses, out of~32, are actually stored in the
  cache at the end of the simulation. Do step~\ref{step:random} 100
  times, and plot the results; give median and average value, and the
  standard deviation. Observe that increasing the associativity
  improves the number of addresses stored. What is the limit
  behaviour? (For bonus points, do a formal statistical analysis.)
