  The Jacobi iteration for the linear system $Ax=b$ is defined as
\[ x_{i+1}=x_i-K^{-1}(Ax_i-b) \]
where $K$ is the diagonal of~$A$. Show that you can transform the
linear system (that is, find a different coefficient matrix and right
hand side vector that will still have the same solution) so that you
can compute the same $x_i$ vectors but with $K=I$, the identity
matrix.

What are the implications of this strategy, in terms of storage and
operation counts? Are there special implications if $A$ is a sparse
matrix?

Suppose $A$ is symmetric. Give a simple example to show that $K^{-1}A$
does not have to be symmetric. Can you come up with a different
transformation of the system so that symmetry of the coefficient
matrix is preserved and that
has the same advantages as the transformation above? You can
assume that the matrix has positive diagonal elements.
