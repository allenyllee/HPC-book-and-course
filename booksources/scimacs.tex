%%%%%%%%%%%%%%%%%%%%%%%%%%%%%%%%%%%%%%%%%%%%%%%%%%%%%%%%%%%%%%%%
%%%%%%%%%%%%%%%%%%%%%%%%%%%%%%%%%%%%%%%%%%%%%%%%%%%%%%%%%%%%%%%%
%%%%
%%%% This text file is part of the source of 
%%%% `Introduction to High-Performance Scientific Computing'
%%%% by Victor Eijkhout, copyright 2012-8
%%%%
%%%% This book is distributed under a Creative Commons Attribution 3.0
%%%% Unported (CC BY 3.0) license and made possible by funding from
%%%% The Saylor Foundation \url{http://www.saylor.org}.
%%%%
%%%%
%%%%%%%%%%%%%%%%%%%%%%%%%%%%%%%%%%%%%%%%%%%%%%%%%%%%%%%%%%%%%%%%
%%%%%%%%%%%%%%%%%%%%%%%%%%%%%%%%%%%%%%%%%%%%%%%%%%%%%%%%%%%%%%%%

\newcommand\furtherreading{\Level 0 {Further Reading}\label{sec:furtherreading-\chapshortname}}

{\catcode`\^^I=13 \globaldefs=1
\newcommand\listing[2]{\begingroup\small\par\vspace{1ex}
  \catcode`\^^I=13 \def^^I{\leavevmode\hspace{40pt}}
  \noindent\fbox{#1}
  \verbatiminput{#2}\endgroup}
  \newcommand\codelisting[1]{\begingroup\small\par\vspace{1ex}
  \catcode`\^^I=13 \def^^I{\leavevmode\hspace{40pt}}
  \noindent\fbox{#1}
  \verbatiminput{#1}\endgroup}
}
\newcommand\inv{^{-1}}\newcommand\invt{^{-t}}
\newcommand\setspan[1]{[\![#1]\!]}
\newcommand\fp[2]{#1\cdot10^{#2}}
\newcommand\fxp[2]{\langle #1,#2\rangle}
\def\n#{\bgroup \catcode`\_=12 \catcode`\>=12 \catcode`\<=12
  \catcode`\&=12 \catcode`\^=12 \catcode`\~=12 \def\\{\char`\\}\relax
  \tt \let\next=}

\newcommand\diag{\mathop{\mathrm {diag}}}
\newcommand\argmin{\mathop{\mathrm {argmin}}}
\newcommand\defined{
  \mathrel{\lower 5pt \hbox{${\equiv\atop\mathrm{\scriptstyle D}}$}}}
\newcommand\lulubreak{\message{Hard page break!}\pagebreak\relax}

\newcommand\bbP{\mathbb{P}}
\newcommand\bbR{\mathbb{R}}

\newtheorem{remark}{Remark}
\expandafter\ifx\csname definition\endcsname\relax
    \newtheorem{definition}{Definition}
\fi
\expandafter\ifx\csname theorem\endcsname\relax
    \newtheorem{theorem}{Theorem}
\fi
\expandafter\ifx\csname lemma\endcsname\relax
    \newtheorem{lemma}{Lemma}
\fi
\expandafter\ifx\csname proof\endcsname\relax
 \newenvironment{proof}{\begin{quotation}\small\sl\noindent Proof.\ \ignorespaces}
     {\end{quotation}}
\fi
%% \newenvironment{highermath}
%%     {\bigskip\begin{quotation}\noindent\emph{MMM}}
%%     {\end{quotation}\bigskip\noindent\ignorespaces}

\usepackage{acronym}
\newwrite\acrowrite
\openout\acrowrite=acronyms.tex
\def\acroitem#1#2{\acrodef{#1}{#2}
    \write\acrowrite{\message{defining #1}\noexpand\acitem{#1}{#2}}
}
\acroitem{AMR}{Adaptive Mesh Refinement}
\acroitem{AOS}{Array-Of-Structures}
\acroitem{API}{Application Programmer Interface}
\acroitem{AVX}{Advanced Vector Extensions}
\acroitem{BEM}{Boundary Element Method}
\acroitem{BFS}{Breadth-First Search}
\acroitem{BLAS}{Basic Linear Algebra Subprograms}
\acroitem{BSP}{Bulk Synchronous Parallel}
\acroitem{BVP}{Boundary Value Problem}
\acroitem{CAF}{Co-array Fortran}
\acroitem{CCS}{Compressed Column Storage}
\acroitem{CG}{Conjugate Gradients}
\acroitem{CGS}{Classical Gram-Schmidt}
\acroitem{COO}{Coordinate Storage}
\acroitem{CPU}{Central Processing Unit}
\acroitem{CRS}{Compressed Row Storage}
%\acroitem{CSV}{comma-separated values}
\acroitem{DAG}{Directed Acyclic Graph}
\acroitem{DRAM}{Dynamic Random-Access Memory}
\acroitem{DSP}{Digital Signal Processing}
\acroitem{FD}{Finite Difference}
\acroitem{FMA}{Fused Multiply-Add}
\acroitem{FDM}{Finite Difference Method}
\acroitem{FEM}{Finite Element Method}
\acroitem{FMM}{Fast Multipole Method}
\acroitem{FOM}{Full Orthogonalization Method}
\acroitem{FPU}{Floating Point Unit}
\acroitem{FFT}{Fast Fourier Transform}
\acroitem{FSA}{Finite State Automaton}
\acroitem{FSB}{Front-Side Bus}
\acroitem{FPGA}{Field-Programmable Gate Array}
\acroitem{GMRES}{Generalized Minimum Residual}
\acroitem{GPU}{Graphics Processing Unit}
\acroitem{GPGPU}{General Purpose Graphics Processing Unit}
\acroitem{GS}{Gram-Schmidt}
\acroitem{HDFS}{Hadoop File System}
\acroitem{HPC}{High-Performance Computing}
\acroitem{HPF}{High Performance Fortran}
\acroitem{IBVP}{Initial Boundary Value Problem}
\acroitem{IDE}{Integrated Development Environment}
\acroitem{ILP}{Instruction Level Parallelism}
\acroitem{ILU}{Incomplete LU}
\acroitem{IVP}{Initial Value Problem}
\acroitem{LAN}{Local Area Network}
\acroitem{LBM}{Lattice Boltzmann Method}
\acroitem{LRU}{Least Recently Used}
\acroitem{MIC}{Many Integrated Cores}
\acroitem{MIMD}{Multiple Instruction Multiple Data}
\acroitem{MGS}{Modified Gram-Schmidt}
\acroitem{MPI}{Message Passing Interface}
\acroitem{MSI}{Modified-Shared-Invalid}
\acroitem{MTA}{Multi-Threaded Architecture}
\acroitem{NUMA}{Non-Uniform Memory Access}
\acroitem{ODE}{Ordinary Diffential Equation}
\acroitem{OS}{Operating System}
\acroitem{PGAS}{Partitioned Global Address Space}
\acroitem{PDE}{Partial Diffential Equation}
\acroitem{PRAM}{Parallel Random Access Machine}
\acroitem{RDMA}{Remote Direct Memory Access}
\acroitem{SAN}{Storage Area Network}
\acroitem{SAS}{Software As a Service}
\acroitem{SFC}{Space-Filling Curve}
\acroitem{SIMD}{Single Instruction Multiple Data}
\acroitem{SIMT}{Single Instruction Multiple Thread}
\acroitem{SM}{Streaming Multiprocessor}
\acroitem{SMP}{Symmetric Multi Processing}
\acroitem{SMT}{Symmetric Multi Threading}
\acroitem{SOA}{Structure-Of-Arrays}
\acroitem{SOR}{Successive Over-Relaxation}
\acroitem{SP}{Streaming Processor}
\acroitem{SPMD}{Single Program Multiple Data}
\acroitem{SPD}{symmetric positive definite}
\acroitem{SRAM}{Static Random-Access Memory}
\acroitem{SSE}{SIMD Streaming Extensions}
\acroitem{TLB}{Translation Look-aside Buffer}
\acroitem{UMA}{Uniform Memory Access}
\acroitem{UPC}{Unified Parallel C}
\acroitem{WAN}{Wide Area Network}
\acresetall
\closeout\acrowrite

\newcommand{\indexterm}[1]{\emph{#1}\index{#1}}
\newcommand{\indextermttdef}[1]{{\ttfamily\slshape #1}\index{#1@\texttt{#1}|textbf}}
\newcommand{\indextermunix}[1]{{\ttfamily\slshape #1}\index{#1@\texttt{#1}}}
\newcommand{\indextermdef}[1]{\emph{#1}\index{#1|textbf}}
\newcommand{\indextermp}[1]{\emph{#1s}\index{#1}}
\newcommand{\indextermsub}[2]{\emph{#1 #2}\index{#2!#1}}
\newcommand{\indextermsubdef}[2]{\emph{#1 #2}\index{#2!#1|textbf}}
\newcommand{\indextermsubp}[2]{\emph{#1 #2s}\index{#2!#1}}
\newcommand{\indextermbus}[2]{\emph{#1 #2}\index{#1!#2}}
\newcommand{\indextermbusdef}[2]{\emph{#1 #2}\index{#1!#2|textbf}}
\newcommand{\indextermstart}[1]{\emph{#1}\index{#1|(}}
\newcommand{\indextermend}[1]{\index{#1|)}}
\newcommand{\indexstart}[1]{\index{#1|(}}
\newcommand{\indexend}[1]{\index{#1|)}}

{ \catcode`\_=13
  %% MPI index routines
%\newcommand{\indextermtt}[1]{{\ttfamily\slshape #1}\index{#1@\texttt{#1}}}
\gdef\indextermtt#{\bgroup \catcode`\_=13 \def_{\char95\discretionary{}{}{}}
  \catcode`\>=12 \catcode`\<=12
  \catcode`\&=12 \catcode`\^=12 \catcode`\~=12 \def\\{\char`\\}\relax
  \tt \afterassignment\tttoindex\edef\indexedtt}
}
\def\tttoindex{%\tracingmacros=2
  \edef\tmp{\noexpand\n{\indexedtt}%
            \noexpand\index[mpi]{\indexedtt@{\noexpand\tt{\indexedtt}}}}%
  \tmp
  \egroup
}

\makeatletter
\newcommand\indexac[1]{\emph{\ac{#1}}%
  %\tracingmacros=2 \tracingcommands=2
  \edef\tmp{\noexpand\index{%
    \expandafter\expandafter\expandafter
        \@secondoftwo\csname fn@#1\endcsname%
    @\acl{#1} (#1)}}\tmp}
\newcommand\indexacdef[1]{\emph{\ac{#1}}%
  %\tracingmacros=2 \tracingcommands=2
  \edef\tmp{\noexpand\index{%
    \expandafter\expandafter\expandafter
        \@secondoftwo\csname fn@#1\endcsname%
    @\acl{#1} (#1)}|textbf}\tmp}
\newcommand\indexacp[1]{\emph{\ac{#1}}%
  %\tracingmacros=2 \tracingcommands=2
  \edef\tmp{\noexpand\index{%
    \expandafter\expandafter\expandafter
        \@secondoftwo\csname fn@#1\endcsname%
    @\aclp{#1} (#1)}}\tmp}
\newcommand\indexacf[1]{\emph{\acf{#1}}%
  \edef\tmp{\noexpand\index{%
    \expandafter\expandafter\expandafter
        \@secondoftwo\csname fn@#1\endcsname
    @\acl{#1} (#1)}}\tmp}
\newcommand\indexacstart[1]{%
  \edef\tmp{\noexpand\index{%
    \expandafter\expandafter\expandafter
        \@secondoftwo\csname fn@#1\endcsname
    @\acl{#1} (#1)|(}}\tmp}
\newcommand\indexacend[1]{%
  \edef\tmp{\noexpand\index{%
    \expandafter\expandafter\expandafter
        \@secondoftwo\csname fn@#1\endcsname
    @\acl{#1} (#1)|)}}\tmp}
\makeatother

%%%%%%%%%%%%%%%%%%%%%%%%%%%%%%%%%%%%%%%%%%%%%%%%%%%%%%%%%%%%%%%%
%%%%
%%%% Outlining
%%%%
%%%%%%%%%%%%%%%%%%%%%%%%%%%%%%%%%%%%%%%%%%%%%%%%%%%%%%%%%%%%%%%%
\usepackage{outliner}
\OutlineLevelStart0{\chapter{#1}
  \maillink{arabic{chapter} "#1"}{#1}{chapter}
}
\OutlineLevelStart1{\section{#1}
\maillink{\arabic{chapter}.\arabic{section}}{#1}{section}
}
\OutlineLevelCont1{\section{#1}
\maillink{\arabic{chapter}.\arabic{section}}{#1}{section}
}
\OutlineLevelStart2{\subsection{#1}
  \maillink
    {\arabic{chapter}.\arabic{section}.\arabic{subsection}}{#1}{subsection}
}
\OutlineLevelStart3{\subsubsection{#1}}
\setcounter{secnumdepth}{4}
\OutlineLevelStart4{\paragraph{\bf #1}}

\newcommand\heading[1]{\paragraph*{\textbf{#1}}}

%%%%%%%%%%%%%%%%
%%%% Titles
%%%%%%%%%%%%%%%%
\def\chaptertitle{\csname\chaptername title\endcsname}
\def\chaptershorttitle{\csname\chaptername shorttitle\endcsname}

\def\sequentialtitle{Processor Architecture}
\def\sequentialshorttitle{CPU}

\def\paralleltitle{Parallel Computer Architecture}
\def\parallelshorttitle{Parallelism}

\def\arithmetictitle{Computer Arithmetic}
\def\arithmeticshorttitle{Arithmetic}

\def\odepdetitle{Ordinary and Partial Differential Equations}
\def\odedeshorttitle{PDEs}

\def\lineartitle{Numerical Linear Algebra}
\def\linearshorttitle{Algebra}

\def\parallellineartitle{High Performance Linear Algebra}
\def\parallellinearshorttitle{HPC}

\def\mdtitle{Molecular Dynamics}
\def\mdshorttitle{MD}

\def\combinatoricstitle{Combinatorics}
\def\combinatoricsshorttitle{Combinatorics}

%% tutorials
\def\unixtitle{Introduction to Unix}
\def\unixshorttitle{unix}

\def\latextitle{Introduction to \LaTeX}
\def\latexshorttitle{latex}

\def\gnumaketitle{Managing software projects with Make}
\def\gnumakeshorttitle{gnumake}

\def\gnuplottitle{Plotting data with GnuPlot}
\def\gnuplotshorttitle{gnuplot}

\def\svntitle{Source code control with Subversion}
\def\svnshorttitle{svn}

%% appendices
\def\petsctitle{The Portable Extendable Toolkit for Scientific
  Computing (PETSc)}
\def\petscshorttitle{petsc}
