%%%%%%%%%%%%%%%%%%%%%%%%%%%%%%%%%%%%%%%%%%%%%%%%%%%%%%%%%%%%%%%%
%%%%%%%%%%%%%%%%%%%%%%%%%%%%%%%%%%%%%%%%%%%%%%%%%%%%%%%%%%%%%%%%
%%%%
%%%% This text file is part of the source of 
%%%% `Introduction to High-Performance Scientific Computing'
%%%% by Victor Eijkhout, copyright 2012-7
%%%%
%%%% This book is distributed under a Creative Commons Attribution 3.0
%%%% Unported (CC BY 3.0) license and made possible by funding from
%%%% The Saylor Foundation \url{http://www.saylor.org}.
%%%%
%%%%
%%%%%%%%%%%%%%%%%%%%%%%%%%%%%%%%%%%%%%%%%%%%%%%%%%%%%%%%%%%%%%%%
%%%%%%%%%%%%%%%%%%%%%%%%%%%%%%%%%%%%%%%%%%%%%%%%%%%%%%%%%%%%%%%%

\thispagestyle{empty}
\section*{Preface}

The field of high performance scientific computing
requires, in addition to a broad of scientific knowledge and 'coputing
folkore', a number of practical skills. Call it the `carpentry' aspect
of the craft of scientific computing.

As a companion to the book `Introduction to High Performance
Scientific Computing', which covers background knowledge, here is then
a set of tutorials on those practical skills that are important to
becoming a successful high performance practitioner.

The tutorials
should be done while sitting at a computer. Given the practice of
scientific computing, they have a clear Unix bias.

\begin{notlulu}
\paragraph*{\bf Public draft}

This book is 
open for comments.
What is missing or incomplete or unclear? Is material
presented in the wrong sequence? Kindly mail me with any comments you
may have.
\end{notlulu}

\begin{download}
You may have found this book in any of a number of places; the
authoritative download location is 
\url{http://www.tacc.utexas.edu/~eijkhout/istc/istc.html}.
That page also links to \n{lulu.com} where you can get a nicely printed copy.
\end{download}
\begin{lulu}
Thank you for buying a printed copy of this book. 
You can also download a pdf 
version, as well as lecture slides and complete source code from
\url{http://www.tacc.utexas.edu/~eijkhout/istc/istc.html}.
\end{lulu}

\bigskip
\noindent
Victor Eijkhout {\tt eijkhout@tacc.utexas.edu}\\
Research Scientist\\
Texas Advanced Computing Center\\
The University of Texas at Austin

\paragraph*{\bf Acknowledgement}

Helpful discussions with Kazushige Goto and John McCalpin are gratefully
acknowledged. Thanks to Dan Stanzione for his notes on cloud computing,
Ernie Chan for his notes on scheduling of block algorithms, and John
McCalpin for his analysis of the top500.
Thanks to Elie de Brauwer, Susan Lindsey, Tim Haines, and Lorenzo Pesce
for proofreading and many comments.
Edmond Chow wrote the chapter on Molecular Dynamics. Robert van de Geijn
contributed several sections on dense linear algebra.

