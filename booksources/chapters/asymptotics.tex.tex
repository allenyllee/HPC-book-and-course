%%%%%%%%%%%%%%%%%%%%%%%%%%%%%%%%%%%%%%%%%%%%%%%%%%%%%%%%%%%%%%%%
%%%%%%%%%%%%%%%%%%%%%%%%%%%%%%%%%%%%%%%%%%%%%%%%%%%%%%%%%%%%%%%%
%%%%
%%%% This text file is part of the source of 
%%%% `Introduction to High-Performance Scientific Computing'
%%%% by Victor Eijkhout, copyright 2012
%%%%
%%%% This book is distributed under a Creative Commons Attribution 3.0
%%%% Unported (CC BY 3.0) license and made possible by funding from
%%%% The Saylor Foundation \url{http://www.saylor.org}.
%%%%
%%%%
%%%%%%%%%%%%%%%%%%%%%%%%%%%%%%%%%%%%%%%%%%%%%%%%%%%%%%%%%%%%%%%%
%%%%%%%%%%%%%%%%%%%%%%%%%%%%%%%%%%%%%%%%%%%%%%%%%%%%%%%%%%%%%%%%

If we ignore limitations such as that the number of processors has to
be finite, or the physicalities of the  interconnect between them, we can
derive theoretical results on the limits of parallel computing. This
section will give a brief introduction to such results, and discuss
their connection to real life high performance computing.

Consider for instance the matrix-matrix multiplication
$C=AB$, which takes $2N^3$ operations where $N$ is the matrix
size. Since there are no dependencies between the operations for the
elements of~$C$, we can perform them all in parallel. If we had $N^2$
processors, we could assign each to an $(i,j)$ coordinate in~$C$, and
have it compute $c_{ij}$ in $2N$ time. Thus, this parallel operation
has efficiency~$1$, which is optimal.

\begin{exercise}
  Show that this algorithm ignores some serious issues about memory
  usage:
  \begin{itemize}
  \item If the matrix is kept in shared memory, how many simultaneous
    reads from each memory locations are performed?
  \item If the processors keep the input and output to the local
    computations in  local storage, how much duplication
    is there of the matrix elements?
  \end{itemize}
\end{exercise}

Adding $N$ numbers $\{x_i\}_{i=1\ldots N}$ can be performed in $\log_2
N$ time with $N/2$ processors. As a simple example, consider the sum
of $n$ numbers: $s=\sum_{i=1}^n a_i$. If we have $n/2$ processors we
could compute:
\begin{enumerate}
\item Define $s^{(0)}_i = a_i$.
\item Iterate with $j=1,\ldots,\log_2 n$:
\item Compute $n/2^j$ partial sums $s^{(j)}_i=s^{(j-1)}_{2i}+s^{(j-1)}_{2i+1}$
\end{enumerate}
We see that the $n/2$ processors perform a total of $n$ operations (as
they should) in $\log_2n$ time. The efficiency of this parallel scheme
is~$O(1/\log_2n)$, a slowly decreasing function of~$n$.  
\begin{exercise}
  Show that, with the scheme for parallel addition just outlined, you
  can multiply two matrices in $\log_2 N$ time with $N^3/2$
  processors. What is the resulting efficiency?
\end{exercise}

It is now a legitimate theoretical question to ask
\begin{itemize}
\item If we had infinitely many processors, what is the lowest
  possible time complexity for matrix-matrix multiplication, or
\item Are there faster algorithms that still have $O(1)$ efficiency?
\end{itemize}
Such questions have been researched (see for
instance~\cite{He:surveyparallel}), but they have little bearing on
high performance computing. 

A first objection to these kinds of theoretical bounds is that they
implicitly assume some form of shared memory. In fact, the formal
model for these algorithms is called a \indexac{PRAM}, where the
assumption is that every memory location is accessible to any
processor.  Often an additional assumption is made that multiple
access to the same location are in fact possible\footnote{This notion
  can be made precise; for instance, one talks of a CREW-PRAM, for
  Concurrent Read, Exclusive Write PRAM.}. These assumptions are
unrealistic in practice, especially in the context of scaling up the
problem size and the number of processors. A~further objection to the
\ac{PRAM} model is that even on a single processor it ignores the
memory hierarchy; section~\ref{sec:hierarchy}.

But even if we take distributed memory into account, theoretical
results can still be unrealistic. The above summation algorithm can
indeed work unchanged in distributed memory, except that we have to
worry about the distance between active processors increasing as we
iterate further. If the processors are connected by a linear array,
the number of `hops' between active processors doubles, and with that,
asymptotically, the computation time of the iteration. The total
execution time then becomes~$n/2$, a disappointing result given that
we throw so many processors at the problem. 

What if the processors are
connected with a hypercube topology? It is not hard to see that the
summation algorithm can then indeed work in $\log_2n$ time. However,
as $n\rightarrow\infty$, can we build a sequence of hypercubes of $n$ nodes
and keep the communication time between two connected constant? Since
communication time depends on latency, which partly depends on the
length of the wires, we have to worry about the physical distance
between nearest neighbours.

The crucial question here is whether the hypercube (an $n$-dimensional
object) can be embedded in 3-dimensional space, while keeping the
distance (measured in meters) constant between connected neighbours.
It is easy to see that a 3-dimensional grid can be scaled up
arbitrarily while maintaining a unit wire length, but the question is
not clear for a hypercube.  There, the length of the wires may have to
increase as $n$ grows, which runs afoul of the finite speed of
electrons.

We sketch a proof (see~\cite{Fisher:fastparallel} for more details)
that, in our three dimensional world and with a finite speed of light,
speedup is limited to $\sqrt[4]{n}$ for a problem on $n$ processors,
no matter the interconnect. The argument goes as follows. Consider an
operation that involves collecting a final result on one processor. Assume
that each processor takes a unit volume of space, produces one result
per unit time, and can send one data item per unit time. Then, in an
amount of time~$t$, at most the processors in a ball with radius~$t$,
that is, $O(t^3)$ processors can contribute to the
final result; all others are too far away. In time~$T$, then, the
number of operations 
that can contribute to the final result is $\int_0^T
t^3dt=O(T^4)$. This means that the maximum achievable speedup is the
fourth root of the sequential time.

Finally, the question `what if we had infinitely many processors' is
not realistic as such, but we will allow it in the sense that we will
ask the \indexterm{weak scaling} question (section~\ref{sec:scaling})
`what if we let the problem size and the number of processors grow
proportional to each other'. This question is legitimate, since it
corresponds to the very practical deliberation whether buying more
processors will allow one to run larger problems, and if so, with what
`bang for the buck'.

