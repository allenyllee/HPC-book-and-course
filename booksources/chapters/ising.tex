%%%%%%%%%%%%%%%%%%%%%%%%%%%%%%%%%%%%%%%%%%%%%%%%%%%%%%%%%%%%%%%%
%%%%%%%%%%%%%%%%%%%%%%%%%%%%%%%%%%%%%%%%%%%%%%%%%%%%%%%%%%%%%%%%
%%%%
%%%% This text file is part of the source of 
%%%% `Introduction to High-Performance Scientific Computing'
%%%% by Victor Eijkhout, copyright 2012
%%%%
%%%% This book is distributed under a Creative Commons Attribution 3.0
%%%% Unported (CC BY 3.0) license and made possible by funding from
%%%% The Saylor Foundation \url{http://www.saylor.org}.
%%%%
%%%%
%%%%%%%%%%%%%%%%%%%%%%%%%%%%%%%%%%%%%%%%%%%%%%%%%%%%%%%%%%%%%%%%
%%%%%%%%%%%%%%%%%%%%%%%%%%%%%%%%%%%%%%%%%%%%%%%%%%%%%%%%%%%%%%%%

The Ising model (for an introduction, see~\cite{Cipra:Ising}) was
originally proposed to model ferromagnetism. Magnetism is the result
of atoms aligning their `spin' direction: let's say spin can only be
`up' or `down', then a material has magnetism if more atoms have spin
up than down, or the other way around. The atoms are said to be in a
structure called a `lattice'.

Now image heating up a material, which loosens up the atoms. If an
external field is applied to the material, the atoms will start
aligning with the field, and if the field is removed the magnetism
disappears again. However, below a certain critical temperature the
material will retain its magnetism.
We will use Monte Carlo simulation to find the stable configurations
that remain.

Let's say the lattice $\Lambda$ has $N$ atoms, and we denote a
configuration of atoms as $\sigma=(\sigma_1,\ldots,\sigma_N)$ where
each $\sigma_i=\pm1$.
The energy of a lattice is modeled as
\[ H=H(\sigma)=-J\sum_i\sigma_i-E\sum_{ij}\sigma_i\sigma_j. \]
The first term models the interaction of individual spins $\sigma_i$
with an external field of strength~$J$. The second term, which sums
over nearest neighbour pairs, models
alignment of atom pairs: the product $\sigma_i\sigma_j$ is positive if
the atoms have identical spin, and negative if opposite.

In \indexterm{statistical mechanics}, the probability of a
configuration is 
\[ P(\sigma) = \exp(-H(\sigma))/Z \]
where the `partitioning function' $Z$ is defined as 
\[ Z = \sum_\sigma \exp(H(\sigma)) \]
where the sum runs over all $2^N$ configurations.

A configuration is stable if its energy does not decrease under small
perturbations. To explore this, we iterate over the lattice, exploring
whether altering the spin of atoms lowers the energy. We introduce an
element of chance to prevent artificial solutions. (This is the
\indexterm{Metropolis algorithm}~\cite{Metropolis}.)

\begin{displayalgorithm}
  \For{fixed number of iterations}{
    \For{each atom $i$}{
      {calculate the change $\Delta E$ from changing the sign of $\sigma_i$}\;
      \If{$\Delta E<$ or $\exp(-\Delta E)$ greater than some random
        number}{
        accept the change}\;
    }
  }
\end{displayalgorithm}

This algorithm can be parallelized, if we notice the similarity with
the structure of the sparse matrix-vector product. In that algorithm too we
compute a local quantity by combining inputs from a few nearest
neighbours. This means we can partitioning the lattice, and compute
the local updates after each processor collects a \indexterm{ghost
  region}.

Having each processor iterate over local points in the lattice
corresponds to a particular global ordering of the lattice; to make
the parallel computation equivalent to a sequential one we also need a
parallel random generator (section~\ref{sec:parallel-random}).

\endinput

Atomic order can be studied theoretically by applying Monte Carlo
simulations to a generalized Ising model. An Ising model describes the
energy of any arrangement of atoms on a lattice.
To illustrate this, consider a triangular two dimensional lattice.

\includegraphics{graphics/ising/ising1}

Assume that two types of atoms (call them A and B) can reside at each
lattice site. At each lattice site i, we associate an occupation
variable s(i) such that

s(i)=+1 if A resides at site i

s(i)=-1 if B resides at site i

It can be shown that any property that depends on the arrangement of A
and B atoms on the lattice (e.g. the energy) can be exactly expanded
in terms of polynomials of occupation variables associated with
different lattice sites according to

\includegraphics{graphics/ising/ising2}

where the polynomial basis functions are simply products of occupation
variables belonging to clusters of sites (e.g. a pair cluster, a
triplet cluster etc.). The expansion goes over all possible clusters
of lattice sites, but in practice the expression is truncated after
some maximal cluster. The coefficients of the polynomial basis
functions are constant and are to be determined with a first
principles electronic structure method.

Monte Carlo simulation

Ordering at different temperatures can be studied by applying Monte
Carlo simulations to the Ising model. In a Monte Carlo simulation,
different arrangements of A and B atoms on the lattice are sampled
according to the probability distribution of statistical mechanics
which is given by

\includegraphics{graphics/ising/ising3}

where:

P(a) is the probability of occurrence of a particular arrangement a

E(a) is the energy of that arrangement as calculated with the Ising model energy expression

T is the temperature

k is the Boltzmann constant

The basic algorithm of a Monte Carlo code is very simple and can be
summarized as follows:

\includegraphics[scale=.6]{graphics/ising/ising4}

Generalized Monte Carlo code

Usually, Monte Carlo programs for Ising model simulations are written
for a specific lattice. The reason for this is that in a typical
simulation, the Ising Model energy expression is evaluated many times
(~1000000 times). In order to evaluate this energy expression quickly,
it is necessary to explicitly write it within the code. This
expression, however, is different for different lattices.

This can be illustrated with the following example:

\includegraphics[scale=.3]{graphics/ising/ising5}

Compare a square lattice with a triangular lattice (see above
figure). Assume that the Ising model expressions only have nearest
neighbor interaction terms (i.e. all other terms corresponding to
larger clusters are truncated). Each site on a square lattice has 4
nearest neighbors while each site on a triangular lattice has 6
nearest neighbors (the nearest neighbors are connected to point 5 with
the bold lines). The Ising Model energy expressions for these lattices
thus look very different, since the connectivity of the sites is
different in different lattices.

To avoid manually writing a new Monte Carlo code for each different
lattice, a preprocessor was created in this work that uses the group
theory of crystallography to automatically write the energy expression
for a particular lattice. This energy expression is then automatically
incorporated into the body of a standard Monte Carlo code. All that is
needed as input is the space group of the lattice (which can be found
in the literature) and the coordinates of a representative cluster of
each type appearing in the Ising Model energy expression. A user of
this preprocessor, is therefore not required to know the details of a
Monte Carlo code.

Parallelization

The Monte Carlo code was parallelized by dividing the lattice into as
many slabs as there are processors. Each processor then performs the
above algorithm locally within each slab. Since the atoms at each
lattice site interact with atoms at distant lattice sites, those atoms
at the edge of each slab will interact with atoms in adjacent
slabs. Therefore, in order to correctly evaluate the Ising model
energy expression, each processor must be made aware of changes made
within a thin layer at the border (call it the "border layer" which
has a thickness equal to the maximum interaction range) of the
adjacent slab. Communication is done with MPI. After each border layer
has been passed by a given processor, the modified occupation
variables are packed into an array and sent to the processor working
in the adjacent slab. This is done before the processor of the
adjacent slab comes within the interaction range of the border layer.

