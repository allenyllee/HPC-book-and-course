%%%%%%%%%%%%%%%%%%%%%%%%%%%%%%%%%%%%%%%%%%%%%%%%%%%%%%%%%%%%%%%%
%%%%%%%%%%%%%%%%%%%%%%%%%%%%%%%%%%%%%%%%%%%%%%%%%%%%%%%%%%%%%%%%
%%%%
%%%% This text file is part of the source of 
%%%% `Introduction to High-Performance Scientific Computing'
%%%% by Victor Eijkhout, copyright 2012
%%%%
%%%% This book is distributed under a Creative Commons Attribution 3.0
%%%% Unported (CC BY 3.0) license and made possible by funding from
%%%% The Saylor Foundation \url{http://www.saylor.org}.
%%%%
%%%%
%%%%%%%%%%%%%%%%%%%%%%%%%%%%%%%%%%%%%%%%%%%%%%%%%%%%%%%%%%%%%%%%
%%%%%%%%%%%%%%%%%%%%%%%%%%%%%%%%%%%%%%%%%%%%%%%%%%%%%%%%%%%%%%%%

\Level 0 {Co-array Fortran}

Extension to Fortran to facilitate parallel programming~\cite{coarray}.

For example, one problem with co-array fortran is that it does not
allow the programmer to specify nonblocking transfer operations. In
MPI (or other explicit message passing environments), the programer
can initiate a transfer of data between processors, and then go off
and do some useful work while the data is being transferred, query
the transfer operation to determine when it is complete, and then
begin operating on that data when it is available. This
functionality is missing in co-array fortran.

A coarray program certainly can start a transfer long before
the data is needed and go do some other unrelated work.
The images synchronize only at a sync statement.
Depending upon your specific application, various combinations
of sync statements, volatile attributes and flush statements
might be best.

For example,

\begin{verbatim}
local_b(:) = remote_b(:)[over_there]
call big_honking_calculation_involving_a( a)
sync memory
! now work with b
\end{verbatim}

Co-arrays part of the Fortran standard;
support in G95~\cite{g95coarray}
