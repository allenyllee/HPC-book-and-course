% -*- latex -*-
%%%%%%%%%%%%%%%%%%%%%%%%%%%%%%%%%%%%%%%%%%%%%%%%%%%%%%%%%%%%%%%%
%%%%%%%%%%%%%%%%%%%%%%%%%%%%%%%%%%%%%%%%%%%%%%%%%%%%%%%%%%%%%%%%
%%%%
%%%% This text file is part of the source of 
%%%% `Introduction to High-Performance Scientific Computing'
%%%% by Victor Eijkhout, copyright 2012-8
%%%%
%%%% This book is distributed under a Creative Commons Attribution 3.0
%%%% Unported (CC BY 3.0) license and made possible by funding from
%%%% The Saylor Foundation \url{http://www.saylor.org}.
%%%%
%%%%
%%%%%%%%%%%%%%%%%%%%%%%%%%%%%%%%%%%%%%%%%%%%%%%%%%%%%%%%%%%%%%%%
%%%%%%%%%%%%%%%%%%%%%%%%%%%%%%%%%%%%%%%%%%%%%%%%%%%%%%%%%%%%%%%%

In chapter~\ref{ch:odepde} you saw how the numerical solution of
partial differential equations can lead to linear algebra
problems. Sometimes this is a simple problem --~a matrix-vector
multiplication in the case of the Euler forward method~-- but
sometimes it is more complicated, such as the solution of a system of
linear equations in the case of Euler backward methods. Solving linear
systems will be the focus of this chapter; in other applications,
which we will not discuss here, eigenvalue problems need to be solved.

You may have learned a simple
algorithm for solving system of linear equations: elimination of
unknowns, also called Gaussian elimination. This method can still be
used, but we need some careful discussion of its efficiency. There are
also other algorithms, the so-called iterative solution methods, which
proceed by gradually approximating the solution of the linear
system. They warrant some discussion of their own.

Because of the PDE background, we only consider linear systems that
are square and nonsingular. Rectangular, in particular overdetermined,
systems have important applications too in a corner of numerical
analysis known as optimization theory. However, we will not cover
that in this book.

The standard work on numerical linear algebra is Golub and Van Loan's
{\it Matrix Computations}~\cite{GoVL:matcomp}.  It covers algorithms,
error analysis, and computational details.  Heath's {\it Scientific
  Computing} covers the most common types of computations that arise
in scientific computing; this book has many excellent exercises and
practical projects.

\Level 0 {Elimination of unknowns}
\label{sec:gauss-example}

In this section we are going to take a closer look at
\emph{Gaussian elimination}%
\index{Gauss!Gaussian elimination},
or elimination of unknowns. You may have seen this method
before (and if not, it will be explained below), but we will be a bit
more systematic here so that we can analyze various aspects of it.

\begin{remark}
  It is also possible to solve the equation $Ax=y$ for $x$ by
  computing the \indextermbus{inverse}{matrix} $A\inv$, for instance
  by executing the \indextermbus{Gauss}{Jordan} algorithm, and multiplying
  $x\leftarrow A\inv x$. The reasons for not doing this are primarily
  of numerical precision, and fall outside the scope of this book.
\end{remark}

One general thread of this chapter will be the discussion of the
efficiency of the various algorithms. When you learned how to solve a
system of unknowns by gradually eliminating unknowns, you most likely
never applied that method to a matrix larger than~$4\times4$. The
linear systems that occur in PDE solving can be thousands of times
larger, and counting how many operations, as well as how much memory,
they require becomes important.

Let us consider an example of the importance of efficiency in choosing
the right algorithm.
The solution of a linear system can be written with a fairly simple
explicit formula, using determinants. This is called
`\indexterm{Cramer's rule}'. It is mathematically elegant, but
completely impractical for our purposes.

If a matrix $A$ and a vector $b$ are given, and a vector~$x$
satisfying $Ax=b$ is wanted, then,
writing $|A|$ for the determinant,
  \[ x_i=\left|
    \begin{matrix}
      a_{11}&a_{12}&\ldots&a_{1i-1}&b_1&a_{1i+1}&\ldots&a_{1n}\\
      a_{21}&      &\ldots&        &b_2&        &\ldots&a_{2n}\\
      \vdots&      &      &        &\vdots&     &      &\vdots\\
      a_{n1}&      &\ldots&        &b_n&        &\ldots&a_{nn}
    \end{matrix}\right|
    / |A|
    \]
For any matrix $M$ the determinant is defined recursively as
\[ |M| = \sum_i (-1)^im_{1i}|M^{[1,i]}| \]
where $M^{[1,i]}$ denotes the matrix obtained by deleting row~1 and
column~$i$ from~$M$. This means that computing the determinant of a
matrix of dimension~$n$ means $n$~times computing a size~$n-1$
determinant. Each of these requires $n-1$ determinants of size~$n-2$, so
you see that the number of operations
required to compute the determinant is factorial in the matrix
size. This quickly becomes prohibitive, even ignoring any issues of
numerical stability.
Later in this chapter you will see complexity estimates for other
methods of solving systems of linear equations that are considerably
more reasonable.

Let us now look at a simple example of solving linear equations
with elimination of unknowns. Consider the system
\begin{equation}
  \begin{array}{rrr@{{}={}}r}
    6x_1&-2x_2&+2x_3&16 \\ 12x_1&-8x_2&+6x_3&26 \\ 3x_1&-13x_2&+3x_3&-19
  \end{array}
  \label{eq:elimination-example}
\end{equation}
We eliminate $x_1$ from the second and third equation by
\begin{itemize}
\item multiplying the first equation $\times 2$ and subtracting the
  result from the second equation, and
\item multiplying the first equation $\times 1/2$ and subtracting the
  result from the third equation.
\end{itemize}
The linear system then becomes
\[
  \begin{array}{rrr@{{}={}}r}
      6x_1&-2x_2&+2x_3&16\\ 0x_1&-4x_2&+2x_3&-6\\ 0x_1&-12x_2&+2x_3&-27
  \end{array}
\]
Finally, we eliminate $x_2$ from the third equation by multiplying
the second equation by~3, and subtracting the result from the third equation:
\[
  \begin{array}{rrr@{{}={}}r}
  6x_1&-2x_2&+2x_3&16\\ 0x_1&-4x_2&+2x_3&-6\\ 0x_1&+0x_2&-4x_3&-9
\end{array}
\]
We can now solve $x_3=9/4$ from the last equation. Substituting that
in the second equation, we get $-4x_2=-6-2x_2=-21/2$ so
$x_2=21/8$. Finally, from the first equation $6x_1=16+2x_2-2x_3=
16+21/4-9/2=76/4$ so $x_1=19/6$.

We can write this more compactly by omitting the $x_i$ coefficients. Write
\[ 
\begin{pmatrix}
  6&-2&2\\ 12&-8&6\\ 3&-13&3
\end{pmatrix}
\begin{pmatrix} x_1\\ x_2\\ x_3 \end{pmatrix}
=
\begin{pmatrix}
  16\\ 26\\ -19
\end{pmatrix}
\]
as
\begin{equation}
\left[
  \begin{matrix}
    6&-2&2&|&16\\ 12&-8&6&|&26\\ 3&-13&3&|&-19      
  \end{matrix}\right]
\label{eq:systemabbrev}
\end{equation}
then the elimination process is
\[
\left[
  \begin{matrix}
    6&-2&2&|&16\\ 12&-8&6&|&26\\ 3&-13&3&|&-19      
  \end{matrix}\right] \longrightarrow
\left[
    \begin{matrix}
      6&-2&2&|&16\\ 0&-4&2&|&-6\\ 0&-12&2&|&-27
    \end{matrix}\right] \longrightarrow
  \left[
    \begin{matrix}
      6&-2&2&|&16\\ 0&-4&2&|&-6\\ 0&0&-4&|&-9
    \end{matrix}\right].
\]

In the above example, the matrix coefficients could have been any real
(or, for that matter, complex) coefficients, and you could follow the
elimination procedure mechanically.
There is  the following exception.
At some point in the computation, we divided by the numbers
$6,-4,-4$ which are found on the diagonal of the matrix in the last
elimination step. These quantities are called the \indexterm{pivots},
and clearly they are required to be nonzero. 

\begin{exercise}
  \label{ex:zero-pivot}
  The system 
  \[
  \begin{array}{rrrr}
    6x_1&-2x_2&+2x_3=&16 \\ 12x_1&-4x_2&+6x_3=&26 \\ 3x_1&-13x_2&+3x_3=&-19
  \end{array}
  \]
  is the same as the one we just investigated in
  equation~\eqref{eq:elimination-example}, except for the $(2,2)$
  element. Confirm that you get a zero pivot in the second step.
\end{exercise}


The first pivot is an element of the original matrix; as you saw in
the preceding exercise, the other pivots can not easily be found
without doing the actual elimination. In particular, there is no easy
way of predicting zero pivots from looking at the system of equations.

If a pivot turns out to be zero, all is not lost for the computation:
we can always exchange two matrix rows; this is known as
\indexterm{pivoting}.  It is not hard to show
\begin{footnoteenv}
  {And you can
    find this in any elementary linear algebra textbook.}
\end{footnoteenv}
that with a
nonsingular matrix there is always a row exchange possible that puts a
nonzero element in the pivot location.

\begin{exercise}
  Suppose you want to exchange matrix rows 2 and 3 of the system of
  equations in equation~\eqref{eq:systemabbrev}. What other
  adjustments would you have to make to make sure you still compute the
  correct solution? 
  Continue the system solution of the previous exercise
  by exchanging rows 2 and~3, and check that you get the correct answer.
\end{exercise}

\begin{exercise}
  Take another look at exercise~\ref{ex:zero-pivot}.  Instead of
  exchanging rows 2 and~3, exchange columns 2 and~3. What does this
  mean in terms of the linear system? Continue the
  process of solving the system; check that you get the same solution
  as before.
\end{exercise}

In general, with floating point numbers and round-off, it is very
unlikely that a matrix element will become exactly zero during a
computation. Also, in a \ac{PDE} context, the diagonal is usually nonzero.
Does that mean that pivoting is in practice almost never
necessary? The answer is \emph{no}: pivoting is desirable from a point
of view of numerical stability. In the next section you will see an
example that illustrates this fact.

\Level 0 {Linear algebra in computer arithmetic}
\label{sec:linear-arith}

In most of this chapter, we will act as if all operations
can be done in exact arithmetic. However, it is good to become aware
of some of the potential problems due to our finite precision computer
arithmetic. This allows us to design algorithms that minimize the effect
of roundoff. A~more rigorous approach to the topic of numerical linear
algebra includes a full-fledged error analysis of the algorithms we
discuss; however, that is beyond the scope of this course.
%
Error analysis of computations in computer arithmetic is the focus of
Wilkinson's classic {\it Rounding errors in Algebraic
  Processes}~\cite{Wilkinson:roundoff} and Higham's more recent {\it
  Accuracy and Stability of Numerical
  Algorithms}~\cite{Higham:2002:ASN}. 

Here, we will only note a few paradigmatic examples of the sort of
problems that can come up in computer arithmetic: we will show why
pivoting during LU factorization is more than a theoretical device,
and we will give two examples of problems in eigenvalue calculations
due to the finite precision of computer arithmetic.

\Level 1 {Roundoff control during elimination}
\index{pivoting|(}

Above, you saw that row interchanging (`pivoting') is necessary if a
zero element appears on the diagonal during elimination of that row
and column. Let us now look at what happens if the pivot element is
not zero, but close to zero.

Consider the linear system
\[ \left(
  \begin{matrix}
    \epsilon&1\\ 1&1
  \end{matrix}\right) x = \left(
  \begin{matrix}
    1+\epsilon\\2
  \end{matrix}\right)
\] 
which has the solution solution $x=(1,1)^t$.
Using the $(1,1)$ element to clear the remainder of the first column gives:
\[ \left(
  \begin{matrix}
    \epsilon&1\\ 0&1-\frac1\epsilon
  \end{matrix}\right) x = \left(
  \begin{matrix}
    1+\epsilon\\ 2-\frac{1+\epsilon}\epsilon
  \end{matrix}\right).
\]
We can now solve $x_2$ and from it~$x_1$.

If $\epsilon$ is small, say $\epsilon<\epsilon_{\mathrm{mach}}$,
the $1+\epsilon$ term in the right hand side will be simply~$1$: our
linear system will be
\[ \left(
  \begin{matrix}
    \epsilon&1\\ 1&1
  \end{matrix}\right) x = \left(
  \begin{matrix}
    1\\2
  \end{matrix}\right)
\] 
but the solution $(1,1)^t$ will still satisfy the system in machine
arithmetic.

Next, $1/\epsilon$ will be very large, so the second component of the
right hand side after elimination will be
$2-\frac1\epsilon=-1/\epsilon$.
Also, the $(2,2)$ element of the matrix is then $-1/\epsilon$ instead of
$1-1/\epsilon$:
\[ 
  \begin{pmatrix}
    \epsilon&1\\ 0&-1/\epsilon
  \end{pmatrix} x = 
  \begin{pmatrix}
    1\\-1/\epsilon
  \end{pmatrix}
\] 
We get the correct value $x_2=1$, but
\[ \epsilon x_1+x_2=1 \Rightarrow \epsilon x_1=0 \Rightarrow x_1=0, \]
which is $100\%$ wrong.

What would have happened if we had pivoted as described above?
We exchange the matrix rows, giving
\[ 
  \begin{pmatrix}
    1&1\\ \epsilon&1
  \end{pmatrix} x =
  \begin{pmatrix}
    2\\1+\epsilon
  \end{pmatrix}
\Rightarrow
  \begin{pmatrix}
    1&1\\ 0&1-\epsilon
  \end{pmatrix} x=
    \begin{pmatrix}
      2\\ 1-\epsilon
    \end{pmatrix}
\]
Now we get, regardless the size of epsilon:
\[ x_2=\frac{1-\epsilon}{1-\epsilon}=1,\qquad
x_1=2-x_2=1
\]
In this example we used a very small value of~$\epsilon$; a much more
refined analysis shows that even with $\epsilon$ greater than the
machine precision pivoting still makes sense. The general rule of
thumb is: \emph{Always do row exchanges to get the largest remaining
  element in the current column into the pivot position.} In
chapter~\ref{ch:odepde} you saw matrices that arise in certain
practical applications; it can be shown that for them pivoting is
never necessary; see exercise~\ref{ex:no-pivot}.

The pivoting that was discussed above is also known as
\indexterm{partial pivoting}\index{pivoting!partial}, since it is
based on row exchanges only.  Another option would be
\indextermsub{full}{pivoting}, where row and column exchanges are
combined to find the largest element in the remaining subblock, to be
used as pivot. Finally, \indextermsub{diagonal}{pivoting} applies the
same exchange to rows and columns. (This is equivalent to renumbering
the unknowns of the problem, a~strategy which we will consider in
section~\ref{sec:ordering} for increasing the parallelism of the problem.)
This means that pivots are only
searched on the diagonal.
From now on we will only consider partial pivoting.

\index{pivoting|)}

\Level 1 {Influence of roundoff on eigenvalue computations}

Consider the matrix 
\[ A=
\begin{pmatrix}
1&\epsilon\\ \epsilon&1  
\end{pmatrix}
\]
where $\epsilon_{\mathrm{mach}}<|\epsilon|<\sqrt{\epsilon_{\mathrm{mach}}}$,
which has eigenvalues $1+\epsilon$ and $1-\epsilon$.
If we calculate its characteristic polynomial in computer arithmetic
\[ \left|
  \begin{matrix}
    1-\lambda&\epsilon\\ \epsilon&1-\lambda
  \end{matrix}\right| = \lambda^2-2\lambda+(1-\epsilon^2)=
  \lambda^2-2\lambda+1.
\]
we find a
double eigenvalue~1. Note that the exact eigenvalues are
expressible in working precision; it is the algorithm that causes
the error. Clearly, using the characteristic polynomial is not the
right way to compute eigenvalues, even in well-behaved, symmetric
positive definite, matrices.

An unsymmetric example: let $A$ be the matrix of size~20
\[ A=
\begin{pmatrix}
  20&20&      &      &\emptyset\\
    &19&20\\
    &  &\ddots&\ddots\\
    &  &      &2     &20\\
  \emptyset&  &      &      &1
\end{pmatrix}.
\]
Since this is a triangular matrix, its eigenvalues are the diagonal
elements. If we perturb this matrix by setting $A_{20,1}=10^{-6}$ we
find a perturbation in the eigenvalues that is much larger than in the
elements:
\[ \lambda=20.6\pm 1.9i, 20.0\pm 3.8i, 21.2,16.6\pm 5.4i,\ldots \]
Also, several of the computed eigenvalues have an imaginary component,
which the exact eigenvalues do not have.

\Level 0 {LU factorization}
\index{factorization|see{LU factorization}}
%\index{LU factorization|seealso{ILU factorization}}
\index{LU factorization|(textbf}
\label{sec:lu-fact}

So far, we have looked at eliminating unknowns in the context of
solving a single system of linear equations.  Suppose you need to
solve more than one system with the same matrix, but with different
right hand sides. This happens for instance if you take multiple time
steps in an implicit method. Can you use any of the work you did in
the first system to make solving subsequent ones easier?

The answer is yes. You can split the solution process in a part that
only concerns the matrix, and a part that is specific to the right hand
side. If you have a series of systems to solve, you have to do the
first part only once, and, luckily, that even turns out to be the
larger part of the work.

Let us take a look at the same example of section~\ref{sec:gauss-example} again.
  \[ A=
  \left(
    \begin{matrix}
      6&-2&2\\ 12&-8&6\\ 3&-13&3
    \end{matrix}\right)
  \]
In the elimination process, we took
the 2nd row minus $2\times$ the first
and the 3rd row minus $1/2\times$ the first. 
Convince yourself that this combining of rows can be done by
multiplying~$A$ from the left by
  \[ 
  L_1=\left(
    \begin{matrix}
      1&0&0\\ -2&1&0\\ -1/2&0&1
    \end{matrix}\right),
  \]
which is the identity with the elimination coefficients in the first column,
below the diagonal.
The first step in elimination of variables is equivalent
to transforming the system $Ax=b$ to $L_1Ax=L_1b$.

In the next step, you subtracted $3\times$ the second row from the
third. Convince yourself that this corresponds to left-multiplying the
current matrix $L_1A$ by
  \[ L_2=\left(
    \begin{matrix}
      1&0&0\\ 0&1&0\\ 0&-3&1
    \end{matrix}\right)
  \]
We have now transformed our system $Ax=b$ into 
$L_2L_1Ax=L_2L_1b$, and $L_2L_1A$ is of `upper triangular' form.
If we define $U=L_2L_1A$, then $A=L_1\inv L_2\inv U$. How hard is it
to compute matrices such as $L_2\inv$? Remarkably easy, it turns out
to be.

We make the following observations:
  \[ 
  L_1=\left(
    \begin{matrix}
      1&0&0\\ -2&1&0\\ -1/2&0&1
    \end{matrix}\right)\qquad
  L_1\inv=\left(
    \begin{matrix}
      1&0&0\\ 2&1&0\\ 1/2&0&1
    \end{matrix}\right)
  \]
and likewise
  \[ L_2=\left(
    \begin{matrix}
      1&0&0\\ 0&1&0\\ 0&-3&1
    \end{matrix}\right)\qquad
  L_2\inv=\left(
    \begin{matrix}
      1&0&0\\ 0&1&0\\ 0&3&1
    \end{matrix}\right)
  \]
and even more remarkable:
\[ L_1^{-1}L_2^{-1} = \left(
    \begin{matrix}
      1&0&0\\ 2&1&0\\ 1/2&3&1
    \end{matrix}\right),
\]
that is, $L_1^{-1}L_2^{-1}$~contains the off-diagonal elements of
$L_1^{-1},L_2^{-1}$ unchanged, and they in turn contain the
elimination coefficients. (This is a special case of \emph{Householder
reflectors}\index{Householder reflectors!in LU factorization}; see~\ref{app:reflector}.)

\begin{exercise}
  Show that a similar statement holds, even if there are elements
  above the diagonal.
\end{exercise}

If we define $L=L_1\inv L_2\inv$, we now have $A=LU$; this is called
an \emph{LU factorization}.
We see that the
coefficients of~$L$ below the diagonal
are the negative of the coefficients used during
elimination. Even better, the first column of~$L$ can be written while
the first column of~$A$ is being eliminated, so the computation of $L$
and~$U$ can be done without extra storage, at least if we can afford
to lose~$A$.

\Level 1 {The algorithm}
\label{sec:lu-algorithm}

Let us write out the $LU$ factorization algorithm in more or less
formal code.

\newcommand\macro[1]{$\langle$#1$\rangle$} %pyskip

\begin{tabbing}
  \kern20pt\=\kern10pt\=\kill
  \macro{$LU$ factorization}:\\
  \>for\={} $k=1,n-1$:\\
  \>\>\macro{eliminate values in column $k$}\\
  \macro{eliminate values in column $k$}:\\
  \>for $i=k+1$ to $n$:\\
  \>\>\macro{compute multiplier for row $i$}\\
  \>\>\macro{update row $i$}\\
  \macro{compute multiplier for row $i$}\\
  \> $a_{ik}\leftarrow a_{ik}/a_{kk}$\\
  \macro{update row $i$}:\\
  \>for $j=k+1$ to $n$:\\
  \>\>$a_{ij}\leftarrow a_{ij}-a_{ik}*a_{kj}$
\end{tabbing}

Or, putting everything together:

\begin{tabbing}
  \kern20pt\=\kern10pt\=\kern10pt\=\kern10pt\=\kill
  \macro{$LU$ factorization}:\\
  \>for\={} $k=1,n-1$:\\
  \>\>for\={} $i=k+1$ to $n$:\\
  \>\>\> $a_{ik}\leftarrow a_{ik}/a_{kk}$\\
  \>\>\>for\={} $j=k+1$ to $n$:\\
  \>\>\>\>$a_{ij}\leftarrow a_{ij}-a_{ik}*a_{kj}$
\end{tabbing}

%\label{eq:LU-algorithm}

This is the most common way of presenting the LU
factorization. However, other ways of computing the same result
exist. Algorithms such as the LU factorization can be coded in several ways
that are mathematically equivalent, but that have different
computational behaviour. This issue, in the context of dense matrices,
is the focus of van de Geijn and Quintana's {\it The Science of
  Programming Matrix Computations}~\cite{TSoPMC}.

\Level 1 {The Cholesky factorization}
\index{Cholesky factorization|(}
\label{sec:cholesky}

The $LU$ factorization of a symmetric matrix does not give an $L$ and
$U$ that are each other's transpose: $L$~has ones on the diagonal  and
$U$~has the pivots. However it is possible to make a factorization of
a symmetric matrix $A$ of the form $A=LL^t$. This has the advantage
that the factorization takes the same amount of space as the original
matrix, namely $n(n+1)/2$ elements. We a little luck we can, just as
in the $LU$ case, overwrite the matrix with the factorization.

We derive this algorithm by reasoning inductively. Let us write $A=LL^t$
on block form:
\[ A=
\begin{pmatrix}
  A_{11}&A_{21}^t\\ A_{21}&A_{22}
\end{pmatrix} = LL^t = 
\begin{pmatrix}  \ell_{11}&0\\ \ell_{21} &L_{22} \end{pmatrix}
\begin{pmatrix}  \ell_{11}& \ell^t_{21} \\ 0&L^t_{22} \end{pmatrix}
\]
then $\ell_{11}^2=a_{11}$, from which we get~$\ell_{11}$. We also find
$\ell_{11}(L^t)_{1j}=\ell_{j1}=a_{1j}$, so we can compute the
whole first column of~$L$. Finally, $A_{22}=L_{22}L_{22}^t +
\ell_{12}\ell_{12}^t$, so 
\[ L_{22}L_{22}^t = A_{22} - \ell_{12}\ell_{12}^t, \]
which shows that $L_{22}$ is the Cholesky factor of the updated
$A_{22}$ block. Recursively, the algorithm is now defined.

\index{Cholesky factorization|)}

\Level 1 {Uniqueness}
\label{sec:LUunique}

It is always a good idea, when discussing numerical algorithms, to
wonder if different ways of computing lead to the same result. This is
referred to as the `uniqueness' of the result, and it is of practical
use: if the computed result is unique, swapping one software library
for another will not change anything in the computation.

Let us consider the uniqueness of $LU$ factorization. The definition
of an $LU$ factorization algorithm (without pivoting) is that, given a
nonsingular matrix~$A$, it will give a lower triangular matrix~$L$ and
upper triangular matrix~$U$ such that $A=LU$.  The above algorithm for
computing an $LU$ factorization is deterministic (it does not contain
instructions `take any row that satisfies\dots'), so given the same
input, it will always compute the same output. However, other
algorithms are possible, so we need worry whether they give the same result.

Let us then assume that $A=L_1U_1=L_2U_2$ where $L_1,L_2$ are lower
triangular and $U_1,U_2$ are upper triangular.
Then, $L_2\inv L_1=U_2U_1\inv$.
In that equation, the left hand side is the product
of lower triangular matrices, while the right hand side contains only
upper triangular matrices.

\begin{exercise}
  Prove that the product of lower triangular matrices is lower
  triangular, and the product of upper triangular matrices upper
  triangular. Is a similar statement true for
  \emph{inverses}\index{inverse!of triangular matrix}
  of nonsingular  triangular matrices?
\end{exercise}

The product $L_2\inv L_1$ is apparently both lower triangular and
upper triangular, so it must be diagonal. Let us call it~$D$, then
$L_1=L_2D$ and $U_2=DU_1$. The conclusion is that $LU$ factorization
is not unique, but it \emph{is} unique `up to diagonal scaling'.

\begin{exercise}
  The algorithm in section~\ref{sec:lu-algorithm} resulted in a lower
  triangular factor~$L$ that had ones on the diagonal. Show that this
  extra condition makes the factorization unique.
\end{exercise}
\begin{exercise}
  Show that an alternative condition of having ones on the diagonal of~$U$
  is also sufficient for the uniqueness of the factorization.
\end{exercise}

Since we can demand a unit diagonal in $L$ or in $U$, you may wonder
if it is possible to have both. (Give a simple argument why this is
not strictly possible.) We can do the following: suppose that $A=LU$
where $L$ and
$U$ are nonsingular lower and upper triangular, but not normalized in any
way. Write \[ L=(I+L')D_L,\qquad U=D_U(I+U'),\qquad D=D_LD_U. \]
After some renaming we now have a factorization
\begin{equation}
  A=(I+L)D(I+U)
  \label{eq:A=LUnormalized}
\end{equation}
where $D$ is a diagonal matrix containing the pivots.

\begin{exercise}
  Show that you can also normalize the factorization on the form
  \[ A=(D+L)D\inv (D+U). \]
  How does this $D$ relate to the previous?
\end{exercise}

\begin{exercise}
  Consider the factorization of a tridiagonal matrix this way. How do
  $L$ and~$U$ relate to the triangular parts of~$A$? Derive a relation
  between $D$ and $D_A$ and show that this is the equation that generates
  the pivots.
\end{exercise}

\Level 1 {Pivoting}
\label{sec:pivoting}
\index{pivoting|(}

Above, you saw examples where pivoting, that is, exchanging rows, was
necessary during the factorization process, either to guarantee the
existence of a nonzero pivot, or for numerical stability. We will now
integrate pivoting into the $LU$ factorization.

Let us first observe that row exchanges can be described by a matrix
multiplication. Let
\[ P^{(i,j)}=
\begin{array}{cc}
% first the top row
\begin{matrix}\hphantom{i}\end{matrix}
\begin{matrix}&&i&&j&\end{matrix}
\\
% then the matrix
\begin{matrix}
  \\ \\ i \\ \\ j\\ \\
\end{matrix}
\begin{pmatrix}
  1&0\\ 0&\ddots&\ddots\\ &\\
  &&0&&1\\ &&&I\\ &&1&&0\\ &&&&&I\\ &&&&&&\ddots
\end{pmatrix}
\end{array}
\]
then $P^{(i,j)}A$ is the matrix $A$ with rows $i$ and~$j$
exchanged. Since we may have to pivot in every iteration of the
factorization process, we introduce a sequence $p(\cdot)$ where $p(i)$
is the $j$ values of the row that row $i$ is switched with. We write
$P^{(i)}\equiv P^{(i,p(i))}$ for short.

\begin{exercise}
  Show that $P^{(i)}$ is its own inverse.
\end{exercise}

The process of factoring with partial pivoting can now be
described as:
\begin{itemize}
\item Let $A^{(i)}$ be the matrix with columns $1\ldots i-1$
  eliminated, and partial pivoting applied to get the desired element in
  the $(i,i)$ location.
\item Let $\ell^{(i)}$ be the vector of multipliers in the
  $i$-th elimination step. (That is, the elimination matrix $L_i$ in this
  step is the identity plus $\ell^{(i)}$ in the $i$-th column.)
\item Let $P^{(i+1)}$ (with $j\geq i+1$) be the matrix that does the
  partial pivoting for the next elimination step as described above.
\item Then $A^{(i+1)}=P^{(i+1)}L_iA^{(i)}$.
\end{itemize}
In this way we get a factorization of the form
\[ L_{n-1}P^{(n-2)}L_{n-2}\cdots L_1P^{(1)}A = U. \]
Suddenly it has become impossible to write $A=LU$: instead we
write 
\begin{equation}
  \label{eq:lu-pivot-interleave}
  A=P^{(1)}L_1\inv\cdots P^{(n-2)}L_{n-1}\inv U.
\end{equation}

\begin{exercise}
  Recall from sections \ref{sec:mvp-opt} and~\ref{sec:scalar-opt} that
  blocked algorithms are often desirable from a performance point of
  view. Why is the `$LU$ factorization with interleaved pivoting
  matrices' in equation~\eqref{eq:lu-pivot-interleave} bad news for
  performance?
\end{exercise}
Fortunately, equation~\eqref{eq:lu-pivot-interleave} can be
simplified: the $P$ and~$L$ matrices `almost commute'. We show this by
looking at an example: $P^{(2)}L_1=\tilde L_1P^{(2)}$ where $\tilde
L_1$ is very close to~$L_1$.
\[
\begin{pmatrix}
  1\\ &0&&1\\ &&I\\ &1&&0\\ &&&&I
\end{pmatrix}
\begin{pmatrix}
  1&&\emptyset\\ \vdots&1\\ \ell^{(1)}&&\ddots\\ \vdots&&&1
\end{pmatrix}
=
\begin{pmatrix}
  1&&\emptyset\\ \vdots&0&&1\\ \tilde\ell^{(1)}\\ 
  \vdots&1&&0\\ \vdots&&&&I
\end{pmatrix}
=
\begin{pmatrix}
  1&&\emptyset\\ \vdots&1\\ \tilde\ell^{(1)}&&\ddots\\ \vdots&&&1
\end{pmatrix}
\begin{pmatrix}
  1\\ &0&&1\\ &&I\\ &1&&0\\ &&&&I
\end{pmatrix}
\]
where $\tilde\ell^{(1)}$ is the same as $\ell^{(1)}$, except that
elements $i$ and~$p(i)$ have been swapped. You can now easily convince
yourself that similarly $P^{(2)}$ et cetera can be `pulled
through'~$L_1$. 

As a result we get
\begin{equation}
  \label{eq:lu-pivot}
  P^{(n-2)}\cdots P^{(1)}A
  =\tilde L_1\inv\cdots L_{n-1}\inv U=\tilde LU.
\end{equation}
This means that we can again form a matrix~$L$ just as before, except
that every time we pivot, we need to update the columns of~$L$ that
have already been computed.

\begin{exercise}
  If we write equation~\eqref{eq:lu-pivot} as $PA=LU$, we get~$A=P\inv
  LU$. Can you come up with a simple formula for $P\inv$ in terms of
  just~$P$? Hint: each $P^{(i)}$ is symmetric and its own inverse; see
  the exercise above.
\end{exercise}

\begin{exercise}
  \label{ex:no-pivot}
  Earlier, you saw that 2D \acs{BVP} (section~\ref{sec:2dbvp}) give
  rise to a certain kind of matrix. We stated, without proof, that for
  these matrices pivoting is not needed. We can now formally prove
  this, focusing on the crucial property of \indexterm{diagonal
    dominance}:
  \[ \forall_i a_{ii}\geq\sum_{j\not=i}|a_{ij}|. \]

  Assume that a matrix $A$ satisfies
  $\forall_{j\not=i}\colon a_{ij}\leq 0$. Show that the matrix is
  diagonally dominant iff there are vectors
  $u,v\geq0$ (meaning that each component is nonnegative) such that
  $Au=v$.

  Show that, after eliminating a variable, for the remaining
  matrix~$\tilde A$ there are again vectors $\tilde u,\tilde v\geq0$
  such that $\tilde A\tilde u=\tilde v$.

  Now finish the argument that (partial) pivoting is not necessary if
  $A$ is symmetric and diagonally dominant. (One can actually prove
  that pivoting is not necessary for any \indexacf{SPD} matrix, and
  diagonal dominance is a stronger condition than \ac{SPD}-ness.)
\end{exercise}

\index{pivoting|)}

\Level 1 {Solving the system}
\label{sec:lu-solve}

Now that we have a factorization $A=LU$, we can use this to solve the
linear system $Ax=LUx=b$. If we introduce a temporary vector $y=Ux$,
then we see this takes two steps:
\[ Ly=b,\qquad Ux=z. \]

The first part, $Ly=b$ is called the `lower triangular solve', since
it involves the lower triangular matrix~$L$.
  \[ \left(
    \begin{matrix}
      1&&&&\emptyset\\ \ell_{21}&1\\ \ell_{31}&\ell_{32}&1\\
      \vdots&&&\ddots\\ \ell_{n1}&\ell_{n2}&&\cdots&1
    \end{matrix}\right) \left(
    \begin{matrix}
      y_1\\ y_2\\ \\ \vdots\\ y_n
    \end{matrix}\right) = \left(
    \begin{matrix}
      b_1\\ b_2\\ \\ \vdots\\ b_n
    \end{matrix}\right)
  \]
In the first row, you see that $y_1=b_1$. Then, in the second row
$\ell_{21}y_1+y_2=b_2$, so $y_2=b_2-\ell_{21}y_1$. You can imagine how
this continues: in every $i$-th row you can compute $y_i$ from the
previous $y$-values:
\[ y_i = b_i-\sum_{j<i} \ell_{ij}y_j. \]
Since we compute $y_i$ in increasing order, this is also known as
the forward substitution, forward solve, or forward sweep.
  

The second half of the solution process, the upper triangular solve,
backward substitution, or backward sweep, computes $x$ from $Ux=y$:
  \[ \left(
    \begin{matrix}
      u_{11}&u_{12}&\ldots&u_{1n}\\ &u_{22}&\ldots&u_{2n}\\
      &&\ddots&\vdots\\ \emptyset&&&u_{nn}
    \end{matrix}\right) \left(
    \begin{matrix}
      x_1\\ x_2\\ \vdots\\ x_n
    \end{matrix}\right)=\left(
    \begin{matrix}
      y_1\\ y_2\\ \vdots\\ y_n
    \end{matrix}\right)
  \]
Now we look at the last line, which immediately tells us that
$x_n=u_{nn}^{-1}y_n$.
From this, the line before the last states
$u_{n-1n-1}x_{n-1}+u_{n-1n}x_n=y_{n-1}$, which gives
$x_{n-1}=u_{n-1n-1}^{-1}(y_{n-1}-u_{n-1n}x_n)$.
In general, we can compute
\[ x_i=u_{ii}\inv (y_i-\sum_{j>i}u_{ij}y_j) \]
for decreasing values of~$i$.

\begin{exercise}
  In the backward sweep you had to divide by the
  numbers~$u_{ii}$. That is not possible if any of them are zero.
  Relate this problem back to the above discussion. 
\end{exercise}

\Level 1 {Complexity}
\index{complexity!computational|(}

In the beginning of this chapter, we indicated that not every method
for solving a linear system takes the same number of operations. Let
us therefore take a closer look at the complexity
\begin{footnoteenv}
  {See
    appendix~\ref{app:complexity} for an explanation of complexity.}
\end{footnoteenv}
,
that is, the number of operations as function of the problem size, of
the use of an LU factorization in solving the linear system.

First we look at the computation of $x$ from $LUx=b$
(`solving the linear system'; section~\ref{sec:lu-solve}), given that
we already have the factorization $A=LU$.
Looking at the lower and upper
triangular part together, you see that you perform a multiplication
with all off-diagonal elements (that is, elements $\ell_{ij}$
or~$u_{ij}$ with $i\not=j$). Furthermore, the upper triangular solve
involves divisions by the $u_{ii}$ elements. Now, division operations
are in general much more expensive than multiplications, so in this
case you would compute the values $1/u_{ii}$, and store them instead.

\begin{exercise}
  Take a look at the factorization algorithm, and argue that storing
  the reciprocals of the pivots does not add to the computational
  complexity.
\end{exercise}

Summing up, you see that, on a system of size $n\times n$, you perform
$n^2$ multiplications and roughly the same number of additions.
This shows that, given a factorization, solving a linear system
has the same complexity as a simple matrix-vector multiplication,
that is, of computing $Ax$ given $A$ and~$x$.

The complexity of constructing the $LU$ factorization is a bit more
involved to compute. Refer to the algorithm in section~\ref{sec:lu-algorithm}.
You see that in the $k$-th step two things happen: the computation of
the multipliers, and the updating of the rows.
%
There are $n-k$ multipliers to be computed, each of which involve a
division. After that, the update takes $(n-k)^2$ additions and
multiplications. If we ignore the divisions for now, because there are
fewer of them, we find that the $LU$ factorization takes
$\sum_{k=1}^{n-1} 2(n-k)^2$
operations. If we number the terms in this sum in the reverse order,
we find \[ \#\mathrm{ops}=\sum_{k=1}^{n-1} 2k^2. \]
Since we can approximate a sum by an integral,
we find that this is $2/3n^3$ plus some lower
order terms. This is of a higher order than solving the linear system:
as the system size grows, the cost of constructing the $LU$
factorization completely dominates.

Of course there is more to algorithm analysis than operation counting.
While solving the linear system has the same complexity
as a matrix-vector multiplication, the two operations
are of a very different nature.
One big difference is that both the factorization
and forward/backward solution involve recursion, so they are not simple
to parallelize. We will say more about that later on.

\index{complexity!computational|)}

\begin{comment}
\Level 1 {Accuracy}

In section \ref{sec:linear-arith} you saw some simple examples of the
problems that stem from the use of computer arithmetic, and how these
motivated the use of pivoting. Even with pivoting, however, we still
need to worry about the accumulated effect of roundoff
errors. A~productive way of looking at the question of attainable
accuracy is to consider that by solving a system $Ax=b$ we get
a numerical solution $x+\Delta x$ which is the exact solution of a
slightly different linear system: 
\[ (A+\Delta A)(x+\Delta x)=b+\Delta b. \]
Analyzing these statements quickly leads to bounds such as 
\[ \frac{\|\Delta x\|}{\|x\|}\leq 
  \frac{2\epsilon \kappa(A)}{1-\epsilon\kappa(A)} 
\]
where $\epsilon$ is the \indexterm{machine precision} (see
section~\ref{sec:machine-eps}) and $\kappa(A)=\|A\|\|A\inv\|$ is
called the \indexterm{condition number} of the matrix~$A$ (see
appendix~\ref{app:norms}). Without going into this in any detail, we
remark that the condition number is related to eigenvalues (strictly
speaking: singular values) of the
matrix.

The analysis of the accuracy of algorithms is a field of study in
itself; see for instance the book by Higham~\cite{Higham:2002:ASN}.
\end{comment}

\Level 1 {Block algorithms}
\label{sec:block-algebra}

Often, matrices have a natural block structure, such as in the case of
two-dimensional \ac{BVP}s; section~\ref{sec:2dbvp}. Many linear
algebra operations can be formulated in terms of these blocks.  This
can have several advantages over the traditional scalar view of
matrices. For instance, it improves \indextermbus{cache}{blocking}
(section~\ref{sec:loop-tiling}); it also facilitates scheduling linear
algebra algorithms on \indexterm{multicore} architectures
(section~\ref{sec:multicore-block}).

For block algorithms
we write a matrix as 
\[ A=
\begin{pmatrix}
  A_{11}&\ldots&A_{1N}\\ \vdots&&\vdots\\ A_{M1}&\ldots&A_{MN}
\end{pmatrix}
\] 
where $M,N$ are the block dimensions, that is, the dimension expressed
in terms of the subblocks. Usually, we choose the blocks such that
$M=N$ and the diagonal blocks are square.

As a simple example, consider the
\emph{matrix-vector product}\index{matrix!vector product!block algorithm}
$y=Ax$, expressed in block terms.
\[ 
\begin{pmatrix}
  Y_1\\ \vdots\\ Y_M
\end{pmatrix}=
\begin{pmatrix}
  A_{11}&\ldots&A_{1M}\\ \vdots&&\vdots\\ A_{M1}&\ldots&A_{MM}
\end{pmatrix}
\begin{pmatrix}
  X_1\\ \vdots\\ X_M
\end{pmatrix}
\]
To see that the block algorithm computes the same result as the old
scalar algorithm, we look at a component $Y_{i_k}$, that is the $k$-th
scalar component of the $i$-th block. First,
\[ Y_i=\sum_j A_{ij}X_j \]
so 
\[
  Y_{i_k}=\bigl(\sum_j A_{ij}X_j\bigr)_k=
  \sum_j \left(A_{ij}X_j\right)_k=\sum_j\sum_\ell A_{ij_{k\ell}}X_{j_\ell}
\]
which is the product of the $k$-th row of the $i$-th blockrow of~$A$
with the whole of~$X$.

A more interesting algorithm is the block version of the
\emph{LU factorization}\index{LU factorization!block algorithm}.
The algorithm \eqref{eq:LU-algorithm} then becomes

\begin{equation}
 \vcenter{%
\begin{tabbing}
  \kern20pt\=\kern10pt\=\kern10pt\=\kern10pt\=\kill
  \macro{$LU$ factorization}:\\
  \>for $k=1,n-1$:\\
  \>\>for $i=k+1$ to $n$:\\
  \>\>\> $A_{ik}\leftarrow A_{ik}A_{kk}\inv$\\
  \>\>\>for $j=k+1$ to $n$:\\
  \>\>\>\>$A_{ij}\leftarrow A_{ij}-A_{ik}\cdot A_{kj}$
\end{tabbing}
  }
\label{eq:LU-block-algorithm}
\end{equation}

which mostly differs from the earlier algorithm in that the division
by $a_{kk}$ has been replaced by a multiplication
by~$A_{kk}\inv$. Also, the $U$ factor will now have pivot blocks,
rather than pivot elements, on the diagonal, so $U$ is only `block
upper triangular', and not strictly upper triangular.

\begin{exercise}
  We would like to show that the block algorithm here again computes
  the same result as the scalar one. Doing so by looking explicitly at
  the computed elements is cumbersome, so we take another
  approach.  First, recall from section~\ref{sec:LUunique} that $LU$ factorizations are unique: if
  $A=L_1U_1=L_2U_2$ and $L_1,L_2$ have unit diagonal, then $L_1=L_2$,
  $U_1=U_2$.

  Next, consider the computation of $A_{kk}\inv$. Show that this can
  be done easily by first computing an LU factorization
  of~$A_{kk}$. Now use this to show that the block LU factorization
  can give $L$ and $U$ factors that are strictly triangular. The
  uniqueness of $LU$ factorizations then proves that the block
  algorithm computes the scalar result.
\end{exercise}

\index{LU factorization|)}

\Level 0 {Sparse matrices}
\label{sec:sparse}
\index{linear algebra!sparse|see{sparse!linear algebra}}
\index{sparse!linear algebra|(}
\input chapters/sparse
\index{sparse!linear algebra|)}

\Level 0 {Iterative methods}
\label{sec:iterative}
\index{Krylov method|see{iterative method}}

\index{iterative method!stationary|(}
\input chapters/stationary
\index{iterative method!stationary|)}

\index{polynomial!iterative method|see{iterative method, polynomial}}
\index{iterative method!polynomial|(}
\input chapters/cg
\index{iterative method!polynomial|)}

\Level 0 {Eigenvalue methods}
\label{sec:eigen}
\index{eigenvalue problems|(}
\input chapters/eigenvalue
\index{eigenvalue problems|)}

\furtherreading

Iterative methods is a very deep field. As a practical introduction to
the issues involved, you can read the `Templates
book'~\cite{Ba:templates}, online at
\url{http://netlib.org/templates/}. For a deeper treatment of the
theory, see the book by Saad~\cite{saad96} of which the first edition
can be downloaded at
\url{http://www-users.cs.umn.edu/~saad/books.html}.
