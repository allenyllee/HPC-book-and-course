% -*- latex -*-
%%%%%%%%%%%%%%%%%%%%%%%%%%%%%%%%%%%%%%%%%%%%%%%%%%%%%%%%%%%%%%%%
%%%%%%%%%%%%%%%%%%%%%%%%%%%%%%%%%%%%%%%%%%%%%%%%%%%%%%%%%%%%%%%%
%%%%
%%%% This text file is part of the source of 
%%%% `Introduction to High-Performance Scientific Computing'
%%%% by Victor Eijkhout, copyright 2012-8
%%%%
%%%% This book is distributed under a Creative Commons Attribution 3.0
%%%% Unported (CC BY 3.0) license and made possible by funding from
%%%% The Saylor Foundation \url{http://www.saylor.org}.
%%%%
%%%%
%%%%%%%%%%%%%%%%%%%%%%%%%%%%%%%%%%%%%%%%%%%%%%%%%%%%%%%%%%%%%%%%
%%%%%%%%%%%%%%%%%%%%%%%%%%%%%%%%%%%%%%%%%%%%%%%%%%%%%%%%%%%%%%%%

In this chapter we have so far limited ourselves to linear system
solving. \emph{Eigenvalue problems} are another important category of
linear algebra applications, but their interest lies more in the
mathematics than the computations as such.
We give a brief outline of the type of computations involved.

\Level 1 {Power method}

The \indextermdef{power method} is a simple iteration process:
%
given a matrix $A$ and an arbitrary starting vector~$v$, compute
repeatedly
\[ v\leftarrow Av,\quad v\leftarrow v/\|v\|. \]
The vector $v$ quickly becomes the eigenvector corresponding to the
eigenvalue with maximum absolute size, and so $\|Av\|/\|v\|$ becomes
an approximation to that largest eigenvalue.

Appying the power method to $A\inv$ is known as
\indextermbusdef{inverse}{iteration}%
\index{iteration!inverse|see{inverse, iteration}}
and it yields the inverse of the eigenvalue that is smallest in
absolute magnitude.

Another variant of the power method is the
\indextermsub{shift-and-inverse}{iteration}%
\index{shift-and-inverse!see{iteration, shift-and-inverse}}
which can be used to find interior eigenvalues. If $\sigma$ is close
to an interior eigenvalue, then inverse iteration on $A-\sigma I$ will
find that interior eigenvalue.

\Level 1 {Orthogonal iteration schemes}

The fact that eigenvectors to different eigenvalues are orthogonal can
be exploited. For instance, after finding one eigenvector, one could
iterate in the subspace orthogonal to that. Another option is to
iterate on a block of vectors, and orthogonalizing this block after
each power method iteration. This produces as many dominant eigenvalue
as the block size. The \indextermsub{restarted}{Arnoldi}
method~\cite{leho:95} is an example of such a scheme.

\Level 1 {Full spectrum methods}

The iterative schemes just discussed yield only localized eigenvalues.
Other methods compute the full spectrum of a matrix. The most popular
of these is the \indextermbus{QR}{method}.

\Level 1 {Parallel execution}

Lanczos-based schemes are more readily parallelized than the QR
method; see~\cite{berry:inria-svdpar} for a discussion.
