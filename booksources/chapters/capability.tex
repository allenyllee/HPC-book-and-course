%%%%%%%%%%%%%%%%%%%%%%%%%%%%%%%%%%%%%%%%%%%%%%%%%%%%%%%%%%%%%%%%
%%%%%%%%%%%%%%%%%%%%%%%%%%%%%%%%%%%%%%%%%%%%%%%%%%%%%%%%%%%%%%%%
%%%%
%%%% This text file is part of the source of 
%%%% `Introduction to High-Performance Scientific Computing'
%%%% by Victor Eijkhout, copyright 2012
%%%%
%%%% This book is distributed under a Creative Commons Attribution 3.0
%%%% Unported (CC BY 3.0) license and made possible by funding from
%%%% The Saylor Foundation \url{http://www.saylor.org}.
%%%%
%%%%
%%%%%%%%%%%%%%%%%%%%%%%%%%%%%%%%%%%%%%%%%%%%%%%%%%%%%%%%%%%%%%%%
%%%%%%%%%%%%%%%%%%%%%%%%%%%%%%%%%%%%%%%%%%%%%%%%%%%%%%%%%%%%%%%%

Large parallel computers can be used in two different ways. In later
chapters you will see how scientific problems can be scaled up almost
arbitrarily. This means that with an increasing need for accuracy or
scale, increasingly large computers are needed. The use of a whole
machine for a single problem, with only time-to-solution as the
measure of success, is known as \indexterm{capability computing}.

On the other hand, many problems need less than a whole supercomputer
to solve, so typically a computing center will set up a machine so
that it serves a continuous stream of user problems, each smaller than
the full machine. In this mode, the measure of success is the
sustained performance per unit cost. This is known as
\indexterm{capacity computing}, and it requires a finely tuned
\indextermsub{job}{scheduling} strategy. 

A~popular scheme is
\indextermsub{fair-share}{scheduling}, which tries to allocate resources
equally between users, rather than between processes. This means
that it will lower a user's priority if the user had recent jobs, and it will 
give higher priority to short or small jobs. Examples of schedulers 
on this principle are \indexterm{SGE} and \indexterm{Slurm}.

Jobs can have dependencies, which makes scheduling harder. In fact,
under many realistic conditions scheduling problems are
\indexterm{NP-complete}, so in practice heuristics will be used.
This topic, while interesting, is not further discussed in this book.
