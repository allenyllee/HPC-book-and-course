% this file doesn't seem to be used
%%%%%%%%%%%%%%%%%%%%%%%%%%%%%%%%%%%%%%%%%%%%%%%%%%%%%%%%%%%%%%%%
%%%%%%%%%%%%%%%%%%%%%%%%%%%%%%%%%%%%%%%%%%%%%%%%%%%%%%%%%%%%%%%%
%%%%
%%%% This text file is part of the source of 
%%%% `Introduction to High-Performance Scientific Computing'
%%%% by Victor Eijkhout, copyright 2012
%%%%
%%%% This book is distributed under a Creative Commons Attribution 3.0
%%%% Unported (CC BY 3.0) license and made possible by funding from
%%%% The Saylor Foundation \url{http://www.saylor.org}.
%%%%
%%%%
%%%%%%%%%%%%%%%%%%%%%%%%%%%%%%%%%%%%%%%%%%%%%%%%%%%%%%%%%%%%%%%%
%%%%%%%%%%%%%%%%%%%%%%%%%%%%%%%%%%%%%%%%%%%%%%%%%%%%%%%%%%%%%%%%

\input chapters/robertmacros

In this section, we discuss the parallelization of the
computation $ y \becomes A x $, where $ x, y \in \Rn $ and $ A \in \Rnxn $.
We will assume that $ p $ nodes will be used, identified by $ {\cal P}_0, 
\ldots , {\cal P}_{p-1} $.

We partition $A$ by rows:
\[
A \rightarrow \defcolvector{A}{p} 
\quad
x \rightarrow \defcolvector{x}{p} ,
\quad
\mbox{and}
\quad
y \rightarrow \defcolvector{y}{p} ,
\]
where $ A_i \in \R^{m_i \times n} $ and $ x_i, y_i \in \R^{m_i} $ with
$ \sum_{i=0}^{p-1} m_i = n $ and $ m_i \approx n / p $.

We will start by assuming
that $ A_i $, $ x_i $, and $ y_i $ are originally assigned to $ {\cal P}_i $.

\paragraph*{Algorithm}

An algorithm for computing $ y = A x $ in parallel is then given by
\\[0.1in]
\begin{center}
\begin{tabular}{| p{3.5in} |  p{3.0in} |}\hline
Step & Cost (lower bound) \\ \whline
Allgather $ x_i $ so that $ x $ is available on all nodes & 
$ \lceil \log_2(p)\rceil \alpha + \frac{p-1}{p} n \beta \approx
\log_2(p) \alpha + n \beta $ \\
Locally compute $ y_i = A_i x $ &
$ \approx 2 \frac{n^2}{p} \gamma $ \\ \hline
\end{tabular}
\end{center}
where $\alpha$ and $\beta$ are the latency and bandwidth,
respectively.

\paragraph*{Cost analysis}

The total cost of the algorithm is given by, approximately,
\[
T_p(n) = T_p^{\mbox{1D-row}}(n) = 
2 \frac{n^2}{p} \gamma + 
\begin{array}[t]{c}
\underbrace{\log_2(p) \alpha + n \beta.}
\\
\mbox{Overhead}
\end{array}
\]
Since the sequential cost is $ T_1(n) = 2 n^2 \gamma $, the speedup is given by
\[
S_p^{\mbox{1D-row}}(n) = 
\frac{T_1(n)}
{T_p^{\mbox{1D-row}}(n)} = 
\frac{2 n^2 \gamma}
{ 2 \frac{n^2}{p} \gamma + 
\log_2(p) \alpha + n \beta}
= 
\frac{p}
{ 1 + \frac{p \log_2(p)}{2 n^2} \frac{\alpha}{\gamma} 
+ \frac{p}{2 n} \frac{\beta}{\gamma} }
\]
and the parallel efficiency by
\[
E_p^{\mbox{1D-row}}(n) = 
\frac{S_p^{\mbox{1D-row}}(n)}{p}
= 
\frac{1}
{ 1 + \frac{p \log_2(p)}{2 n^2} \frac{\alpha}{\gamma} 
+ \frac{p}{2 n} \frac{\beta}{\gamma} }.
\]

\begin{exercise}
  Show that, for fixed $p$ and growing $n$, the efficiency goes
  to~1. Why is this result not practically useful? Show that, for
  fixed~$n$ and $p$~growing, efficiency goes to zero. Why is that
  result not surprising?
\end{exercise}

A realistic analysis recognizes that the total amount of memory that is
available to store the problem scales linearly with~$ p $.  Let $ M $
equal the number of floating point numbers that can be stored in a
single node's memory.  The aggregate memory is then given by~$ M p $.
Let $ n_{\mbox{max}}(p) $ equal the largest problem size that can be
stored in the aggregate memory of $ p $ nodes.  Then, if {\em all}
memory can be used for the matrix,
\[
(n_{\mbox{max}}(p))^2 = M p
\quad
\mbox{or}
\quad
n_{\mbox{max}}(p) = \sqrt{M} {p}.
\]
The efficiency for the largest problem that can be stored on $ p $
nodes is then:
\begin{eqnarray*}
E_p^{\mbox{1D-row}}(n_{\mbox{max}}(p)) &=& 
\frac{1}
{1 + \frac{p \log_2(p)}{2 (n_{\mbox{max}}(p))^2} \frac{\alpha}{\gamma} 
+ \frac{p}{2 n_{\mbox{max}}(p)} \frac{\beta}{\gamma} }
\\
&=&
\frac{1}
{ 1 + \frac{\log_2(p)}{2 M} \frac{\alpha}{\gamma} 
+ \frac{\sqrt{p}}{2 \sqrt{M}} \frac{\beta}{\gamma} }.
\end{eqnarray*}
Now, if one analyzes what happens when the number of nodes
becomes large, one finds that
\[
\lim_{p \rightarrow \infty} E_p( n_{\mbox{max}}(p) ) 
=
\lim_{p \rightarrow \infty}
\left[
\frac{1}
{1 + \frac{\log_2(p)}{2 M} \frac{\alpha}{\gamma} 
+ \frac{\sqrt{p}}{2 \sqrt{M}} \frac{\beta}{\gamma} }
\right]
=
0.
\]
Thus, this parallel algorithm for matrix-vector multiplication
does not scale.

Alternatively, a realist realizes that he/she has a limited amount of time,
$ T_{\mbox{max}} $ on
his/her hands.  Under the best of circumstances, the largest problem
that we can solve in time $ T_{\mbox{max}} $ is given by
\[ 
T_p(n_{\mbox{max}}(p)) = 2 \frac{(n_{\mbox{max}}(p))^2}{p} \gamma = T_{\mbox{max}} .
\]
Thus
\[
(n_{\mbox{max}}(p))^2 = \frac{T_{\mbox{max}} p}{2 \gamma}
\quad
\mbox{or}
\quad
n_{\mbox{max}}(p) = \frac{\sqrt{T_{\mbox{max}}} \sqrt{p}}{\sqrt{2 \gamma}}.
\]
Then the parallel efficiency that is attained by the algorithm for the largest
problem that can be solved in time $ T_{\mbox{max}} $ is given by
\[
~
\cdots
~\]
and the parallel efficiency as the number of nodes becomes large approaches
\[
~
\cdots
~
\]
Again, efficiency cannot be maintained as the number of processors increases and the execution time is capped.

\begin{exercise}
  Perform a similar analysis for a partitioning by columns.
\end{exercise}
