% this file is no longer used

%%%%%%%%%%%%%%%%%%%%%%%%%%%%%%%%%%%%%%%%%%%%%%%%%%%%%%%%%%%%%%%%
%%%%%%%%%%%%%%%%%%%%%%%%%%%%%%%%%%%%%%%%%%%%%%%%%%%%%%%%%%%%%%%%
%%%%
%%%% This text file is part of the source of 
%%%% `Introduction to High-Performance Scientific Computing'
%%%% by Victor Eijkhout, copyright 2012
%%%%
%%%% This book is distributed under a Creative Commons Attribution 3.0
%%%% Unported (CC BY 3.0) license and made possible by funding from
%%%% The Saylor Foundation \url{http://www.saylor.org}.
%%%%
%%%%
%%%%%%%%%%%%%%%%%%%%%%%%%%%%%%%%%%%%%%%%%%%%%%%%%%%%%%%%%%%%%%%%
%%%%%%%%%%%%%%%%%%%%%%%%%%%%%%%%%%%%%%%%%%%%%%%%%%%%%%%%%%%%%%%%

Collective operations play an important part in linear algebra
operations. In fact, the scalability of the operations can depend on
the cost of these collectives. (See~\cite{Chan2007Collective} for
details.)

\index{3@$\alpha$|see{latency}}
\index{3@$\beta$|see{bandwidth}}
\index{3@$\gamma$|see{computation rate}}
In computing the cost of a collective operation, three architectural
constants are enough to give lower bounds: $\alpha$,~the \index{latency} of
sending a single message, $\beta$,~the inverse of the \index{bandwidth} for
sending data (see section~\ref{sec:latencybandwidth}), and $\gamma$,
the inverse of the \index{computation rate}, the
time for performing an arithmetic operation. Sending $n$ data
items then takes time $\alpha +\beta n$. We further assume that a
processor can only send one message at a time. We make no assumptions
about the connectivity of the processors; thus, the lower bounds
derived here will hold for a wide range of architectures.

The main implication of the architectural model above is that the
number of active processors can only double in each step of an
algorithm. For instance, to do a broadcast, first processor~0 sends
to~1, then 0~and~1 can send to 2~and~3, then 0--3 send to 4--7, et
cetera. This cascade of messages is called
a \indextermbus{minimum}{spanning tree} of the processor network, and
it follows that any collective algorithm has at least $\alpha\log_2p$
cost associated with the accumulated latencies.

\Level 2 {Broadcast}
\index{broadcast|(}

In a \emph{broadcast} operation, a single processor has $n$ data elements
that is needs to send to all others: the other processors need a full
copy of all $n$ elements.  By
the above doubling argument, we conclude that a broadcast to $p$
processors takes time at least $\lceil\log_2 p\rceil$ steps with a
total latency of $\lceil\log_2 p\rceil \alpha$. Since $n$ elements are
sent, this adds at least a time $n\beta$ for all elements to leave the
sending processor, giving a total cost lower bound of
\[ \lceil\log_2 p\rceil \alpha+n\beta. \]

We can illustrate the spanning tree method as follows:
\[
\begin{array}{|c|cccc|}
\hline
  &t=0&t=1&t=2&t=3\\ \hline
p_0&x_0,x_1,x_2,x_3&x_0\downarrow,x_1\downarrow,x_2\downarrow,x_3\downarrow
   &x_0\downarrow,x_1\downarrow,x_2\downarrow,x_3\downarrow
   &x_0,x_1,x_2,x_3\\
p_1&&&x_0\downarrow,x_1\downarrow,x_2\downarrow,x_3\downarrow
   &x_0,x_1,x_2,x_3\\
p_2&&&&x_0,x_1,x_2,x_3\\
p_3&&&&x_0,x_1,x_2,x_3\\
\hline
\end{array}
\]
(On $t=1$, $p_0$ sends to $p_1$; on $t=2$ $p_0,p_1$ send to $p_2,p_3$.)
This algorithm has the correct $\log_2p\cdot\alpha$ term, but processor~0 repeatedly
sends the whole vector, so the bandwidth cost is $\log_2p\cdot n\beta$.

The following algorithm implements the broadcast as a combination of a scatter
and a \indexterm{bucket brigade algorithm}:
\[
\begin{array}{|c|lll|}
\hline
  &t=0&t=1&et cetera\\ \hline
p_0&x_0\downarrow\hphantom{,x_1\downarrow,x_2\downarrow,x_3\downarrow}
   &x_0\hphantom{\downarrow,x_1\downarrow,x_2\downarrow,}x_3\downarrow
   &x_0,\hphantom{x_1,}x_2,x_3\\
p_1&\hphantom{x_0\downarrow,}x_1\downarrow\hphantom{,x_2\downarrow,x_3\downarrow}
   &x_0\downarrow,x_1\hphantom{\downarrow,x_2\downarrow,x_3\downarrow}
   &x_0,x_1,\hphantom{x_2,}x_3\\
p_2&\hphantom{x_0\downarrow,x_1\downarrow,}x_2\downarrow\hphantom{,x_3\downarrow}
   &\hphantom{x_0\downarrow,}x_1\downarrow,x_2\hphantom{\downarrow,x_3\downarrow}
   &x_0,x_1,x_2\hphantom{,x_3}\\
p_3&\hphantom{x_0\downarrow,x_1\downarrow,x_2\downarrow,}x_3\downarrow
   &\hphantom{x_0\downarrow,x_1\downarrow,}x_2\downarrow,x_3\hphantom{\downarrow}
   &\hphantom{x_0,}x_1,x_2,x_3\\
\hline
\end{array}
\]
The complexity now becomes \[ p\alpha+\beta n(p-1)/p \]
which is not optimal in latency, but is a better algorithm if $n$ is large.

\index{broadcast|)}

\Level 2 {Reduction}
\index{reduction|(}

In a \emph{reduction} operation, each processor has $n$ data elements, and
one processor needs to combine them elementwise, for instance
computing $n$ sums or products.

By running the broadcast backwards in time, we see that a reduction
operation has the same lower bound on the communication of
$\lceil\log_2 p\rceil \alpha+n\beta$.  A~reduction operation also
involves computation, which would take a total time of $(p-1)\gamma n$
sequentially: each of
$n$ items gets reduced over $p$ processors. Since these operations can
potentially be parallelized, the lower bound on the computation is
$\frac{p-1}p \gamma n$, giving a total of
    \[ \lceil\log_2 p\rceil \alpha+n\beta +\frac{p-1}p \gamma n. \]

We illustrate this again, using the notation $x_i^{(j)}$ for the data item~$i$
that was originally on processor~$j$, and $x_i^{(j:k)}$ for the sum of
the items~$i$ of processors $j\ldots k$.

\[
\begin{array}{|c|ccc|}
\hline
  &t=1&t=2&t=3\\ \hline
p_0&x_0^{(0)},x_1^{(0)},x_2^{(0)},x_3^{(0)}
   &x_0^{(0:1)},x_1^{(0:1)},x_2^{(0:1)},x_3^{(0:1)}
   &x_0^{(0:3)},x_1^{(0:3)},x_2^{(0:3)},x_3^{(0:3)}\\
p_1&x_0^{(1)}\uparrow,x_1^{(1)}\uparrow,x_2^{(1)}\uparrow,x_3^{(1)}\uparrow
   &&\\
p_2&x_0^{(2)},x_1^{(2)},x_2^{(2)},x_3^{(2)}
   &x_0^{(2:3)}\uparrow,x_1^{(2:3)}\uparrow,x_2^{(2:3)}\uparrow,x_3^{(2:3)}\uparrow
   &\\
p_3&x_0^{(3)}\uparrow,x_1^{(3)}\uparrow,x_2^{(3)}\uparrow,x_3^{(3)}\uparrow
   &&\\
\hline
\end{array}
\]

On time $t=1$ processors $p_0,p_2$ receive from $p_1,p_3$, and on
$t=2$ $p_0$ receives from~$p_2$.

\index{reduction|)}

\Level 2 {Allreduce}
\index{allreduce|(}

An \emph{allreduce} operation computes the same elementwise reduction of $n$
elements on each processor, but leaves the result on each processor,
rather than just on the root of the spanning tree. This could be
implemented as a reduction followed by a broadcast, but more clever
algorithms exist.

The lower bound on the cost of an allreduce is, somewhat remarkably,
almost the same as of a simple reduction: since in a reduction not all
processors are active at the same time, we assume that the extra work
can be spread out perfectly. This means that the lower bound on the
latency and computation stays the same. For the bandwidth we reason as
follows: in order for the communication to be perfectly parallelized,
$\frac{p-1}p n$ items have to arrive at, and leave each
processor. Thus we have a total time of
\[ \lceil \log_2 p\rceil\alpha +2\frac{p-1}pn\beta
    +\frac{p-1}pn\gamma. \]

\index{allreduce|)}

\Level 2 {Allgather}
\index{allgather|(}

In a \indexterm{gather} operation on $n$ elements, each processor has
$n/p$ elements, and one processor collects them all, without combining
them as in a reduction. The \emph{allgather} computes the same gather,
but leaves the result on all processors.

Again we assume that gathers with multiple
targets are active simultaneously. Since every processor originates a
minimum spanning tree, we have $\log_2p\alpha$ latency; since each
processor receives $n/p$ elements from $p-1$ processors, there is
$\frac{p-1}p\beta$ bandwidth cost. The total cost for constructing a
length~$n$ vector by allgather is then
\[ \lceil \log_2 p\rceil\alpha +\frac{p-1}pn\beta. \]

\[
\begin{array}{|c|ccc|}
\hline
  &t=1&t=2&t=3\\ \hline
p_0&x_0\downarrow&x_0x_1\downarrow&x_0x_1x_2x_3\\
p_1&x_1\uparrow&x_0x_1\downarrow&x_0x_1x_2x_3\\
p_2&x_2\downarrow&x_2x_3\uparrow&x_0x_1x_2x_3\\
p_3&x_3\uparrow&x_2x_3\uparrow&x_0x_1x_2x_3\\
\hline
\end{array}
\]
At time $t=1$, there is an exchange between neighbours $p_0,p_1$ and
likewise $p_2,p_3$; at $t=2$ there is an exchange over distance two
between $p_0,p_2$ and likewise~$p_1,p_3$.

\index{allgather|)}

\Level 2 {Reduce-scatter}
\index{reduce-scatter|(}

In a \indexterm{reduce-scatter} operation, each processor has $n$
elements, and an $n$-way reduction is done on them. Unlike in the
reduce or allreduce, the result is then broken up, and distributed as
in a scatter operation.

Formally, processor~$i$ has an
item~$x_i^{(i)}$, and it needs $\sum_j x_i^{(j)}$. We could implement
this by doing a size~$p$ reduction, collecting the vector $(\sum_i
x_0^{(i)},\sum_i x_1^{(i)},\ldots)$ on one processor, and scattering
the results. However it is possible to combine these operations in a
so-called \indexterm{bidirectional exchange} algorithm:

\[
\begin{array}{|c|ccc|}
\hline
  &t=1&t=2&t=3\\ \hline
p_0&x_0^{(0)},x_1^{(0)},x_2^{(0)}\downarrow,x_3^{(0)}\downarrow
   &x_0^{(0:2:2)},x_1^{(0:2:2)}\downarrow
    \hphantom{x_0^{(0:2:2)},x_1^{(0:2:2)}\downarrow}
   &x_0^{(0:3)}
    \hphantom{x_3^{(0:3)}x_3^{(0:3)}x_3^{(0:3)}}\\
p_1&x_0^{(1)},x_1^{(1)},x_2^{(1)}\downarrow,x_3^{(1)}\downarrow
   &x_0^{(1:3:2)}\uparrow,x_1^{(1:3:2)}
    \hphantom{x_0^{(0:2:2)},x_1^{(0:2:2)}\downarrow}
   &\hphantom{x_3^{(0:3)}} x_1^{(0:3)}
    \hphantom{x_3^{(0:3)}x_3^{(0:3)}} \\
p_2&x_0^{(2)}\uparrow,x_1^{(2)}\uparrow,x_2^{(2)},x_3^{(2)}
   &\hphantom{x_0^{(0:2:2)},x_1^{(0:2:2)}\downarrow}
    x_2^{(0:2:2)},x_3^{(0:2:2)}\downarrow
   &\hphantom{x_3^{(0:3)}x_3^{(0:3)}} x_2^{(0:3)}
    \hphantom{x_3^{(0:3)}}\\
p_3&x_0^{(3)}\uparrow,x_1^{(3)}\uparrow,x_2^{(3)},x_3^{(3)}
   &\hphantom{x_0^{(0:2:2)},x_1^{(0:2:2)}\downarrow}
    x_0^{(1:3:2)}\uparrow,x_1^{(1:3:2)}
   &\hphantom{x_3^{(0:3)}x_3^{(0:3)}x_3^{(0:3)}}
    x_3^{(0:3)}\\
\hline
\end{array}
\]

The reduce-scatter can be considered as a allgather run in reverse,
with arithmetic added, so the cost is 
\[ \lceil \log_2 p\rceil\alpha +\frac{p-1}pn(\beta+\gamma). \]

\index{reduce-scatter|)}
