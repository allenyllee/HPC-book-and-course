%%%%%%%%%%%%%%%%%%%%%%%%%%%%%%%%%%%%%%%%%%%%%%%%%%%%%%%%%%%%%%%%
%%%%%%%%%%%%%%%%%%%%%%%%%%%%%%%%%%%%%%%%%%%%%%%%%%%%%%%%%%%%%%%%
%%%%
%%%% This text file is part of the source of 
%%%% `Introduction to High-Performance Scientific Computing'
%%%% by Victor Eijkhout, copyright 2012
%%%%
%%%% This book is distributed under a Creative Commons Attribution 3.0
%%%% Unported (CC BY 3.0) license and made possible by funding from
%%%% The Saylor Foundation \url{http://www.saylor.org}.
%%%%
%%%%
%%%%%%%%%%%%%%%%%%%%%%%%%%%%%%%%%%%%%%%%%%%%%%%%%%%%%%%%%%%%%%%%
%%%%%%%%%%%%%%%%%%%%%%%%%%%%%%%%%%%%%%%%%%%%%%%%%%%%%%%%%%%%%%%%

Modern architectures are often a mix of shared and distributed
memory. For instance, a cluster will be distributed on the level of
the nodes, but sockets and cores on a node will have shared
memory. One level up, each socket can have a shared L3 cache but
separate L2 and L1 caches. Intuitively it seems clear that a mix of
shared and distributed programming techniques would give code
that is optimally matched to the architecture. In this section we will
discuss such hybrid programming models, and discuss their efficacy.

A common setup of clusters uses distributed memory
\emph{nodes}\index{node}, where each node contains several
\emph{sockets}\index{socket}, that share memory. This suggests
using MPI to communicate between the nodes
(\indexterm{inter-node communication}) and OpenMP for parallelism on
the node (\indexterm{intra-node communication}).

In practice this is realized as follows:
\begin{itemize}
\item On each node a single MPI process is started (rather than one
  per socket or core);
\item This one MPI process than uses OpenMP (or another threading
  protocol) to spawn as many threads are there are independent sockets
  or cores on the node.
\item The OpenMP threads can then easily access the same shared memory.
\end{itemize}
The alternative would be to have an MPI process on each core or
socket, and do all communication through message passing, even between
processes that can see the same shared memory.

This hybrid strategy may sound like a good idea but the truth is complicated.
\begin{itemize}
\item Message passing between MPI processes sounds like it's more
  expensive than communicating through shared memory. However,
  optimized versions of MPI can typically detect when processes are on
  the same node, and they will replace the message passing by a simple
  data copy. The only argument against using MPI is then that each
  process has its own data space, so there is memory overhead because
  each process has to allocate space for buffers and duplicates of the
  data that is copied.
\item Threading is more flexible: if a certain part of the code needs
  more memory per process, an OpenMP approach could limit the number
  of threads on that part. On the other hand, flexible handling of
  threads incurs a certain amount of \ac{OS} overhead that MPI does
  not have with its fixed processes.
\item Shared memory programming is conceptually simple, but there can
  be unexpected performance pitfalls. For instance, the performance of
  two processes now
  can impeded by the need for maintaining
  \indextermsub{cache}{coherence} and by \indexterm{false sharing}.
\end{itemize}

On the other hand, the hybrid approach offers some advantage since it
bundles messages. For instance, if two MPI processes on one node send
messages to each of two processes on another node there would be four
messages; in the hybrid model these would be bundled into one
message.

\begin{exercise}
  Analyze the discussion in the last item above. Assume that the
  bandwidth between the two nodes is only enough to sustain one
  message at a time. What is the cost savings of the hybrid model over
  the purely distributed model? Hint: consider bandwidth and latency
  separately.
\end{exercise}

This bundling of MPI processes may have an advantage for a deeper
technical reason. In order to support a \indexterm{handshake protocol},
each MPI process needs a small amount of buffer space for each other process.
With a larger number of processes this can be a limitation, so bundling
is attractive on high core count processors such as the \indextermbus{Intel}{Xeon Phi}.

The MPI library is explicit about what sort of threading it supports:
you can query whether multi-threading is supported at all, whether all MPI 
calls have to originate from one thread or one thread at-a-time, or whether 
there is complete freedom in making MPI calls from threads.
