%%%%%%%%%%%%%%%%%%%%%%%%%%%%%%%%%%%%%%%%%%%%%%%%%%%%%%%%%%%%%%%%
%%%%%%%%%%%%%%%%%%%%%%%%%%%%%%%%%%%%%%%%%%%%%%%%%%%%%%%%%%%%%%%%
%%%%
%%%% This text file is part of the source of 
%%%% `Introduction to High-Performance Scientific Computing'
%%%% by Victor Eijkhout, copyright 2012
%%%%
%%%% This book is distributed under a Creative Commons Attribution 3.0
%%%% Unported (CC BY 3.0) license and made possible by funding from
%%%% The Saylor Foundation \url{http://www.saylor.org}.
%%%%
%%%%
%%%%%%%%%%%%%%%%%%%%%%%%%%%%%%%%%%%%%%%%%%%%%%%%%%%%%%%%%%%%%%%%
%%%%%%%%%%%%%%%%%%%%%%%%%%%%%%%%%%%%%%%%%%%%%%%%%%%%%%%%%%%%%%%%

If OpenMP is the way to program shared memory,
\acf{MPI}~\cite{mpi-reference} is the standard solution for
programming distributed memory. MPI (`Message Passing Interface') is a
specification for a library interface for moving data
between processes that do
not otherwise share data. The MPI routines can be divided roughly in
the following categories:
\begin{itemize}
\item Process management. This includes querying the parallel
  environment and constructing subsets of processors.
\item Point-to-point communication\index{point-to-point
  communication}. This is a set of calls where two processes
  interact. These are mostly variants of the send and receive calls.
\item\index{collective communication} Collective calls. In these
  routines, all processors (or the whole of a specified subset) are
  involved. Examples are the \indexterm{broadcast} call, where one
  processor shares its data with every other processor, or the
  \indexterm{gather} call, where one processor collects data from all
  participating processors.
\end{itemize}

Let us consider how the OpenMP examples can be coded in
MPI\footnote{This is not a course in MPI programming, and consequently
  the examples will leave out many details of the MPI calls. If you
  want to learn MPI programming, consult for
  instance~\cite{Gropp:UsingMPI1}.}. First of all, we no longer allocate
\begin{verbatim}
double a[ProblemSize];
\end{verbatim}
but
\begin{verbatim}
double a[LocalProblemSize];
\end{verbatim}
where the local size is roughly a $1/P$ fraction of the global
size. (Practical considerations dictate whether you want this
distribution to be as evenly as possible, or rather biased in some
way.)

The parallel loop
is trivially parallel, with the only difference that it now operates
on a fraction of the arrays:
\begin{verbatim}
for (i=0; i<LocalProblemSize; i++) {
  a[i] = b[i];
}
\end{verbatim}

However, if the loop involves a calculation based on the iteration
number, we need to map that to the global value:
\begin{verbatim}
for (i=0; i<LocalProblemSize; i++) {
  a[i] = b[i]+f(i+MyFirstVariable);
}
\end{verbatim}
(We will assume that each process has somehow calculated the values of
\n{LocalProblemSize} and \n{MyFirstVariable}.)
Local variables are now automatically local, because each process has
its own instance:
\begin{verbatim}
for (i=0; i<LocalProblemSize; i++) {
  t = b[i]*b[i];
  a[i] = sin(t) + cos(t);
}
\end{verbatim}
However, shared variables are harder to implement. Since each process
has its own data, the local accumulation has to be explicitly assembled:
\begin{verbatim}
for (i=0; i<LocalProblemSize; i++) {
  s = s + a[i]*b[i];
}
MPI_Allreduce(s,globals,1,MPI_DOUBLE,MPI_SUM);
\end{verbatim}
The `reduce' operation sums together all local values~\n{s} into a
variable \n{globals} that receives an identical value on each
processor. This is known as a \indexterm{collective operation}.

Let us make the example slightly more complicated:
\begin{verbatim}
for (i=0; i<ProblemSize; i++) {
  if (i==0)
    a[i] = (b[i]+b[i+1])/2
  else if (i==ProblemSize-1)
    a[i] = (b[i]+b[i-1])/2
  else
    a[i] = (b[i]+b[i-1]+b[i+1])/3
\end{verbatim}
If we had shared memory, we could write the following parallel code:
\begin{verbatim}
for (i=0; i<LocalProblemSize; i++) {
  bleft = b[i-1]; bright = b[i+1];
  a[i] = (b[i]+bleft+bright)/3
\end{verbatim}
To turn this into valid distributed memory code,
first we account for the fact that \n{bleft} and \n{bright} need to be
obtained from a different processor for \n{i==0} (\n{bleft}), and for
\n{i==LocalProblemSize-1} (\n{bright}). We do this with a exchange
operation with our left and right neighbour processor:
\begin{verbatim}
// get bfromleft and bfromright from neighbour processors, then
for (i=0; i<LocalProblemSize; i++) {
  if (i==0) bleft=bfromleft;
    else bleft = b[i-1]
  if (i==LocalProblemSize-1) bright=bfromright;
    else bright = b[i+1];
  a[i] = (b[i]+bleft+bright)/3
\end{verbatim}
Obtaining the neighbour values is done as follows. First we need to
ask our processor number, so that we can start a communication with
the processor with a number one higher and lower.
\begin{verbatim}
MPI_Comm_rank(MPI_COMM_WORLD,&myTaskID);
MPI_Sendrecv
   (/* to be sent:  */ &b[LocalProblemSize-1],
    /* destination  */ myTaskID+1,
    /* to be recvd: */ &bfromleft,
    /* source:      */ myTaskID-1, 
    /* some parameters omited */ 
   );
MPI_Sendrecv(&b[0],myTaskID-1,
    &bfromright, /* ... */ );
\end{verbatim}
There are still two problems with this code. First, the sendrecv
operations need exceptions for the first and last processors. This can
be done elegantly as follows:
\begin{verbatim}
MPI_Comm_rank(MPI_COMM_WORLD,&myTaskID);
MPI_Comm_size(MPI_COMM_WORLD,&nTasks);
if (myTaskID==0) leftproc = MPI_PROC_NULL;
  else leftproc = myTaskID-1;
if (myTaskID==nTasks-1) rightproc = MPI_PROC_NULL;
  else rightproc = myTaskID+1;
MPI_Sendrecv( &b[LocalProblemSize-1], &bfromleft,  rightproc );
MPI_Sendrecv( &b[0],                  &bfromright, leftproc);
\end{verbatim}

\begin{exercise}
  There is still a problem left with this code: the boundary
  conditions from the original, global, version have not been taken
  into account. Give code that solves that problem.
\end{exercise}

MPI gets complicated if different processes need to take
different actions, for example, if one needs to send data to
another. The problem here is that each process executes the same
executable, so it needs to contain both the send and the receive
instruction, to be executed depending on what the rank of the process
is.
\begin{verbatim}
if (myTaskID==0) {
  MPI_Send(myInfo,1,MPI_INT,/* to: */ 1,/* labeled: */,0,
    MPI_COMM_WORLD);
} else {
  MPI_Recv(myInfo,1,MPI_INT,/* from: */ 0,/* labeled: */,0,
    /* not explained here: */&status,MPI_COMM_WORLD);
}   
\end{verbatim}

\Level 2 {Blocking}
\label{sec:blocking}

Although MPI is sometimes called the `assembly language of parallel
programming', for its perceived difficulty and level of explicitness,
it is not all that hard to learn, as evinced by the large number of scientific
codes that use it. The main issues that make MPI somewhat intricate to
use, are buffer management and blocking semantics.

These issues are related, and stem from the fact that, ideally, data
should not be in two places at the same time. Let us briefly
consider what happens if processor~1 sends data to processor~2. The
safest strategy is for processor~1 to execute the send instruction,
and then wait until processor~2 acknowledges that the data was
successfully received. This means that processor~1 is temporarily
blocked until processor~2 actually executes its receive instruction,
and the data has made its way through the network. This is the
standard behaviour of the \n{MPI_Send} and \n{MPI_Recv} calls, which
are said to use \indextermsub{blocking}{communication}.

Alternatively,
processor~1 could put its data in a buffer, tell the system to make
sure that it gets sent at some point, and later checks to see that the
buffer is safe to reuse. This second strategy is called
\indexterm{non-blocking communication}, and it requires the use
of a temporary buffer.

\Level 2 {Collective operations}
\label{sec:mpi-collective}
\index{collective communication|(}

In the above examples, you saw the \n{MPI_Allreduce} call, which
computed a global sum and left the result on each processor. There is
also a local version \n{MPI_Reduce} which computes the result only on
one processor. These calls are examples of \emph{collective
  operations} or collectives. The collectives are:
\begin{description}
\item[reduction]: each processor has a data item, and these items need
  to be combined arithmetically with an addition, multiplication, max,
  or min operation. The result can be left on one processor, or on
  all, in which case we call this an {\bf allreduce} operation.
\item[broadcast]: one processor has a data item that all processors
  need to receive.
\item[gather]: each processor has a data item, and these items need
  to be collected in an array, without
  combining them in an operations such as an addition. The result can
  be left on one processor, or on all, in which case we call this an
  {\bf allgather}.
\item[scatter]: one processor has an array of data items, and each
  processor receives one element of that array.
\item[all-to-all]: each processor has an array of items, to be
  scattered to all other processors.
\end{description}
Collective operations are blocking (see section~\ref{sec:blocking}),
although MPI~3.0\index{MPI!MPI 3.0 draft}
(which is currently only a draft) will have
non-blocking collectives.
We will analyze the cost of collective operations in detail in
section~\ref{sec:collective-cost}.

\index{collective communication|)}

\Level 2 {Non-blocking communication}
\label{sec:nonblocking}

In a simple computer program, each instruction takes some time to
execute, in a way that depends on what goes on in the processor. In
parallel programs the situation is more complicated. A~send operation,
in its simplest form, declares that a certain buffer of data needs to
be sent, and program execution will then stop until that buffer has
been safely sent and received by another processor. This sort of
operation is called a \indexterm{non-local operation} since it depends
on the actions of other processes, and a \indexterm{blocking
  communication} operation since execution will halt until a certain
event takes place.

Blocking operations have the disadvantage that they can lead to
\indextermdef{deadlock}. In the context of message passing this
describes the situation that a process is waiting for an event that
never happens; for instance, it can be waiting to receive a message
and the sender of that message is waiting for something else.
You can easily recognize that deadlock occurs if two processes are waiting
for each other, or more generally, if you have a cycle of processes
where each is waiting for the next process in the cycle. Example:
\begin{verbatim}
if ( /* this is process 0 */ )
   // wait for message from 1
else if ( /* this is process 1 */ )
   // wait for message from 0
\end{verbatim}
A block receive here leads to deadlock.
Even without deadlock, they can lead to considerable \indexterm{idle
  time} in the processors, as they wait without performing any useful work.
On the other hand,  they have the advantage that it is clear when the
buffer can be reused: after the operation completes, there is a
guarantee that the data has been safely received at the other end.

The blocking behaviour can be avoided, at the cost of complicating the
buffer semantics, by using \indexterm{non-blocking communication} operations. A
non-blocking send (\n{MPI_Isend}) declares that a data buffer needs to
be sent, but then does not wait for the completion of the
corresponding receive. There is a second operation \n{MPI_Wait} that
will actually block until the receive has been completed. The
advantage of this decoupling of sending and blocking is that it now
becomes possible to write:
\begin{verbatim}
MPI_ISend(somebuffer,&handle); // start sending, and
    // get a handle to this particular communication
{ ... }  // do useful work on local data
MPI_Wait(handle); // block until the communication is completed;
{ ... }  // do useful work on incoming data
\end{verbatim}
With a little luck, the local operations take more time than the
communication, and you have completely eliminated the communication
time.

In addition to non-blocking sends, there are non-blocking receives. A
typical piece of code then looks like
\begin{verbatim}
MPI_ISend(sendbuffer,&sendhandle);
MPI_IReceive(recvbuffer,&recvhandle);
{ ... }  // do useful work on local data
MPI_Wait(sendhandle); Wait(recvhandle);
{ ... }  // do useful work on incoming data
\end{verbatim}

\begin{exercise}
  Take another look at equation~\eqref{eq:cyclic-add} and give pseudocode that
  solves the problem using non-blocking sends and receives. What is
  the disadvantage of this code over a blocking solution?
\end{exercise}

\Level 2 {MPI version 1 and 2}
\label{sec:mpi-1-2}

The first MPI standard~\cite{mpi-ref} had a number of notable
omissions, which are included in the MPI~2
standard~\cite{mpi-2-reference}. One of these concerned parallel
input/output: there was no facility for multiple processes to access
the same file, even if the underlying hardware would allow
that. A~separate project MPI-I/O has now been rolled into the MPI-2
standard. We will discuss parallel I/O in this book.

A second facility missing in MPI, though it was present in
\indexterm{PVM}~\cite{pvm-1,pvm-2} which predates MPI, is process
management: there is no way to create new processes and have them be
part of the parallel run.

Finally, MPI-2 has support for one-sided communication: one process
put data into the memory of another, without the receiving process
doing an actual receive instruction. We will have a short discussion
in section~\ref{sec:one-sided} below.


\Level 2 {One-sided communication}
\index{one-sided communication|(}
\label{sec:one-sided}

The MPI way of writing matching send and receive instructions is not
ideal for a number of reasons. First of all, it requires the
programmer to give the same data description twice, once in the send
and once in the receive call. Secondly, it requires a rather precise
orchestration of communication if deadlock is to be avoided; the
alternative of using asynchronous calls is tedious to program,
requiring the program to manage a lot of buffers.
Lastly, it requires a receiving processor to know how many incoming
messages to expect, which can be tricky in irregular applications.
Life would be so much easier if it was
possible to pull data from another processor, or conversely to put it
on another processor, without that other processor being explicitly
involved. 

This style of programming is further encouraged by the
existence of \indexacf{RDMA} support on some hardware. An early example was
the  \indextermbus{Cray}{T3E}. 
These days, one-sided communication is widely available through its
incorporation in the MPI-2 library; section~\ref{sec:mpi-1-2}.

Let us take a brief look at one-sided communication in MPI-2, using
averaging of array values as an example:
\[ \forall_i\colon a_i\leftarrow (a_i+a_{i-1}+a_{i+1})/3. \]
The MPI parallel code will look like
\begin{verbatim}
// do data transfer
a_local = (a_local+left+right)/3
\end{verbatim}
It is clear what the transfer has to accomplish: the \n{a_local}
variables needs to become the \n{left} variable on the processor with
the next higher rank, and the \n{right} variables on the one with the
next lower rank.

First of all, processors need to declare explicitly what memory area
is available for one-sided transfer, the so-called `window'. In this
example, that consists of the \n{a_local}, \n{left}, and \n{right}
variables on the processors:
\begin{verbatim}
MPI_Win_create(&a_local,...,&data_window);
MPI_Win_create(&left,....,&left_window);
MPI_Win_create(&right,....,&right_window);
\end{verbatim}
The code now has two options: it is possible to push data out
\begin{verbatim}
target = my_tid-1;
MPI_Put(&a_local,...,target,right_window);
target = my_tid+1;
MPI_Put(&a_local,...,target,left_window);
\end{verbatim}
or to pull it in
\begin{verbatim}
data_window = a_local;
source = my_tid-1;
MPI_Get(&right,...,data_window);
source = my_tid+1;
MPI_Get(&left,...,data_window);
\end{verbatim}
The above code will have the right semantics if the Put and Get calls
are blocking; see section~\ref{sec:blocking}. However, part of the
attraction of one-sided communication is that it makes it easier to
express communication, and for this, a non-blocking semantics is
assumed. 

The problem with non-blocking one-sided calls is that it becomes
necessary to ensure explicitly that communication is successfully
completed. For instance, if one processor does a one-sided \emph{put}
operation on another, the other processor has no way of checking that
the data has arrived, or indeed that transfer has begun at
all. Therefore it is necessary to insert a global barrier in the program,
for which every package has its own implementation. In MPI-2 the
relevant call is the 
\n{MPI_Win_fence} routine. These barriers in effect divide the program
execution in \indexterm{supersteps}; see section~\ref{sec:bsp}.

Another form of one-sided communication is used in the Charm++
package; see section~\ref{sec:charm++}.

\index{one-sided communication|)}
